\Chapter{REVUE DE LITTÉRATURE}\label{sec:RevLitt}

Le \emph{model-checking} rassemble des techniques permettant de vérifier
et valider des systèmes. Ces techniques se basent sur l'exploration
exhaustive d'un modèle du système, c'est à dire d'une représentation
abstraite et simplifiée de celui-ci. Historiquement, le model-checking a
tout d'abord été utilisé pour la vérification de composants
électroniques. Les techniques ont ensuite été adaptées afin de permettre
la validation de programmes. Nous allons par la suite nous intéresser à
ces techniques uniquement.

De nombreux algorithmes de model-checking ont été développés en fonction
des propriétés à vérifier sur le système, du formalisme dans lequel est
exprimé le modèle et de la complexité de celui-ci.

Dans cette partie, nous présentons les principales techniques de
model-checking logiciel et nous identifions leurs avantages et
inconvénients respectifs.

Pour chacune de ces techniques, nous identifions les outils les plus performant
les implémentant à l'heure actuelle. Nous restreignons cependant cette étude au cadre
qui nous intéressera pour la suite de ce rapport. Nous ne considèrerons donc que des
outils capables de vérifier des programmes concurrents en C, suivant le modèle de
d'exécution définis par la norme POSIX (pthread).

\textbf{Structure :} Dans la section~\ref{sec:model-checking-logiciel}, nous
définissons le model-checking logiciel et une représentation formelle des modèle
qui lui sont associés. Dans la section~\ref{sec:specification}, nous présentons
les types de propriétés des programmes vérifiables grâce au model-checking et
les formalismes utilisés pour les exprimer et les représenter formellement.
Enfin, dans la section~\ref{sec:techniques-et-outils-de-model-checking}, nous
présentons différentes techniques utilisées pour le model-checking de programmes
multi-thread ainsi que les outils qui les implémentent.
