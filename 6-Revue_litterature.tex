\Chapter{REVUE DE LITTÉRATURE}\label{sec:RevLitt}

Le \emph{model-checking} rassemble des techniques permettant de vérifier
et valider des systèmes. Ces techniques se basent sur l'exploration
exhaustive d'un modèle du système, c'est à dire d'une représentation
abstraite et simplifiée de celui-ci. Historiquement, le model-checking a
tout d'abord été utilisé pour la vérification de composants
électroniques. Les techniques ont ensuite été adaptées afin de permettre
la validation de programmes. Nous allons par la suite nous intéresser à
ces techniques uniquement.

De nombreux algorithmes de model-checking ont été développés en fonction
des propriétés à vérifier sur le système, du formalisme dans lequel est
exprimé le modèle et de la complexité de celui-ci.

Dans cette partie, nous présentons les principales techniques de
model-checking logiciel et nous identifions leurs avantages et
inconvénients respectifs.

Pour chacune de ces techniques, nous identifions les outils les plus performant
les implémentant à l'heure actuelle. Nous restreignons cependant cette étude au
cadre qui nous intéressera pour la suite de ce rapport. Nous ne considèrerons
donc que des outils capables de vérifier des programmes concurrents en C,
suivant le modèle de d'exécution définis par la norme POSIX (pthread).

\textbf{Structure :} Dans la section~\ref{sec:model-checking-logiciel}, nous
définissons le model-checking logiciel et une représentation formelle des modèle
qui lui sont associés. Dans la section~\ref{sec:specification}, nous présentons
les types de propriétés des programmes vérifiables grâce au model-checking et
les formalismes utilisés pour les exprimer et les représenter formellement.
Enfin, dans la section~\ref{sec:techniques-et-outils-de-model-checking}, nous
présentons différentes techniques utilisées pour le model-checking de programmes
multi-thread ainsi que les outils qui les implémentent.

\section{Model-checking logiciel}\label{sec:model-checking-logiciel}

Le model-checking permet de vérifier qu'un système respecte une
spécification donnée. Le principe sous-jacent est de mener une recherche
exhaustive à travers une modélisation du système plutôt que de
construire une preuve formelle de validité. Alors qu'une telle preuve
peut demander une certaine part d'intuition, le model-checking permet
une approche plus systématique, qui peut être plus facilement
automatisée.

\subsection{Modèle}


Le point caractéristique du model-checking est l'utilisation de
\emph{modèles} des systèmes à vérifier. Un modèle est une représentation
abstraite du système. Au lieu de vérifier un système directement, les techniques
de modèle checking s'appliquent sur le modèle du système. On considère un
système comme correct lorsque son modèle ne contient pas d'erreurs.

L'origine des modèle est liée au domaine d'application initial du
model-checking : la vérification du design de composants électroniques. Il
était nécessaire de représenter ces derniers de manière abstraite afin de les
vérifier, ce qui a donné naissance à la notion de modèles.

L'utilisation d'un modèle permet de simplifier les tâches de vérification : un
modèle bien conçu capture le fonctionnement du système en retirant les détails
n'ayant pas d'impact sur les propriétés à vérifier. Cette version simplifiée du
système permet ainsi de réaliser des tâches de vérification complexes, qui ne
pourraient aboutir autrement. Cependant, le modèle doit être suffisamment fidèle
pour préserver les propriétés à vérifier. Un modèle trop simple pourrait
provoquer l'apparition de faux positifs ou pire, ne pas contenir une erreur
pourtant bien présente dans le système.

La phase de modélisation est donc une phase délicate et propice aux erreurs. Il
est nécessaire de bâtir un équilibre entre l'abstraction du système et le
respect de son fonctionnement, cet équilibre dépendant des propriétés que l'on
souhaite vérifier. La modélisation demande donc à la fois une bonne connaissance
du système et des outils de model-checking utilisés. Les tendances actuelles
visent à réduire l'intervention humaine dans la phase de modélisation en
automatisant celle-ci.

\subsection{Système de transitions et trace d'exécution}

Un modèle prends souvent la forme d'un système de transitions.

\begin{description}
\item[Système de transitions]
  Formellement, un système de transitions est une paire \((S, \rightarrow)\),
  avec :

\begin{itemize}
\item
  \(S\) l'ensemble des états du système ;
\item
  \(\rightarrow \subset S \times S\) la relation de transitions du
  système ;
\end{itemize}
\end{description}

Intuitivement, les états d'un système de transitions représentent un
statut possible du système, tandis que les transitions représentent les
actions qui peuvent le faire évoluer.

\begin{description}
\item[Trace d'exécution]
Une trace d'exécution est un chemin dans le système de transition
représentant le modèle, possiblement infini, dont le premier état est
l'état initial du modèle. Une trace d'exécution est donc formé d'une
suite \(s_0, s_1, .. \in S\) d'états, avec \(s_0\) un état initial, et
d'une suite \(t_0, t_1, ... \in \rightarrow\) de transitions telles que
\(\forall i, t_i = (s_i, s_{i+1})\).
\end{description}

Le model-checking consiste à examiner l'ensemble des traces
d'exécution du modèle afin de vérifier si elles respectent la spécification.

\subsection{Model-checking logiciel}

Lorsque le système à vérifier est un programme, il est possible de
générer le modèle automatiquement à partir du code source du programme.
On parle dans ce cas de \emph{software model checking}, ou
model-checking logiciel. Générer le modèle à partir du code source (ou
même considérer directement que le modèle \emph{est} le code source)
permet de gagner sur deux plans :

Le model-checking logiciel, ou \emph{software model checking} est un cas
particulier de model-checking. Le système a vérifier est un programme, et le
modèle va être extrait automatiquement à partir de son code source.
On automatise ainsi la conception du modèle, ce qui présente les avantages
suivants :

\begin{itemize}
\item
  on réduit le risque d'une erreur lors de la modélisation du système,
  en supprimant l'intervention humaine ;
\item
  on automatise d'avantage le procédé de vérification, ce qui le rend
  plus facilement utilisable en pratique.
\end{itemize}

Cependant, le code doit encore être compilé afin d'obtenir le système,
ce qui constitue une encore une certaine distance entre le modèle et le
système. Afin de réduire cette distance, certains
model-checkers logiciels se basent sur des représentations intermédiaires
plus proches de l'assembleur, comme la représentation intermédiaire de
LLVM\footnote{L'utilisation de cette représentation est aussi expliquée
  par des raisons de compatibilité. De nombreux langages dont C, C++ ou
  C\# peuvent en effet être compilés vers la représentation
  intermédiaire de LLVM, un model-checker utilisant cette dernière est
  alors compatible avec tout ces langages.}.

La programmation impérative consiste à faire évoluer l'état du programme,
définis par la valeur de ses variables, à l'aide d'instruction. L'état d'un
programme est défini par l'état de la mémoire à laquelle il a accès. On peut
donc modéliser un programme par un système de transitions, dont les états sont
une partition de l'ensemble des états du programme.
Un état du modèle est donc principalement défini par :

\begin{itemize}
\item
  la configuration du tas du programme ;
\item
  la configuration de la pile de chaque thread ;
\item
  la valeur du pointeur d'instruction de chaque thread.
\end{itemize}

Les instructions permettant de faire évoluer l'état du programme sont
représentées par des transitions entres les états du modèle. Une transition
représente plus précisément l'action d'une instruction atomique du programme sur
les trois zones mémoires citées précédemment. Les transitions peuvent aussi
dépendre de facteurs externes, comme les entrées du programme.

Dans le cadre du model-checking logiciel, deux hypothèses sont fréquemment
faites sur le système :

\begin{itemize}
\item
  la consistance séquentielle. On considère alors que les écritures et
  les lectures mémoire ont lieu dans l'ordre où les instructions sont
  rencontrées dans une trace d'exécution. En pratique, cette hypothèse
  n'est pas toujours vérifiée. Les compilateurs et les processeurs
  peuvent modifier l'ordre de ces opérations pour optimiser un
  programme. On parle alors de modèle mémoire \emph{faible}.
\item
  la sémantique d'entrelacement. Dans le cas d'un système multi-thread,
  on considère que les traces d'exécution possibles sont produites par
  un entrelacement des instructions atomiques de chaque thread. Cette
  hypothèse n'est pas valide si le programme est exécuté par plusieurs
  processeurs : il est alors possible d'exécuter plusieurs instructions
  simultanément, ce qui peut produire des traces d'exécution
  supplémentaires.
\end{itemize}

\textbf{Exemple :} Considérons le programme suivant, formé de deux
threads.

\noindent\begin{minipage}{.45\textwidth}
\begin{lstlisting}[caption=Thread 1, frame=tlrb]
a <- 1
a <- 2
\end{lstlisting}
\end{minipage}\hfill
\begin{minipage}{.45\textwidth}
\begin{lstlisting}[caption=Thread 2,frame=tlrb]
c <- 1
d <- a + c
\end{lstlisting}
\end{minipage}

Ce programme contient trois variables, a, c et d, et deux threads. Les états du
systèmes sont donc définis par les valeurs de ces variables et la position des
pointeurs d'instruction de chaque thread.
On peut modéliser ce programme par le système de transitions en
Figure \ref{fig:model_example}.

\begin{figure}
\begin{center}
\includegraphics[width=\textwidth]{model_example}
\end{center}
\caption{Système de transition modélisant un programme simple.}
\label{fig:model_example}
\end{figure}

\subsection{Non-determinisme}\label{non-determinisme}

Un système est rarement complètement déterministe. De nombreux facteurs (les
données en entrée du système, l'instant où un évènement à lieu, la réussite des
allocations mémoire, la concurrence) tendent généralement à rendre l'évolution
d'un système non-deterministe.

Face à un choix non-déterministe, les techniques de model-checking doivent
explorer toute les alternatives possibles.

Lors de la modélisation, certaine sources de non-determinisme sont ignorées.
Elles sont alors remplacées par une exécution déterministe.
C'est généralement le cas pour les appels systèmes ayant la possibilité
d'échouer (les allocations de mémoire, par exemple).

Pour représenter les évènements et les paramètres extérieurs du programme,
une fonction simulant le non-determinisme est généralement mise à disposition
de l'utilisateur par les outils de model-checking.

\subsubsection{Explosion combinatoire}\label{explosion-combinatoire}

Le nombre d'états d'un programme augmente exponentiellement selon de nombreux
paramètres (nombre de variables du programme, taille des types de données, degré
de concurrence\dots). Il peut éventuellement être infinis si des appels de
fonctions ou de l'allocation dynamique de mémoire entre en jeu. Cependant, afin
de vérifier un système, un model-checker doit explorer l'ensemble de ces états.
Cette tache devient donc extrêmement coûteuse en temps et en mémoire quand le
nombre d'état devient trop important.
Ce problème, nommé l'explosion combinatoire, est la principale limite du
model-checking.

Le non-determinisme est une des principales raison de l'explosion combinatoire :
le nombre d'exécution est exponentiel selon le nombre de choix non-deterministe.

Cela explique les difficultés rencontrées par les techniques de model-checking
face aux programme concurrents : l'ordonnancement entre les instructions est
non-déterministe. Il existe un nombre d'entrelacements exponentiel en fonction
du nombre d'instructions du programme.

Les différents algorithmes de model-checking utilisent tous des
techniques afin de limiter l'explosion combinatoire. Elle reste
cependant le plus grand obstacle rencontré par le model-checking et
limite le passage à l'échelle de la plupart des techniques.
