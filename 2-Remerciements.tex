% Remerciements
%
%   Grâce aux remerciements, l'auteur attire l'attention du lecteur
% sur l'aide que certaines personnes lui ont apportée, sur leurs
% conseils ou sur toute autre forme de contribution lors de la
% réalisation de son mémoire. Le cas échéant, c'est dans cette section
% que le candidat doit témoigner sa reconnaissance à son directeur de
% recherche, aux organismes dispensateurs de subventions ou aux
% entreprises qui lui ont accordé des bourses ou des fonds de
% recherche.
\chapter*{REMERCIEMENTS}\thispagestyle{headings}
\addcontentsline{toc}{compteur}{REMERCIEMENTS}
%
Je tiens tout d'abord à remercier ma directrice de recherche, Hanifa Boucheneb,
sans qui ce travail n'aurait pas été possible. Ses conseils et ses remarques ont
constitué une aide précieuse. Un grand merci aussi à Amira Aouina, Laila Boumlik
ainsi que tous mes collègues et voisins de laboratoires pour vos suggestions et
le partage de votre expérience. Ils m'ont souvent permis d'avancer et de
progresser dans mes recherches. Je remercie enfin mes amis, ceux que j'ai
rencontrés à Montréal ou ceux qui m'ont soutenu de loin, ainsi que ma famille.
Vous m'avez aidé à toujours aller de l'avant.