\Chapter{CONCLUSION}\label{sec:Conclusion}

%%
%%  SYNTHESE DES TRAVAUX
%%
\section{Synthèse des travaux}

Dans ce mémoire, nous avons examiné l'état de l'art des techniques de
model-checking logiciel, en nous concentrant sur la vérification de programmes
concurrents en C, suivant la norme POSIX (pThread). Nous décrivons les
principales approches actuellement développés ainsi que les principaux projets
actifs implémentant ces approches. Nous identifions deux principales
limitations. L'explosion combinatoire reste, malgré les optimisations et les
techniques de réduction, le principal obstacle à surmonter afin de permettre au
model-checking logiciel de passer à l'échelle. De plus, les mécanismes de
spécification dans le cas de programmes concurrents sont peut développés : les
assertions sont peut adaptées à la concurrence, tandis que peu d'outils sont
capables de vérifier des propriétés \ac{LTL}, et toujours dans une forme
limitée.

Nous nous sommes alors intéressés aux limitations des formalismes de
spécifications. Nous avons mis en place un nouveau formalisme, basé sur
\ac{LTL}, et utilisant le concept de \emph{zones de validité} afin de permettre
de manipuler des positions du programmes et des variables locales dans les
propriétés \ac{LTL}. Nous vérifions que ce formalisme permet de représenter à la
fois la restriction de \ac{LTL} classiquement utilisée ainsi que des assertions,
tout en permettant de dépasser leurs principales limitations.

Enfin, nous présentons une transformation de source à source d'un programme en C
permettant de vérifier un code spécifié par notre formalisme. Nous contournons
le manque d'outils supportant le model-checking de propriétés LTL en reprenant
l'approche présenté par \cite{morse_ltl} et nous l'adaptons à notre cas. Nous
construisons ainsi un code instrumenté représentant le produit entre le code
d'origine et un automate de Büchi, spécifié par des assertions. Ce code
instrumenté peut ensuite être vérifié par un model-checker, sans nécessiter un
support de LTL. Cette solution reste cependant limitée à l'exploration de traces
finies. Nous avons testé cette instrumentation sur un jeu de tests, à l'aide des
model-checkers CBMC et ESBMC, ce qui nous a permis de valider notre approche. Il
est donc possible de vérifier des programmes spécifié dans notre formalisme à
l'aide de n'importe quel model-checker borné, d'autres techniques de
model-checking étant probablement utilisables aussi au prix de modification de
l'instrumentation ou d'une intégration de cette dernière directement dans le
model-checker. Cependant, notre approche n'est pas encore capable de passer à
l'échelle, car elle n'intègre pas de technique de réduction permettant de
limiter l'explosion combinatoire due à la concurrence.

%%
%%  LIMITATIONS
%%
% \section{Limitations de la solution proposée}\label{sec:Limitations}
% Déjà fait avant

%%
%%  AMELIORATIONS FUTURES
%%
\section{Améliorations futures}

Le formalisme de spécification que nous proposons permet d'exprimer un ensemble
de propriétés significativement plus large que celui exprimable par des
assertions. Afin de mesurer l'impact de ce formalisme, des tests plus conséquent
sur du code issus de système réels serait nécessaire. Une telle étude
permettrait aussi de déterminer si des extensions plus larges sont encore
nécessaires pour exprimer des propriétés fréquemment utilisées.

Une autre piste de travail intéressante serait d'intégrer la procédure de
vérification de notre instrumentation directement dans un outil de
model-checking. On perdrait alors les avantages d'une instrumentation, mais il
serait possible de gagner en performances et de contourner un certain nombre des
limitations actuelles de baProduct. Améliorer les performances pourrait aussi
passer par l'intégration de techniques de réduction.

Enfin, il serait possible d'étudier comment tirer parti de l'instrumentation dans
d'autres contextes de vérification. En particulier, l'instrumentation donne
accès à l'automate de Büchi. En le modifiant, il est possible de restreindre les
exécutions explorées par un model-checker et se concentrer sur des scénarios
critiques. L'instrumentation pourrait aussi être modifiée afin d'être rendue
exécutable, quitte à remplacer les choix non-déterministes par des choix
aléatoires. Cela ouvre la porte à des approches de vérification de notre
spécification en utilisant des outils de test (stress testing ou fuzzing par
exemple).

\section{Remarques conclusives}

A travers ce mémoire, nous avons contribué à étendre légèrement les possibilités
de spécification pour le model-checking de programmes concurrents. Cependant,
ces progrès viennent avec un coût de vérification plus important. Ces coûts de
vérification, liés en grande partie à l'explosion combinatoire, constituent la
principale limite du model-checking actuellement. Ce problème ne semble pas
proche d'être résolu, à moins d'une percée spectaculaire dans ce domaine : on
assiste plutôt à un ensemble d'optimisation qui permettent de vérifier des
programmes de taille toujours plus grande. De manière concomitante, de nouveaux
langages de programmation émergent, axés sur la sécurité et permettant de
prouver certaines propriétés d'un programme à sa compilation. Un exemple est
Rust\ref{Rust}, qui permet d'assurer l'absence de data-race et une certaine
forme de consistance entre les threads. Le model-checking pourrait profiter des
propriétés assurées par ces langages afin de réduire d'avantage la taille des
modèles.