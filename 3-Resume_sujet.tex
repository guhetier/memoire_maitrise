% Résumé du mémoire.
%
%   Le résumé est un bref exposé du sujet traité, des objectifs visés,
% des hypothèses émises, des méthodes expérimentales utilisées et de
% l'analyse des résultats obtenus. On y présente également les
% principales conclusions de la recherche ainsi que ses applications
% éventuelles. En général, un résumé ne dépasse pas quatre pages.
%
%   Le résumé doit donner une idée exacte du contenu du mémoire ou de la thèse. Ce ne
% peut pas être une simple énumération des parties du document, car il
% doit faire ressortir l'originalité de la recherche, son aspect
% créatif et sa contribution au développement de la technologie ou à
% l'avancement des connaissances en génie et en sciences appliquées.
% Un résumé ne doit jamais comporter de références ou de figures.
\chapter*{RÉSUMÉ}\thispagestyle{headings}
\addcontentsline{toc}{compteur}{RÉSUMÉ}

% Le résumé est un bref exposé du sujet traité, des objectifs visés,
% des hypothèses émises, des méthodes expérimentales utilisées et de
% l'analyse des résultats obtenus. On y présente également les
% principales conclusions de la recherche ainsi que ses applications
% éventuelles. En général, un résumé ne dépasse pas quatre pages.

% Le résumé doit donner une idée exacte du contenu du mémoire ou de la thèse. Ce ne
% peut pas être une simple énumération des parties du document, car il
% doit faire ressortir l'originalité de la recherche, son aspect
% créatif et sa contribution au développement de la technologie ou à
% l'avancement des connaissances en génie et en sciences appliquées.
% Un résumé ne doit jamais comporter de références ou de figures.

L'utilisation croissante de systèmes informatiques critiques et l'augmentation
de leur complexité rendent leur bon fonctionnement essentiel. Le model-checking
logiciel permet de prouver formellement l'absence d'erreurs dans un programme.
Il reste cependant limité par deux facteurs : l'explosion combinatoire et la
capacité à spécifier le comportement correct d'un programme. Ces deux problèmes
sont amplifiés dans le cas de programmes concurrents, à cause des différents
entrelacements possibles entre les fils d'exécutions. Les assertions et les
formules logiques LTL sont les deux formalisme de spécification les plus
utilisés dans le cadre du model-checking logiciel, cependant, ils sont limité
dans un contexte concurrent : les assertions ne permettent pas d'exprimer des
relations temporelles entres différents fils d'exécutions alors que les formules
LTL utilisées par les outils actuels ne permettent pas d'exprimer des propriétés
sur les variables locales et les positions dans le code du programme. Ces
notions sont pourtant importante dans le cas de programme concurrents.

Dans ce mémoire, nous établissons un formalisme de spécification visant à
corriger ces limitations. Ce formalisme englobe LTL et les assertions, en
permettant d'exprimer des relations temporelles sur des propositions faisant
intervenir les variables locales et globales d'un programme ainsi que les
positions dans le code source. Nous présentons aussi un outil permettant de
vérifier une spécification exprimée dans ce formalisme dans le cas d'un
programme concurrent codé en C.

Notre formalisme se base sur la logique LTL. Il permet de surmonter deux des
principales limitation rencontrée par les variantes de LTL utilisées par les
outils de model-checking logiciel : manipuler des positions dans le code et des
variables locales dans les propositions atomiques. La principale difficulté est
de définir correctement dans l'ensemble du programme une proposition atomique
lorsqu'elle dépend d'une variable locale. En effet, cette variable locale n'a de
sens que dans une partie limitée du programme. Nous résolvons ce problème en
utilisant le concept de \emph{zones de validité}. Une zone de validité est un
intervalle de positions dans le code source du programme dans laquelle la valeur
d'une proposition atomique est définie à l'aide de sa fonction d'évaluation.
Une valeur par défaut est utilisée hors de la zone de validité. Ceci permet de
limiter l'utilisation de la fonction d'évaluation aux contextes où tous ses
paramètres locaux sont définis.

Nous avons développé un outil, \texttt{baProduct}, afin de vérifier un programme
concurrent en C spécifié selon notre formalisme. Cet outil prend la forme d'une
transformation de source à source, et délègue les tâches de vérifications à un
model-checker utilisé en arrière plan. Ce choix permet de simplifier grandement
le développement de l'outil, mais aussi de profiter des performances de
model-checkers reconnus et de les comparer. À partir du code source d'un
programme et d'une spécification, \texttt{baProduct} produit un programme
spécifié par des assertions dont la validité correspond au respect de la
spécification initiale. Plusieurs model-checkers peuvent être utilisés en
arrière plan, la principale condition étant qu'ils supportent les spécifications
sous forme d'assertions. La transformation de source à source se base sur
l'implémentation en C de l'automate de Büchi associé à la spécification et la
construction de son produit avec le code source du système.

Nous appliquons ensuite \texttt{baProduct} sur un ensemble de tests simples.
Nous comparons les performances obtenues à l'aide de deux model-checkers
utilisés en arrière plan, CBMC et ESBMC. Nous obtenons généralement des
meilleures performances avec CBMC en raison d'une différence de la gestions des
entrelacements par ces outils. Ces tests nous permettent de confirmer qu'il est
possible de spécifier des propriétés difficiles à exprimer par des assertions ou
une version plus classique de LTL (comme des propriétés d'exclusion mutuelle),
de manière efficace et sans qu'une instrumentation manuelle soit nécessaire.
Nous vérifions aussi que notre transformation de source à source permet de
ramener le problème de la vérification de propriétés LTL à la vérifications
d'assertions dans le cadre de traces finies. Cependant, nous mettons aussi en
évidence un certain nombre de limitations. En particulier, notre transformation
est limitée à des traces finies. De plus, notre outil n'implémente pas de
méthode de réduction, ce qui nuit à ses performances.