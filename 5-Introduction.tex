% Dans l'introduction, on présente le problème étudié et les buts
% poursuivis. L'introduction permet de faire connaître le cadre de la
% recherche et d'en préciser le domaine d'application. Elle fournit
% les précisions nécessaires en ce qui concerne le contexte de
% réalisation de la recherche, l'approche envisagée, l'évolution de
% la réalisation. En fait, l'introduction présente au lecteur ce
% qu'il doit savoir pour comprendre la recherche et en connaître la
% portée.

\chapter{INTRODUCTION}\label{sec:Introduction}  % 10-12 lignes pour introduire le sujet.

Les systèmes informatiques ont pris une place sans précédent dans la société
actuelle. D'une part, ils sont devenus omniprésents à l'échelle personnelle et
dans le monde économique. Les smartphones et les ordinateurs personnels sont
entrés dans notre quotidien. L'informatique embarquée est utilisée massivement,
que ce soit dans les véhicules ou les usines. À travers internet, nous
interagissons en permanence avec des programmes sur des serveurs distants.

Simultanément, le champ d'action des systèmes informatiques s'est étendu. En
plus de leur rôle dans les systèmes embarqués et les communications, les
logiciels ont maintenant un large accès aux données personnelles de leurs
utilisateurs. À travers la domotique, ils peuvent agir sur notre environnement
immédiat. Les systèmes de recommandation et d'intelligence artificielle
influencent les prises de décisions.

Dans ce contexte, les dysfonctionnements des systèmes informatiques sont un
sujet de préoccupation. Une erreur peut avoir des conséquences de grande
ampleur, un coût important et porter atteinte à la sécurité des informations et
des personnes. Il est cependant difficile de concevoir un système correct. On
peut s'en convaincre en consultant la liste des bugs et des failles
informatiques apparus ces dernières années. \cite{horror_story} présente
certains des dysfonctionnements ayant eu les conséquences les plus importantes.
Les exemples les plus connus restent sans doute l'explosion d'Ariane 5, due à
une taille de variable sous-dimensionnée, et la faille HearthBleed dans la
bibliothèque de cryptographie OpenSSL.

Vérifier les systèmes informatiques a ainsi gagné une importance qui n'est
plus discutable. Cependant, l'inspection manuelle des logiciels est une
tâche longue, coûteuse et souvent source d'erreurs. Sur les systèmes les
plus complexes, elle devient irréalisable. Il est donc nécessaire de se
reposer sur des techniques et des outils permettant d'automatiser,
partiellement ou totalement, le processus de vérification.

On peut séparer les techniques et les outils utilisés pour vérifier un
logiciel selon plusieurs critères. L'un de ces critères est le degré
d'automatisation. Les outils peuvent aller d'un système totalement
autonome à un simple assistant pour l'ingénieur responsable de la tache de
vérification. L'automatisation est un facteur important pour la
vérification de systèmes de grande taille. Le risque d'une erreur
humaine dans un processus fortement automatisé est aussi réduit. Un
second critère est la rigueur du verdict fourni par la technique de
vérification. Certaines méthodes (les méthodes formelles) permettent
d'obtenir une preuve de validité tandis que d'autres (comme le test ou
certaines méthodes d'analyse statiques) permettent seulement d'augmenter
son degré de confiance en la validité du programme.

Dans ce mémoire, nous allons nous intéresser aux techniques de
vérification automatisées et fournissant une garantie de qualité
rigoureuse du programme à vérifier. Nous allons en particulier nous
concentrer sur les techniques de model-checking. Nous allons tout
d'abord rapidement présenter quelques-unes des méthodes de vérifications
existantes afin de dresser le contexte dans lequel le model-checking
s'inscrit.

%%
%%  CONCEPTS DE BASE
%%
\section{Définitions et concepts de base}  % environ 2-3 pages

\subsection{Le test}

La méthode de vérification la plus employée, de loin, est le test. Un
programme ou un système informatique a généralement pour objectif de
réaliser une tâche précise, en respectant des critères de qualité et de
performances. Un test permet, étant donné un environnement et des
paramètres d'entrée pour le programme, de vérifier s’il produit le
résultat attendu (indiqué par un oracle).

La grande majorité des projets de développement maintiennent ainsi des
suites de tests afin de s'assurer que le programme se comporte de la
manière attendue. Différents paramètres et scénarios d'exécution sont
testés afin de détecter autant d'erreurs que possible.

Ces suites de tests sont gérées de manière automatisée et sont étendues au
tout au long du développement d'un logiciel. Ils peuvent prendre des
dimensions considérables. Par exemple, dans le cas de SQLite, les
suites de tests représentent près de cent millions de lignes, soit plus
de 700 fois le nombre de lignes de la bibliothèque elle-même
(https://www.sqlite.org/testing.html).

Cependant, les tests souffrent cependant d'un certain nombre de
défauts. Tout d'abord, il est nécessaire d'avoir une bonne connaissance
du système et de faire preuve d'une certaine imagination afin de
concevoir les scénarios susceptibles de mener à une erreur. Lorsque
c'est le cas, il est souvent complexe de remonter jusqu'à la source de
l'erreur afin de la corriger. Les tests rencontrent aussi un problème de
couverture : il est extrêmement difficile de produire un test pour
chaque comportement du système. Ce problème est accru si l'on passe à
l'échelle : le nombre de comportements d'un système augmente
exponentiellement avec sa taille, le nombre de tests nécessaire pour
couvrir l'ensemble des comportements devrait donc faire de même. Créer, maintenir et
corriger les tests et les oracles représente alors un investissement conséquent.

Les tests assurent donc généralement une couverture partielle du
programme à vérifier. Une suite de tests ne permet donc pas de garantir
l'absence d'erreurs, mais seulement l'absence d'erreurs dans les chemins
d'exécutions testés.

Différents critères de couvertures existent afin de mesurer la qualité
d'une suite de tests. On peut demander que toutes les fonctions du
programme soient exécutées au moins une fois lors des tests (on parle
alors de \emph{function coverage}), ou que l'ensemble des instructions
du programme soit exécuté (\emph{statement coverage}). On peut aussi
demander à une suite de tests de contenir des exécutions sollicitant
successivement différentes instructions du programme.

Les tests sont généralement moins performants dans le cas de programmes
concurrents. L'entrelacement entre les instructions peut avoir un impact
sur l'issue du test, et cet ordre change de manières non déterministes
entre les exécutions. Une erreur peut alors être détectée lors d'une
exécution des tests, mais être très difficile à reproduire. Pour tenter
de détecter plus efficacement les erreurs liées à la concurrence, des
techniques comme le stress-testing ont été mises en place. Elles
consistent à ajouter du bruit à l'exécution des tests ou à lancer
plusieurs instances du test en parallèle afin d'augmenter la probabilité
des entrelacements d'instructions les moins fréquents.

Malgré ces limitations, le test reste généralement la méthode de
vérification la plus simple à mettre en place. Les autres techniques de
vérification sont parfois utilisées de manière complémentaire à une
suite de tests, afin d'apporter une preuve de correction plus rigoureuse
sur les fragments critiques du programme.

\subsection{Les méthodes formelles}

Les méthodes de test permettent de trouver des erreurs dans un programme
et d'augmenter la confiance des développeurs dans le fait que le
programme est correct, mais elles ne permettent pas de prouver l'absence
d'erreurs.

Les méthodes de vérification formelles sont une réponse à ce défaut :
elles permettent de prouver l'absence d'erreurs dans un programme,
quels que soient les entrées et les chemins d'exécution.

En pratique, vérifier qu'un système est complètement correct (c.-à-d il ne contient aucune erreur, que ce soit des erreurs
fonctionnelles ou d'exécution) est généralement trop complexe. Il est
cependant possible de garantir l'absence de certains types d'erreurs et de
vérifier des propriétés critiques pour le système.

De nombreuses méthodes existent afin de prouver la correction d'un
programme vis-à-vis d'une propriété. Nous allons ici nous intéresser aux
méthodes automatiques, qui ne requièrent pas (ou peu) d'interaction
humaine pour fournir un résultat. En particulier, nous excluons les
méthodes basées sur des assistants de preuve (comme Coq ou HOL), qui
demandent un effort significatif et des connaissances avancées à
l'utilisateur afin de construire une preuve.

\subsubsection{L'interprétation abstraite}

L'interprétation abstraite fait partie du domaine de l'analyse statique.
L'analyse statique regroupe les techniques permettant d'obtenir des
informations sur le comportement d'un programme à partir de son code
source, sans l'exécuter.

Le théorème de Rice indique que la plupart des propriétés d'un programme
ne sont pas décidables. L'essence de l'interprétation abstraite est de
contourner ce problème en construisant une approximation du résultat de
manière efficace et correcte. Le principal enjeu de
l'interprétation abstraite est d'être suffisamment précise afin de ne
pas indiquer un trop grand nombre de fausses (\emph{spurious}) erreurs.

Plutôt que d'essayer de suivre l'ensemble des valeurs possibles pour
chaque variable, à chaque point du programme, l'interprétation abstraite
utilise un domaine abstrait qui permet d'approximer l'ensemble des
valeurs qu'une variable peut prendre. Par exemple, on peut approximer
l'ensemble des valeurs d'une variable par le plus petit intervalle
contenant l'ensemble de ces valeurs. Le programme est alors interprété,
chaque opération concrète sur les variables étant transposée dans le
domaine abstrait. L'utilisation des domaines abstraits permet de réduire
largement le nombre de valeurs à manipuler et facilite les opérations.

On peut alors prouver la correction du programme vis-à-vis d'une
propriété en l'évaluant dans le domaine abstrait. Pour poursuivre
l'exemple précédent, supposons que l'on veut s'assurer qu'une variable n'est jamais
nulle à un certain point du programme (pour éviter une erreur arithmétique
par exemple). Si l'analyse permet d'établir qu'elle se situe dans
l'intervalle \([1, 5]\), on peut conclure que la propriété est vérifiée.
Cependant, si l'on obtient l'intervalle \([-1, 1]\), il n'est pas possible
de conclure : une erreur est possible, mais elle n'est pas certaine.

En pratique, l'interprétation abstraite est souvent utilisée pour
vérifier des propriétés génériques, intégrées dans les outils
(\emph{built-in}). C'est une technique qui
passe bien à l'échelle et qui peut analyser de larges bases de code. Des
propriétés typiquement vérifiées sont l'absence d'erreurs arithmétiques,
l'absence de variables utilisées avant d'être initialisées, ou le
respect des bornes des tableaux.

Cependant, l'interprétation abstraite souffre de deux principaux défauts
: tout d'abord, les approximations réalisées peuvent provoquer de
nombreux faux positifs, signalant une erreur éventuelle dans un code
correct. La précision de l'analyse peut être parfois être améliorée en
choisissant un domaine abstrait plus raffiné, mais cela a généralement
un coût au niveau des performances. De plus, l'interprétation abstraite
ne permet pas d'obtenir un contre-exemple menant vers une erreur
indiquée, ou toute autre information indiquant la cause de cette erreur.
Il peut alors être complexe de déterminer si une erreur reportée est un
faux positif ou non, et de comprendre sa cause pour la corriger.

\subsubsection{Le model-checking}

Les techniques de model-checking se basent sur l'exploration d'un modèle
du système à vérifier. Ce modèle prend généralement la forme d'un
système de transition, il peut être créé manuellement ou extrait
automatiquement du système. Les outils de model-checking vont alors
explorer la partie accessible du modèle de manière exhaustive.

Le modèle doit à la fois représenter fidèlement le système vis-à-vis des
propriétés à vérifier, et être suffisamment simple pour qu'il soit
possible de l'explorer en un temps raisonnable. S’il est possible
d'atteindre un état, ou d'emprunter une succession d'états ne respectant
pas la propriété désirée, une erreur est reportée. Un contre-exemple ---
une trace d'exécution menant à cette erreur --- peut alors être généré.
Il a en général une grande valeur pour comprendre la source de l'erreur
et la corriger, et constitue un des plus grands atouts du model-checking.

La distinction entre le model-checking et l'interprétation abstraite est
principalement historique. L'analyse statique a été conçue pour vérifier
des propriétés simples sur les programmes. Les algorithmes ont souvent
mis l'accent sur les performances quitte à perdre en précision. D'autre
part, le model-checking a été développé pour prouver des propriétés
complexes sur le design de circuits. La priorité a été la précision des
algorithmes et la capacité à vérifier des propriétés liées au
programme. Cependant, les analyseurs statiques ont désormais gagné en
précisions et sont capables vérifier des spécifications. En parallèle, les
model-checkers ont développé des abstractions pour gagner en
performances. La distinction reste donc principalement une question
d'intention dans l'orientation de l'outil.

Le model-checking reste limité par le problème de \emph{l'explosion
combinatoire}. La taille des modèles et la durée des taches de
vérification augmentent généralement de manière exponentielle avec la
complexité du système et des propriétés à vérifier.

\subsection{Concurrence}

Les lois de Moore sont des conjectures empiriques qui prédisent que la
puissance de calcul des processeurs double chaque année. Cependant, la
miniaturisation des transistors approche ces limites physiques,
annonçant la fin des lois de Moore dans leur interprétation
traditionnelle. Cependant, afin continuer à gagner en puissance de
calcul, les processeurs ont commencé à évoluer dans une autre direction
: plutôt que d'augmenter la puissance, c'est le nombre de cœurs qui va
croître. Pour tirer partit de ces processeurs multicœurs, il est
cependant nécessaire de réaliser des programmes parallèles.

Plusieurs modèles de programmation concurrente existent. Par exemple, ils
peuvent être basés sur des évènements (SystemC), ou sur des algorithmes
de type map / reduce (MPI). Dans ce mémoire, nous allons nous concentrer
sur les threads suivant la norme POSIX (\emph{pthread}).

La concurrence introduit tout un nouveau panel de difficultés dans la
conception d'un programme. Les accès à la mémoire partagée par plusieurs
fils d'exécutions doivent être protégés pour éviter les \emph{race conditions}, qui peuvent fausser le résultat d'une opération. Il est
aussi nécessaire de synchroniser les fils d'exécution et de tenir compte
des différents entrelacements possibles entre les instructions. Ces
difficultés provoquent souvent l'apparition d'erreurs subtiles. En
particulier, les erreurs dépendant de l'entrelacement des instructions
sont souvent complexes à détecter et à diagnostiquer. L'entrelacement
étant non déterministe, ces erreurs sont souvent difficiles à détecter
et à reproduire.

Les programmes concurrents représentent un défi pour les méthodes de
vérification. Les méthodes reposant sur l'exécution du programme, comme
le test, n'ont pas de contrôle sur l'entrelacement des instructions. Il
devient alors difficile de tester certains chemins critiques et de
diagnostiquer les erreurs rencontrées. Les techniques de model-checking
possèdent ce contrôle --- elles ont initialement été conçues pour vérifier
des systèmes électroniques concurrents. Cependant, dans le cas du
model-checking logiciel, la concurrence amplifie l'explosion
combinatoire de manière exponentielle. Il est nécessaire d'explorer tous
les entrelacements entre les instructions d'un programme. Or, pour un
programme composé de \(n\) threads, contenant \(k\) instructions chacun,
il existe \(\frac{(nk)!}{(k!)^n}\) entrelacements. Pour \(k = 100\)
instructions et \(n = 2\) threads, on obtient déjà près de \(10^{59}\)
entrelacements !

\section{Domaine de ce mémoire}

Dans ce mémoire, nous allons restreindre le champ de notre étude aux
outils de model-checking pour des programmes écrits en C.

Le langage C est un langage bas niveau parmi les plus utilisés et les
plus présents dans les bases de code actuelles. Il est en particulier
utilisé dans les applications embarquées, les applications systèmes et
les applications temps réel, qui nécessitent un contrôle bas niveaux et
des performances élevées. Ces types d'applications sont souvent
critiques et ne peuvent tolérer un dysfonctionnement. Le C est donc un
langage privilégié par les méthodes de vérification formelle, supporté
par de nombreux outils de model-checking.

Quand il sera question de concurrence, nous nous intéresserons en
particulier aux programmes multi threads suivant la norme POSIX, et
utilisant la bibliothèque \emph{pthread}. Elle implémente une gestion
bas niveau de la concurrence, souvent utilisée dans les programmes
embarqués ou les codes systèmes.

Deux questions ouvertes forment actuellement le cœur de la recherche
sur le model-checking. La première, et la principale, est la question
des performances et de la lutte contre l'explosion combinatoire. La
seconde est la question de la spécification, les propriétés qu'il est
possible de vérifier sur un système, et la manière de les exprimer. Dans
ce mémoire, nous allons nous intéresser à ce second point.

Les assertions sont le mécanisme de spécification le plus utilisé dans
le cadre de la vérification logicielle. Une assertion prend la forme
d'une instruction dans le code, prenant en paramètre une expression
booléenne pure --- sans effet de bord --- qui joue le rôle d'un
invariant. Si cet invariant n'est pas respecté, l'assertion est
déclenchée (ce qui met fin au programme dans le cas d'une exécution, ou
permet de faire remonter une erreur dans le cas du model-checking). La
plupart des outils de model-checking logiciel supportent une
spécification sous la forme d'assertion.

Cependant, dans le cas de programmes multi threads, les assertions
présentent des faiblesses. En particulier, l'invariant d'une assertion
ne peut utiliser des valeurs hors de son contexte (une variable d'un
autre thread par exemple). De plus, une assertion est bloquante :
l'exécution ou l'exploration du programme stoppe une fois une assertion
atteinte. Il n'est pas possible d'exprimer des relations temporelles
(simultanéité ou succession par exemple) entre des assertions pour
déclencher une erreur dans ces cas seulement.

Ma question de recherche sera donc composée des deux parties suivantes :

\begin{itemize}
\item
  Comment peut-on améliorer la spécification des programmes concurrents
  et la rendre plus expressive ?
\item
  Comment peut-on rendre les outils de model-checking actuels
  compatibles avec une spécification plus expressive ?
\end{itemize}

Pour améliorer l'expressivité de la spécification des programmes
concurrents, nous allons nous intéresser aux logiques temporelles, et en
particulier aux formules \ac{LTL}. Ces dernières ont déjà été utilisées dans
le cadre de model-checking logiciel, dans un cadre restreint. Nous
proposons une extension de ce cadre, permettant en particulier de faire
référence à des positions du code et à des variables locales dans les
formules. En particulier, notre extension permet d'englober les
assertions classiques et de créer des relations temporelles entre ces
dernières.

Les outils de model-checking sont généralement complexes et optimisés
pour le type de propriétés qu'ils prennent en charge. Afin de rendre
les outils existants compatibles avec notre spécification, nous mettons
en place une transformation de source à source du système. Nous
construisons ainsi un nouveau programme spécifié par des assertions,
dont la validité est équivalente à celle du système d'origine. Cette
méthode nous permet en particulier de tester notre approche avec
différents outils et de profiter de leurs optimisations.

%%
%% PLAN DU MEMOIRE
%%
\section{Plan du mémoire}  % 0.5 page

La partie \ref{sec:RevLitt} de ce mémoire consiste en un état de l'art. Nous présentons
d'abord les spécifications à base d'assertion et la logique \ac{LTL}. Nous
introduisons ensuite les principaux algorithmes de model-checking
logiciels, ainsi que les outils les implémentant et leurs spécificités.

Dans la partie \ref{sec:Theme1}, nous détaillons le formalisme de spécification que
nous avons établi. Nous le comparons aux autres tentatives allant dans
le même sens et nous mettons en lumière les principaux problèmes
résolus ou encore ouverts.

Dans la partie \ref{sec:Theme2}, nous présentons une transformation de source à source
destinée à rendre les outils existants compatibles avec notre
spécification. Nous construisons pour cela le produit entre le système
et l'automate de Büchi associé à la propriété à vérifier. Nous évaluons
ensuite les performances obtenues pour la vérification d'exemples
simples.
