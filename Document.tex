\documentclass[letterpaper,12pt,oneside,final]{book}
%%
%%  Template de mémoire de maîtrise ou thèse de doctorat.
%%  Normalement, il n'est pas nécessaire de modifier ce document
%%  sauf pour changer les noms des fichiers à inclure.
%%
%%  Version: 2014-10-28
%%
%%  Accepte les caractères accentués dans le document (UTF-8).
\usepackage[utf8]{inputenc}
%%
%% Support pour l'anglais et le français (français par défaut).
%\usepackage[cyr]{aeguill}
\usepackage{lmodern}      % Police de caractères plus complète et généralement indistinguable visuellement de la police standard de LaTeX (Computer Modern).
\usepackage[T1]{fontenc}  % Bon encodage des caractères pour qu'Acrobat Reader reconnaisse les accents et les ligatures telles que ffi.

%% Utilisation de natbib pour les citations et la bibliographie.
\usepackage[numbers]{natbib}

\usepackage[english,french]{babel} % le langage par défaut est le dernier de la liste, c'est-à-dire français
%%
%% Charge le module d'affichage graphique.
\usepackage{graphicx}
\usepackage{epstopdf}  % Permet d'utiliser des .eps avec pdfLaTeX.
%%
%% Recherche des images dans les répertoires.
\graphicspath{{./images/}{./dia/}{./gnuplot/}}
%%
%% Un float peut apparaître seulement après sa définition, jamais avant.
\usepackage{flafter,placeins}
%%
%% Autres packages.
\usepackage{amsmath,color,soulutf8,longtable,colortbl,setspace,ifthen,xspace,url,pdflscape}
%%
%% Support des acronymes.
\usepackage[nolist]{acronym}
\onehalfspacing                % Interligne 1.5.
%%
%% Définition d'un style de page avec seulement le numéro de page à
%% droite. On s'assure aussi que le style de page par défaut soit
%% d'afficher le numéro de page en haut à droite.
\usepackage{fancyhdr}
\fancypagestyle{pagenumber}{\fancyhf{}\fancyhead[R]{\thepage}}
\renewcommand\headrulewidth{0pt}
\makeatletter
\let\ps@plain=\ps@pagenumber
\makeatother
%%
%% Module qui permet la création des bookmarks dans un fichier PDF.
%\usepackage[dvipdfm]{hyperref}
\usepackage{hyperref}
\usepackage{caption}  % Hyperlien vers la figure plutôt que son titre.
\makeatletter
\providecommand*{\toclevel@compteur}{0}
\makeatother
%%
%% Définitions spécifiques au format de rédaction de Poly.
\usepackage{MemoireThese}
%%
%% Définitions spécifiques à l'étudiant.
%% -----------------------------------
%% ---> A MODIFIER PAR L'ETUDIANT <---
%% -----------------------------------
%%
%% Commandes qui affichent le titre du document, le nom de l'auteur, etc.
\newcommand\monTitre{Titre de mon document}
\newcommand\monPrenom{Jules}
\newcommand\monNom{C\ae{}sar}
\newcommand\monDepartement{génie informatique et génie logiciel}
\newcommand\maDiscipline{génie informatique}
\newcommand\monDiplome{D}        % (M)aîtrise ou (D)octorat
\newcommand\anneeDepot{2010}
\newcommand\moisDepot{avril}
\newcommand\monSexe{M}           % "M" ou "F"
\newcommand\PageGarde{N}         % "O" ou "N"
\newcommand\AnnexesPresentes{O}  % "O" ou "N". Indique si le document comprend des annexes.
\newcommand\mesMotsClef{Liste,de,mot-clés,séparés,par,des,virgules}
%%
%%  DEFINITION DU JURY
%%
%%  Pour la définition du jury, les macros suivantes sont definies:
%%  \PresidentJury, \DirecteurRecherche, \CoDirecteurRecherche, \MembreJury, \MembreExterneJury
%%
%%  Toutes les macros prennent 4 paramètres: Sexe (M/F), Prénom, Nom, Titres
\newcommand\monJury{\PresidentJury{M}{Prénom}{NOM}{Doct.}\\
\DirecteurRecherche{F}{Prénom}{Nom}{Ph.~D.}\\
\MembreJury{M}{Prénom}{Nom}{Ph.~D.}}

\ifthenelse{\equal{\monDiplome}{M}}{
\newcommand\monSujet{Mémoire de maîtrise}
\newcommand\monDipl{Maîtrise ès sciences appliquées}
}{
\newcommand\monSujet{Thèse de doctorat}
\newcommand\monDipl{Philosophi\ae{} Doctor}
}

%%
%% Définitions et packages personnels
%%
\usepackage{listings}   % code listings
\usepackage{mathtools}  % case bracket and underbraces
\usepackage{amssymb}    % checkmarks

\setlength{\headheight}{15pt} % avoid fancydir warning

%%
%% Informations qui sont stockées dans un fichier PDF.
\hypersetup{
  pdftitle={\monTitre},
  pdfsubject={\monSujet},
  pdfauthor={\monPrenom{} \monNom},
  pdfkeywords={\mesMotsClef},
  bookmarksnumbered,
  pdfstartview={FitV},
  hidelinks,
  linktoc=all
}
%%
%% Il y a un document par chapitre du mémoire.
%%
\begin{document}
%%
%% Page de titre du mémoire.
\frontmatter
% Compte optionellement la page de garde dans la pagination.
\ifthenelse{\equal{\PageGarde}{O}}{\addtocounter{page}{1}}{}
\thispagestyle{empty}%
\begin{center}%
\vspace*{\stretch{1}}
UNIVERSITÉ DE MONTRÉAL\\
\vspace*{\stretch{1}}
\MakeUppercase{\monTitre}\\
\vspace*{\stretch{1}}
\MakeUppercase{\monPrenom~\monNom}\\
DÉPARTEMENT DE \MakeUppercase{\monDepartement}\\
ÉCOLE POLYTECHNIQUE DE MONTRÉAL\\
\vspace*{\stretch{1}}
\ifthenelse{\equal{\monDiplome}{M}}{MÉMOIRE PRÉSENTÉ}{THÈSE PRÉSENTÉE} EN VUE DE L'OBTENTION\\
DU DIPLÔME DE \MakeUppercase{\monDipl}\\
(\MakeUppercase{\maDiscipline})\\
\MakeUppercase{\moisDepot} \anneeDepot
\end{center}%
\vspace*{\stretch{1}}
\copyright~\monPrenom~\monNom, \anneeDepot.
%%
%% Identification des membres du jury.
%%
\newpage\thispagestyle{empty}%
\begin{center}%
\vspace*{\stretch{2}}
\ul{UNIVERSITÉ DE MONTRÉAL}\\
\vspace*{\stretch{1}}
\ul{ÉCOLE POLYTECHNIQUE DE MONTRÉAL}\\
\vspace*{\stretch{2}}
Ce\ifthenelse{\equal{\monDiplome}{M}}{~mémoire intitulé}{tte thèse intitulée}:\\
\vspace*{\stretch{1}}
\MakeUppercase{\monTitre}\\
\vspace*{\stretch{2}}
\end{center}%
\begin{flushleft}
présenté\ifthenelse{\equal{\monDiplome}{M}}{}{e}
par:~\ul{\mbox{\MakeUppercase{\monNom} \monPrenom}}\\
en vue de l'obtention du diplôme de:~\ul{\mbox{\monDipl}}\\
a été dûment accepté\ifthenelse{\equal{\monDiplome}{M}}{}{e} par le jury d'examen constitué de:\end{flushleft}
\vspace*{\stretch{2}}
\monJury
%%
\pagestyle{pagenumber}%
%% Dédicace
%%
%% La dédicace est un hommage que l'auteur souhaite
%% rendre à une ou plusieurs personnes de son choix.
%%
\chapter*{DÉDICACE}\thispagestyle{headings}
\addcontentsline{toc}{compteur}{DÉDICACE}
\begin{flushright}
  \itshape
  À Anne, Aude, Adrien, Alexandre et Thomas,\\
  pour votre soutien durant cette maîtrise,\\
  et les bons moments passés ensembles.

\end{flushright}
          % Dédicace du document.
% Remerciements
%
%   Grâce aux remerciements, l'auteur attire l'attention du lecteur
% sur l'aide que certaines personnes lui ont apportée, sur leurs
% conseils ou sur toute autre forme de contribution lors de la
% réalisation de son mémoire. Le cas échéant, c'est dans cette section
% que le candidat doit témoigner sa reconnaissance à son directeur de
% recherche, aux organismes dispensateurs de subventions ou aux
% entreprises qui lui ont accordé des bourses ou des fonds de
% recherche.
\chapter*{REMERCIEMENTS}\thispagestyle{headings}
\addcontentsline{toc}{compteur}{REMERCIEMENTS}
%
Texte.
     % Remerciements.
% Résumé du mémoire.
%
%   Le résumé est un bref exposé du sujet traité, des objectifs visés,
% des hypothèses émises, des méthodes expérimentales utilisées et de
% l'analyse des résultats obtenus. On y présente également les
% principales conclusions de la recherche ainsi que ses applications
% éventuelles. En général, un résumé ne dépasse pas quatre pages.
%
%   Le résumé doit donner une idée exacte du contenu du mémoire ou de la thèse. Ce ne
% peut pas être une simple énumération des parties du document, car il
% doit faire ressortir l'originalité de la recherche, son aspect
% créatif et sa contribution au développement de la technologie ou à
% l'avancement des connaissances en génie et en sciences appliquées.
% Un résumé ne doit jamais comporter de références ou de figures.
\chapter*{RÉSUMÉ}\thispagestyle{headings}
\addcontentsline{toc}{compteur}{RÉSUMÉ}

Le résumé est un bref exposé du sujet traité, des objectifs visés,
des hypothèses émises, des méthodes expérimentales utilisées et de
l'analyse des résultats obtenus. On y présente également les
principales conclusions de la recherche ainsi que ses applications
éventuelles. En général, un résumé ne dépasse pas quatre pages.

Le résumé doit donner une idée exacte du contenu du mémoire ou de la thèse. Ce ne
peut pas être une simple énumération des parties du document, car il
doit faire ressortir l'originalité de la recherche, son aspect
créatif et sa contribution au développement de la technologie ou à
l'avancement des connaissances en génie et en sciences appliquées.
Un résumé ne doit jamais comporter de références ou de figures.

- état de l'art des méthodes de model-checking logiciel pour le cas du C concurrent
    - fait ressortir de nombreuses techniques, mais une limitation sur les
    possiblité de spécification, inaptée au cas logiciel + concurrent
- proposition d'une nouvelle spécification
    - recouvre LTL et les assertions
    - basée sur LTL
    - concept de zone de validités
- instrumentation permettant de vérifier un programme spécifié par notre
spécification à l'aide de tout model-checker supportant les assertions
    - repose sur la crétation d'un automate de büchi, son implémentation en C et
    la construiction du produit avec le programme
    - permet l'utilisation de différents backends
      % Résumé du sujet en français.
% Abstract
%
%   Résumé de la recherche écrit en anglais sans être
% une traduction mot à mot du résumé écrit en français.

\chapter*{ABSTRACT}\thispagestyle{headings}
\addcontentsline{toc}{compteur}{ABSTRACT}
%
\begin{otherlanguage}{english}

Written in English, the abstract is a brief summary similar to the previous
section {\selectlanguage{frenchb}(Résumé)}.  However, this section is not a
word for word translation of the French.

\end{otherlanguage}
          % Résumé du sujet en anglais.

{\setlength{\parskip}{0pt}
%%
%% Table des matières.
\renewcommand\contentsname{TABLE DES MATIÈRES}
\tableofcontents
%%
%% Liste des tableaux.
\renewcommand\listtablename{LISTE DES TABLEAUX}
\listoftables
%%
%% Table des figures.
\renewcommand\listfigurename{LISTE DES FIGURES}
\listoffigures
%%
%% Liste des annexes au besoin.
}

% Liste des sigles et abbréviations
\newcommand\abbrevname{LISTE DES SIGLES ET ABRÉVIATIONS}
\chapter*{\abbrevname}
\addcontentsline{toc}{compteur}{\abbrevname}
\pagestyle{pagenumber}
%
\begin{acronym}
  \acro{IETF}{Internet Engineering Task Force}
  \acro{OSI}{Open Systems Interconnection}
\end{acronym}
%
\begin{longtable}{lp{5in}}
IETF       & Internet Engineering Task Force\\
OSI        & Open Systems Interconnection\\
\end{longtable}

       % Liste des sigles et abréviations.
\ifthenelse{\equal{\AnnexesPresentes}{O}}{\listofappendices}{}
\mainmatter
% Dans l'introduction, on présente le problème étudié et les buts
% poursuivis. L'introduction permet de faire connaître le cadre de la
% recherche et d'en préciser le domaine d'application. Elle fournit
% les précisions nécessaires en ce qui concerne le contexte de
% réalisation de la recherche, l'approche envisagée, l'évolution de
% la réalisation. En fait, l'introduction présente au lecteur ce
% qu'il doit savoir pour comprendre la recherche et en connaître la
% portée.

\chapter{INTRODUCTION}\label{sec:Introduction}  % 10-12 lignes pour introduire le sujet.

Les systèmes informatiques ont pris une place sans précédent dans la société
actuelle. D'une part, ils sont devenus omniprésents à l'échelle personnelle et
dans le monde économique. Les smartphones et les ordinateurs personnels sont
entrés dans notre quotidien. L'informatique embarquée est utilisée massivement,
que ce soit dans les véhicules ou les usines. À travers internet, nous
interagissons en permanence avec des programmes sur des serveurs distants.

Simultanément, le champ d'action des systèmes informatiques s'est étendu. En
plus de leur rôle dans les systèmes embarqués et les communications, les
logiciels ont maintenant un large accès aux données personnelles de leurs
utilisateurs. À travers la domotique, ils peuvent agir sur notre environnement
immédiat. Les systèmes de recommandation et d'intelligence artificielle
influencent les prises de décisions.

Dans ce contexte, les dysfonctionnements des systèmes informatiques sont un
sujet de préoccupation. Une erreur peut avoir des conséquences de grande
ampleur, un coût important et porter atteinte à la sécurité des informations et
des personnes. Il est cependant difficile de concevoir un système correct. On
peut s'en convaincre en consultant la liste des bugs et des failles
informatiques apparus ces dernières années. \cite{horror_story} présente
certains des dysfonctionnements ayant eu les conséquences les plus importantes.
Les exemples les plus connus restent sans doute l'explosion d'Ariane 5, due à
une taille de variable sous-dimensionnée, et la faille HearthBleed dans la
bibliothèque de cryptographie OpenSSL.

Vérifier les systèmes informatiques a ainsi gagné une importance qui n'est
plus discutable. Cependant, l'inspection manuelle des logiciels est une
tâche longue, coûteuse et souvent source d'erreurs. Sur les systèmes les
plus complexes, elle devient irréalisable. Il est donc nécessaire de se
reposer sur des techniques et des outils permettant d'automatiser,
partiellement ou totalement, le processus de vérification.

On peut séparer les techniques et les outils utilisés pour vérifier un
logiciel selon plusieurs critères. L'un de ces critères est le degré
d'automatisation. Les outils peuvent aller d'un système totalement
autonome à un simple assistant pour l'ingénieur responsable de la tache de
vérification. L'automatisation est un facteur important pour la
vérification de systèmes de grande taille. Le risque d'une erreur
humaine dans un processus fortement automatisé est aussi réduit. Un
second critère est la rigueur du verdict fourni par la technique de
vérification. Certaines méthodes (les méthodes formelles) permettent
d'obtenir une preuve de validité tandis que d'autres (comme le test ou
certaines méthodes d'analyse statiques) permettent seulement d'augmenter
son degré de confiance en la validité du programme.

Dans ce mémoire, nous allons nous intéresser aux techniques de
vérification automatisées et fournissant une garantie de qualité
rigoureuse du programme à vérifier. Nous allons en particulier nous
concentrer sur les techniques de model-checking. Nous allons tout
d'abord rapidement présenter quelques-unes des méthodes de vérifications
existantes afin de dresser le contexte dans lequel le model-checking
s'inscrit.

%%
%%  CONCEPTS DE BASE
%%
\section{Définitions et concepts de base}  % environ 2-3 pages

\subsection{Le test}

La méthode de vérification la plus employée, de loin, est le test. Un
programme ou un système informatique a généralement pour objectif de
réaliser une tâche précise, en respectant des critères de qualité et de
performances. Un test permet, étant donné un environnement et des
paramètres d'entrée pour le programme, de vérifier s’il produit le
résultat attendu (indiqué par un oracle).

La grande majorité des projets de développement maintiennent ainsi des
suites de tests afin de s'assurer que le programme se comporte de la
manière attendue. Différents paramètres et scénarios d'exécution sont
testés afin de détecter autant d'erreurs que possible.

Ces suites de tests sont gérées de manière automatisée et sont étendues au
tout au long du développement d'un logiciel. Ils peuvent prendre des
dimensions considérables. Par exemple, dans le cas de SQLite, les
suites de tests représentent près de cent millions de lignes, soit plus
de 700 fois le nombre de lignes de la bibliothèque elle-même
(https://www.sqlite.org/testing.html).

Cependant, les tests souffrent cependant d'un certain nombre de
défauts. Tout d'abord, il est nécessaire d'avoir une bonne connaissance
du système et de faire preuve d'une certaine imagination afin de
concevoir les scénarios susceptibles de mener à une erreur. Lorsque
c'est le cas, il est souvent complexe de remonter jusqu'à la source de
l'erreur afin de la corriger. Les tests rencontrent aussi un problème de
couverture : il est extrêmement difficile de produire un test pour
chaque comportement du système. Ce problème est accru si l'on passe à
l'échelle : le nombre de comportements d'un système augmente
exponentiellement avec sa taille, le nombre de tests nécessaire pour
couvrir l'ensemble des comportements devrait donc faire de même. Créer, maintenir et
corriger les tests et les oracles représente alors un investissement conséquent.

Les tests assurent donc généralement une couverture partielle du
programme à vérifier. Une suite de tests ne permet donc pas de garantir
l'absence d'erreurs, mais seulement l'absence d'erreurs dans les chemins
d'exécutions testés.

Différents critères de couvertures existent afin de mesurer la qualité
d'une suite de tests. On peut demander que toutes les fonctions du
programme soient exécutées au moins une fois lors des tests (on parle
alors de \emph{function coverage}), ou que l'ensemble des instructions
du programme soit exécuté (\emph{statement coverage}). On peut aussi
demander à une suite de tests de contenir des exécutions sollicitant
successivement différentes instructions du programme.

Les tests sont généralement moins performants dans le cas de programmes
concurrents. L'entrelacement entre les instructions peut avoir un impact
sur l'issue du test, et cet ordre change de manières non déterministes
entre les exécutions. Une erreur peut alors être détectée lors d'une
exécution des tests, mais être très difficile à reproduire. Pour tenter
de détecter plus efficacement les erreurs liées à la concurrence, des
techniques comme le stress-testing ont été mises en place. Elles
consistent à ajouter du bruit à l'exécution des tests ou à lancer
plusieurs instances du test en parallèle afin d'augmenter la probabilité
des entrelacements d'instructions les moins fréquents.

Malgré ces limitations, le test reste généralement la méthode de
vérification la plus simple à mettre en place. Les autres techniques de
vérification sont parfois utilisées de manière complémentaire à une
suite de tests, afin d'apporter une preuve de correction plus rigoureuse
sur les fragments critiques du programme.

\subsection{Les méthodes formelles}

Les méthodes de test permettent de trouver des erreurs dans un programme
et d'augmenter la confiance des développeurs dans le fait que le
programme est correct, mais elles ne permettent pas de prouver l'absence
d'erreurs.

Les méthodes de vérification formelles sont une réponse à ce défaut :
elles permettent de prouver l'absence d'erreurs dans un programme,
quels que soient les entrées et les chemins d'exécution.

En pratique, vérifier qu'un système est complètement correct (c.-à-d il ne contient aucune erreur, que ce soit des erreurs
fonctionnelles ou d'exécution) est généralement trop complexe. Il est
cependant possible de garantir l'absence de certains types d'erreurs et de
vérifier des propriétés critiques pour le système.

De nombreuses méthodes existent afin de prouver la correction d'un
programme vis-à-vis d'une propriété. Nous allons ici nous intéresser aux
méthodes automatiques, qui ne requièrent pas (ou peu) d'interaction
humaine pour fournir un résultat. En particulier, nous excluons les
méthodes basées sur des assistants de preuve (comme Coq ou HOL), qui
demandent un effort significatif et des connaissances avancées à
l'utilisateur afin de construire une preuve.

\subsubsection{L'interprétation abstraite}

L'interprétation abstraite fait partie du domaine de l'analyse statique.
L'analyse statique regroupe les techniques permettant d'obtenir des
informations sur le comportement d'un programme à partir de son code
source, sans l'exécuter.

Le théorème de Rice indique que la plupart des propriétés d'un programme
ne sont pas décidables. L'essence de l'interprétation abstraite est de
contourner ce problème en construisant une approximation du résultat de
manière efficace et correcte. Le principal enjeu de
l'interprétation abstraite est d'être suffisamment précise afin de ne
pas indiquer un trop grand nombre de fausses (\emph{spurious}) erreurs.

Plutôt que d'essayer de suivre l'ensemble des valeurs possibles pour
chaque variable, à chaque point du programme, l'interprétation abstraite
utilise un domaine abstrait qui permet d'approximer l'ensemble des
valeurs qu'une variable peut prendre. Par exemple, on peut approximer
l'ensemble des valeurs d'une variable par le plus petit intervalle
contenant l'ensemble de ces valeurs. Le programme est alors interprété,
chaque opération concrète sur les variables étant transposée dans le
domaine abstrait. L'utilisation des domaines abstraits permet de réduire
largement le nombre de valeurs à manipuler et facilite les opérations.

On peut alors prouver la correction du programme vis-à-vis d'une
propriété en l'évaluant dans le domaine abstrait. Pour poursuivre
l'exemple précédent, supposons que l'on veut s'assurer qu'une variable n'est jamais
nulle à un certain point du programme (pour éviter une erreur arithmétique
par exemple). Si l'analyse permet d'établir qu'elle se situe dans
l'intervalle \([1, 5]\), on peut conclure que la propriété est vérifiée.
Cependant, si l'on obtient l'intervalle \([-1, 1]\), il n'est pas possible
de conclure : une erreur est possible, mais elle n'est pas certaine.

En pratique, l'interprétation abstraite est souvent utilisée pour
vérifier des propriétés génériques, intégrées dans les outils
(\emph{built-in}). C'est une technique qui
passe bien à l'échelle et qui peut analyser de larges bases de code. Des
propriétés typiquement vérifiées sont l'absence d'erreurs arithmétiques,
l'absence de variables utilisées avant d'être initialisées, ou le
respect des bornes des tableaux.

Cependant, l'interprétation abstraite souffre de deux principaux défauts
: tout d'abord, les approximations réalisées peuvent provoquer de
nombreux faux positifs, signalant une erreur éventuelle dans un code
correct. La précision de l'analyse peut être parfois être améliorée en
choisissant un domaine abstrait plus raffiné, mais cela a généralement
un coût au niveau des performances. De plus, l'interprétation abstraite
ne permet pas d'obtenir un contre-exemple menant vers une erreur
indiquée, ou toute autre information indiquant la cause de cette erreur.
Il peut alors être complexe de déterminer si une erreur reportée est un
faux positif ou non, et de comprendre sa cause pour la corriger.

\subsubsection{Le model-checking}

Les techniques de model-checking se basent sur l'exploration d'un modèle
du système à vérifier. Ce modèle prend généralement la forme d'un
système de transition, il peut être créé manuellement ou extrait
automatiquement du système. Les outils de model-checking vont alors
explorer la partie accessible du modèle de manière exhaustive.

Le modèle doit à la fois représenter fidèlement le système vis-à-vis des
propriétés à vérifier, et être suffisamment simple pour qu'il soit
possible de l'explorer en un temps raisonnable. S’il est possible
d'atteindre un état, ou d'emprunter une succession d'états ne respectant
pas la propriété désirée, une erreur est reportée. Un contre-exemple ---
une trace d'exécution menant à cette erreur --- peut alors être généré.
Il a en général une grande valeur pour comprendre la source de l'erreur
et la corriger, et constitue un des plus grands atouts du model-checking.

La distinction entre le model-checking et l'interprétation abstraite est
principalement historique. L'analyse statique a été conçue pour vérifier
des propriétés simples sur les programmes. Les algorithmes ont souvent
mis l'accent sur les performances quitte à perdre en précision. D'autre
part, le model-checking a été développé pour prouver des propriétés
complexes sur le design de circuits. La priorité a été la précision des
algorithmes et la capacité à vérifier des propriétés liées au
programme. Cependant, les analyseurs statiques ont désormais gagné en
précisions et sont capables vérifier des spécifications. En parallèle, les
model-checkers ont développé des abstractions pour gagner en
performances. La distinction reste donc principalement une question
d'intention dans l'orientation de l'outil.

Le model-checking reste limité par le problème de \emph{l'explosion
combinatoire}. La taille des modèles et la durée des taches de
vérification augmentent généralement de manière exponentielle avec la
complexité du système et des propriétés à vérifier.

\subsection{Concurrence}

Les lois de Moore sont des conjectures empiriques qui prédisent que la
puissance de calcul des processeurs double chaque année. Cependant, la
miniaturisation des transistors approche ces limites physiques,
annonçant la fin des lois de Moore dans leur interprétation
traditionnelle. Cependant, afin continuer à gagner en puissance de
calcul, les processeurs ont commencé à évoluer dans une autre direction
: plutôt que d'augmenter la puissance, c'est le nombre de cœurs qui va
croître. Pour tirer partit de ces processeurs multicœurs, il est
cependant nécessaire de réaliser des programmes parallèles.

Plusieurs modèles de programmation concurrente existent. Par exemple, ils
peuvent être basés sur des évènements (SystemC), ou sur des algorithmes
de type map / reduce (MPI). Dans ce mémoire, nous allons nous concentrer
sur les threads suivant la norme POSIX (\emph{pthread}).

La concurrence introduit tout un nouveau panel de difficultés dans la
conception d'un programme. Les accès à la mémoire partagée par plusieurs
fils d'exécutions doivent être protégés pour éviter les \emph{race conditions}, qui peuvent fausser le résultat d'une opération. Il est
aussi nécessaire de synchroniser les fils d'exécution et de tenir compte
des différents entrelacements possibles entre les instructions. Ces
difficultés provoquent souvent l'apparition d'erreurs subtiles. En
particulier, les erreurs dépendant de l'entrelacement des instructions
sont souvent complexes à détecter et à diagnostiquer. L'entrelacement
étant non déterministe, ces erreurs sont souvent difficiles à détecter
et à reproduire.

Les programmes concurrents représentent un défi pour les méthodes de
vérification. Les méthodes reposant sur l'exécution du programme, comme
le test, n'ont pas de contrôle sur l'entrelacement des instructions. Il
devient alors difficile de tester certains chemins critiques et de
diagnostiquer les erreurs rencontrées. Les techniques de model-checking
possèdent ce contrôle --- elles ont initialement été conçues pour vérifier
des systèmes électroniques concurrents. Cependant, dans le cas du
model-checking logiciel, la concurrence amplifie l'explosion
combinatoire de manière exponentielle. Il est nécessaire d'explorer tous
les entrelacements entre les instructions d'un programme. Or, pour un
programme composé de \(n\) threads, contenant \(k\) instructions chacun,
il existe \(\frac{(nk)!}{(k!)^n}\) entrelacements. Pour \(k = 100\)
instructions et \(n = 2\) threads, on obtient déjà près de \(10^{59}\)
entrelacements !

\section{Domaine de ce mémoire}

Dans ce mémoire, nous allons restreindre le champ de notre étude aux
outils de model-checking pour des programmes écrits en C.

Le langage C est un langage bas niveau parmi les plus utilisés et les
plus présents dans les bases de code actuelles. Il est en particulier
utilisé dans les applications embarquées, les applications systèmes et
les applications temps réel, qui nécessitent un contrôle bas niveaux et
des performances élevées. Ces types d'applications sont souvent
critiques et ne peuvent tolérer un dysfonctionnement. Le C est donc un
langage privilégié par les méthodes de vérification formelle, supporté
par de nombreux outils de model-checking.

Quand il sera question de concurrence, nous nous intéresserons en
particulier aux programmes multi threads suivant la norme POSIX, et
utilisant la bibliothèque \emph{pthread}. Elle implémente une gestion
bas niveau de la concurrence, souvent utilisée dans les programmes
embarqués ou les codes systèmes.

Deux questions ouvertes forment actuellement le cœur de la recherche
sur le model-checking. La première, et la principale, est la question
des performances et de la lutte contre l'explosion combinatoire. La
seconde est la question de la spécification, les propriétés qu'il est
possible de vérifier sur un système, et la manière de les exprimer. Dans
ce mémoire, nous allons nous intéresser à ce second point.

Les assertions sont le mécanisme de spécification le plus utilisé dans
le cadre de la vérification logicielle. Une assertion prend la forme
d'une instruction dans le code, prenant en paramètre une expression
booléenne pure --- sans effet de bord --- qui joue le rôle d'un
invariant. Si cet invariant n'est pas respecté, l'assertion est
déclenchée (ce qui met fin au programme dans le cas d'une exécution, ou
permet de faire remonter une erreur dans le cas du model-checking). La
plupart des outils de model-checking logiciel supportent une
spécification sous la forme d'assertion.

Cependant, dans le cas de programmes multi threads, les assertions
présentent des faiblesses. En particulier, l'invariant d'une assertion
ne peut utiliser des valeurs hors de son contexte (une variable d'un
autre thread par exemple). De plus, une assertion est bloquante :
l'exécution ou l'exploration du programme stoppe une fois une assertion
atteinte. Il n'est pas possible d'exprimer des relations temporelles
(simultanéité ou succession par exemple) entre des assertions pour
déclencher une erreur dans ces cas seulement.

Ma question de recherche sera donc composée des deux parties suivantes :

\begin{itemize}
\item
  Comment peut-on améliorer la spécification des programmes concurrents
  et la rendre plus expressive ?
\item
  Comment peut-on rendre les outils de model-checking actuels
  compatibles avec une spécification plus expressive ?
\end{itemize}

Pour améliorer l'expressivité de la spécification des programmes
concurrents, nous allons nous intéresser aux logiques temporelles, et en
particulier aux formules \ac{LTL}. Ces dernières ont déjà été utilisées dans
le cadre de model-checking logiciel, dans un cadre restreint. Nous
proposons une extension de ce cadre, permettant en particulier de faire
référence à des positions du code et à des variables locales dans les
formules. En particulier, notre extension permet d'englober les
assertions classiques et de créer des relations temporelles entre ces
dernières.

Les outils de model-checking sont généralement complexes et optimisés
pour le type de propriétés qu'ils prennent en charge. Afin de rendre
les outils existants compatibles avec notre spécification, nous mettons
en place une transformation de source à source du système. Nous
construisons ainsi un nouveau programme spécifié par des assertions,
dont la validité est équivalente à celle du système d'origine. Cette
méthode nous permet en particulier de tester notre approche avec
différents outils et de profiter de leurs optimisations.

%%
%% PLAN DU MEMOIRE
%%
\section{Plan du mémoire}  % 0.5 page

La partie \ref{sec:RevLitt} de ce mémoire consiste en un état de l'art. Nous présentons
d'abord les spécifications à base d'assertion et la logique \ac{LTL}. Nous
introduisons ensuite les principaux algorithmes de model-checking
logiciels, ainsi que les outils les implémentant et leurs spécificités.

Dans la partie \ref{sec:Theme1}, nous détaillons le formalisme de spécification que
nous avons établi. Nous le comparons aux autres tentatives allant dans
le même sens et nous mettons en lumière les principaux problèmes
résolus ou encore ouverts.

Dans la partie \ref{sec:Theme2}, nous présentons une transformation de source à source
destinée à rendre les outils existants compatibles avec notre
spécification. Nous construisons pour cela le produit entre le système
et l'automate de Büchi associé à la propriété à vérifier. Nous évaluons
ensuite les performances obtenues pour la vérification d'exemples
simples.
       % Introduction au sujet de recherche.
\Chapter{REVUE DE LITTÉRATURE}\label{sec:RevLitt}

Le \emph{model-checking} rassemble des techniques permettant de vérifier
et valider des systèmes. Historiquement, le model-checking a
tout d'abord été utilisé pour la vérification de composants
électroniques. Les techniques ont ensuite été adaptées afin de permettre
la validation de programmes. On parle alors de model-checking logiciel.
Nous allons par la suite nous intéresser à celui-ci uniquement.

Le principe du model-checking est de mener une recherche
exhaustive à travers une modélisation du système plutôt que de
construire une preuve formelle de validité. Alors qu'une telle preuve
peut demander une certaine part d'intuition, le model-checking permet
une approche plus systématique, qui peut être plus facilement
automatisée.

De nombreux algorithmes de model-checking ont été développés en fonction
des propriétés à vérifier sur le système, du formalisme dans lequel est
décrit le système et de la complexité de celui-ci.
Dans ce chapitre, nous présentons les principales techniques de
model-checking logiciel et nous identifions leurs avantages et
inconvénients respectifs.

Nous identifions les outils les plus performants implémentant chacune de ces
techniques, à l'heure actuelle. Nous restreignons cependant cette étude aux
outils qui nous intéresseront pour la suite de ce rapport. Ces outils sont donc
susceptibles d'être utilisés en arrière-plan (backend) pour la vérification des
codes sources considérés dans ce mémoire. Nous ne considérerons donc que
des outils capables de vérifier des programmes concurrents en C, suivant le
modèle d'exécution défini par la norme POSIX (pThread).

\textbf{Structure :} Dans la section~\ref{sec:model-checking-logiciel}, nous
définissons le model-checking logiciel ainsi que les formalismes associés.
Dans la section~\ref{sec:specification}, nous présentons
les types de propriétés des programmes vérifiables grâce au model-checking et
les formalismes utilisés pour les exprimer et les représenter formellement.
Enfin, dans la section~\ref{sec:techniques-et-outils-de-model-checking}, nous
présentons différentes techniques utilisées pour le model-checking de programmes
multi thread ainsi que les outils qui les implémentent.

\section{Model-checking logiciel}\label{sec:model-checking-logiciel}

Le model-checking permet de prouver qu'un modèle d'un système vérifie une
spécification. Avant de nous intéresser davantage aux différentes techniques de
model-checking, nous allons présenter les notions de modèle et de spécification.
Nous définissions aussi un certain nombre de notions et termes techniques liés à
cette problématique.

\subsection{Modèle}

Le point caractéristique du model-checking est l'utilisation de
\emph{modèles} des systèmes à vérifier. Un modèle est une représentation
abstraite du système. Au lieu de vérifier un système directement, les techniques
de model-checking s'appliquent sur le modèle du système. On considère un
système comme correct lorsque son modèle ne contient pas d'erreurs.

L'origine des modèles est liée au domaine d'application initial du
model-checking : la vérification du design de composants électroniques. Il
était nécessaire de représenter ces derniers de manière abstraite afin de les
vérifier, ce qui a donné naissance à la notion de modèles.

L'utilisation d'un modèle permet de simplifier les tâches de vérification : un
modèle bien conçu capture le fonctionnement du système en retirant les détails
n'ayant pas d'impact sur les propriétés à vérifier. Cette version simplifiée du
système permet ainsi de réaliser des tâches de vérification complexes, qui ne
pourraient aboutir autrement. Cependant, le modèle doit être suffisamment fidèle
pour préserver les propriétés à vérifier. Un modèle trop simple pourrait
provoquer l'apparition de faux positifs ou pire, ne pas contenir une erreur
pourtant bien présente dans le système.

La phase de modélisation est donc une phase délicate et propice aux erreurs. Il
est nécessaire de bâtir un équilibre entre l'abstraction du système et le
respect de son fonctionnement. Cet équilibre dépend des propriétés que l'on
souhaite vérifier. La modélisation demande donc une bonne connaissance
du système et des outils de model-checking utilisés. Les tendances actuelles
visent à réduire l'intervention humaine dans la phase de modélisation en
automatisant celle-ci.

\subsection{Système de transitions et trace d'exécution}

Un modèle prend souvent la forme d'un système de transitions.

\paragraph{Système de transitions}
Formellement, un système de transitions est un triplet \((S, s_0, \rightarrow)\),
avec :

\begin{itemize}
\item
  \(S\) l'ensemble des états du système ;
\item
  \(s_0 \in S\) l'état initial du système ;
\item
  \(\rightarrow \subset S \times S\) la relation de transitions du
  système ;
\end{itemize}

Intuitivement, les états d'un système de transitions représentent un
statut possible du système, tandis que les transitions représentent les
actions qui peuvent le faire évoluer.

\paragraph{Trace d'exécution}
Une trace d'exécution est un chemin dans le système de transition représentant
le modèle, possiblement infini, dont le premier état est l'état initial du
système de transition. Une trace d'exécution est donc formée d'une suite \(s_0,
s_1, .. \in S\) d'états, avec \(s_0\) l'état initial, et d'une suite \(t_0, t_1,
... \in \rightarrow\) de transitions telles que \(\forall i, t_i = (s_i,
s_{i+1})\).

Le model-checking consiste à examiner l'ensemble des traces
d'exécution du modèle afin de vérifier si elles respectent la spécification.

\subsection{Vocabulaire technique}

Nous allons définir ici un certain nombre de termes liés au model-checking ou
aux méthodes formelles de manière générale, que nous réutiliserons par la suite.

\paragraph{Faux positif / Faux négatif}

Lorsqu'un outil de vérification indique une erreur dans un système alors que ce
dernier est en fait correct, on parle de \emph{faux positif}. À l'inverse, si un
outil reporte un système comme correct alors qu'il contient une erreur, on parle
de \emph{faux négatif}.

\paragraph{Complet / Correct}

Un outil ou une technique de vérification est complet s’il est conçu de sorte à
ne jamais faire de faux négatif : si une erreur existe dans un système, elle
sera toujours signalée.
Un outil ou une technique de vérification est correct s’il est conçu de sorte à
ne jamais faire de faux positif : si une erreur est signalée dans un système,
elle est réellement présente dans ce système.

Concevoir un outil de model-checking logiciel à la fois complet et correct est
extrêmement difficile, voire impossible, dans la plupart des cas. En effet, le
théorème de Rice indique que toute propriété sémantique non triviale d'un
programme est indécidable. Une technique de vérification ne respecte donc généralement
qu'une seule de ces caractéristiques, ou aucune des deux.

\subsection{Le cas du model-checking logiciel}

Le model-checking logiciel, ou \emph{software model-checking}, est un cas
particulier de model-checking. Le système à vérifier est un programme, et le
modèle va être extrait automatiquement à partir de son code source.
On automatise ainsi la conception du modèle, ce qui présente les avantages
suivants :

\begin{itemize}
\item
  on réduit le risque d'erreur lors de la modélisation du système,
  en supprimant l'intervention humaine ;
\item
  on automatise davantage le procédé de vérification, ce qui le rend
  plus facilement utilisable en pratique.
\end{itemize}

Cependant, le code doit encore être compilé afin d'obtenir le système,
ce qui constitue une encore une certaine distance entre le modèle et le
système. Afin de réduire cette distance, certains
model-checkers logiciels se basent sur des représentations intermédiaires
plus proches de l'assembleur, comme la représentation intermédiaire de
LLVM\footnote{L'utilisation de cette représentation est aussi expliquée
  par des raisons de compatibilité. De nombreux langages, dont C, C++ et
  C\# peuvent en effet être compilés vers la représentation
  intermédiaire de LLVM, un model-checker utilisant cette dernière est
  alors compatible avec tous ces langages.}.

La programmation impérative consiste à considérer un programme comme un état que
l'on fait évoluer à l'aide d'instructions. L'état du programme est
principalement défini par l'état de la mémoire du programme.
On peut alors modéliser un programme par un système de transition. L'ensemble
des états du système de transition est une partition de l'ensemble des états du
programme. Les états du système de transition sont donc caractérisés par :

\begin{itemize}
\item
  la configuration du tas du programme ;
\item
  la valeur des variables globales du programme ;
\item
  la configuration de la pile de chaque thread ;
\item
  la valeur du pointeur d'instruction de chaque thread.
\end{itemize}

Les instructions permettant de faire évoluer l'état du programme sont
représentées par des transitions entre les états de son modèle. Une transition
représente plus précisément l'action d'une instruction atomique du programme sur
les trois zones mémoires citées précédemment. Les transitions peuvent aussi
dépendre de facteurs externes, comme les entrées du programme.

Dans le cadre du model-checking logiciel, deux hypothèses sont fréquemment
faites sur le système :

\begin{itemize}
\item
  la consistance séquentielle. On considère alors que les écritures et
  les lectures mémoires ont lieu dans l'ordre où les instructions sont
  rencontrées dans une trace d'exécution. En pratique, cette hypothèse
  n'est pas toujours vérifiée. Les compilateurs et les processeurs
  peuvent modifier l'ordre de ces opérations pour optimiser un
  programme. On parle alors de modèle mémoire \emph{faible}.
\item
  la sémantique d'entrelacement. Dans le cas d'un système multi thread,
  on considère que les traces d'exécution possibles sont produites par
  un entrelacement des instructions atomiques de chaque thread. Cette
  hypothèse n'est pas valide si le programme est exécuté par plusieurs
  processeurs : il est alors possible d'exécuter plusieurs instructions
  simultanément, ce qui peut produire des traces d'exécution
  supplémentaires.
\end{itemize}

\paragraph{Exemple}

Considérons le programme suivant, formé de deux threads.

\noindent\begin{minipage}{.45\textwidth}
\begin{lstlisting}[caption=Thread 1, frame=single]
a <- 1
a <- 2
\end{lstlisting}
\end{minipage}\hfill
\begin{minipage}{.45\textwidth}
\begin{lstlisting}[caption=Thread 2,frame=tlrb]
c <- 1
d <- a + c
\end{lstlisting}
\end{minipage}

Ce programme contient trois variables globales, \texttt{a}, \texttt{c} et
\texttt{d}, et deux threads. Les états du système sont donc définis par les
valeurs de ces variables et la position des pointeurs d'instruction de chaque
thread. On peut modéliser ce programme par le système de transitions en Figure
\ref{fig:model_example}. L'étiquette de chaque état représente la valeur des
variables. Les transitions représentent l'exécution d'une instruction. On
remarque l'existence de nombreuses traces d'exécution, pouvant mener à trois
résultats différents.

\begin{figure}
\begin{center}
\includegraphics[width=\textwidth]{model_example}
\end{center}
\caption{Système de transition modélisant un programme simple.}
\label{fig:model_example}
\end{figure}

\subsection{Non-déterminisme}

Un système est rarement complètement déterministe. De nombreux facteurs (les
données en entrée du système, l'instant où un évènement a lieu, la réussite des
allocations mémoires, la concurrence) tendent généralement à rendre l'évolution
d'un système non déterministe.

Face à un choix non déterministe, les techniques de model-checking doivent
explorer toutes les alternatives possibles.

Lors de la modélisation, certaines sources de non-déterminisme sont ignorées.
Elles sont alors remplacées par une exécution déterministe.
C'est généralement le cas pour les appels système ayant la possibilité
d'échouer (les allocations de mémoire, par exemple).

Pour représenter les évènements et les paramètres extérieurs du programme,
une fonction simulant le non-déterminisme est généralement mise à disposition
de l'utilisateur par les outils de model-checking.

\subsection{Explosion combinatoire}

Le nombre d'états d'un programme augmente exponentiellement selon de nombreux
paramètres (nombre de variables du programme, taille des types de données, degré
de concurrence\dots). Il peut éventuellement être infini si des appels de
fonctions ou de l'allocation dynamique de mémoire entrent en jeu. Cependant, afin
de vérifier un système, un model-checker doit explorer l'ensemble de ces états.
Cette tâche devient donc extrêmement coûteuse en temps et en mémoire quand le
nombre d'états devient trop important.
Ce problème, nommé l'explosion combinatoire, est la principale limite du
model-checking.

Le non-déterminisme est une des principales raisons de l'explosion combinatoire :
le nombre d'exécutions est exponentiel selon le nombre de choix non déterministe.

Cela explique les difficultés rencontrées par les techniques de model-checking
face aux programmes concurrents : l'ordonnancement entre les instructions est
non-déterministe. Il existe un nombre d'entrelacements exponentiel en fonction
du nombre d'instructions du programme.

Les différents algorithmes de model-checking utilisent tous des
techniques afin de limiter l'explosion combinatoire. Elle reste
cependant le plus grand obstacle rencontré par le model-checking et
limite le passage à l'échelle de la plupart des techniques.

\section{Formalismes de spécification}\label{sec:specification}

Afin de vérifier un système, il faut tout d'abord établir ce que signifie être
correct pour ce système. Un ingénieur responsable du design d'un système a
généralement une connaissance informelle de la manière dont le système doit se
comporter. Il est alors nécessaire de traduire cette connaissance d'une manière
non ambiguë et compréhensible par des outils. On réalise pour cela une
spécification du système.

Une spécification est constituée d'un ensemble de propriétés, qui représentent
des invariants logiques que le système doit respecter. Si l'un de ces invariants
est brisé, une erreur est présente : le système ne se comporte pas de la manière
attendue. Un outil de vérification peut alors inspecter un système afin de
vérifier si l'ensemble de la spécification est respecté.

Une spécification peut prendre de nombreux aspects : il existe des langages de
spécification (UML par exemple), elle peut être constituée de formules logiques
portant sur les variables du système ou prendre la forme d'un système de
transitions. Un langage de programmation constitue en lui même une spécification
bas niveau du comportement d'un programme.

Les propriétés d'une spécification se répartissent en plusieurs catégories. Une
propriété peut être un comportement généralement attendu pour le type de système
concerné. Par exemple, on attend généralement d'un programme qu'il ne contienne
pas de comportements indéterminés. Une propriété peut aussi être spécifique au
système et à son comportement. On peut par exemple spécifier qu'une variable ne
doit jamais dépasser un certain seuil.

\subsection{\texorpdfstring{Propriétés ``built-in''}{Propriétés built-in}}

Certaines propriétés sont fortement --- si ce n'est toujours --- désirables
pour une catégorie de systèmes. Un model-checker ciblant cette catégorie
de systèmes peut alors implémenter la vérification de ces propriétés
nativement, sans qu'il soit nécessaire de les spécifier. On parle alors
de propriétés \emph{built-in}.

Dans le cas du model-checking logiciel, des propriétés built-in
classiques sont la vérification de la validité des pointeurs et des
opérations arithmétiques, l'initialisation des variables et, de manière
plus générale, l'absence de tout comportement ne respectant pas la norme
du langage.
Dans le cas de programmes concurrents, on cherche généralement à s'assurer de
l'absence de dreadlocks (lorsque tous les threads du programme sont bloqués par une
condition, et que le système ne peut plus évoluer) et de data-races (plusieurs
accès simultanés à une même adresse mémoire, dont au moins un en écriture).

Les propriétés built-in sont souvent des propriétés qu'il serait difficile à
l'utilisateur de spécifier, parce qu'elles impliquent la plupart des variables
du système.

Pour vérifier des propriétés plus spécifiques au système (des propriétés
fonctionnelles par exemple, qui portent sur le résultat que le système doit
produire) il est cependant nécessaire de permettre à l'utilisateur d'exprimer
lui-même la spécification.

\subsection{Assertions}

Dans un article de son blog, John Regehr aborde le sujet des
assertions\cite{assertion_regehr}.
Il reprend en particulier la définition suivante :

\begin{quotation}
\selectlanguage{english}
An assertion is a Boolean expression at a specific point in a program which will
be true unless there is a bug in the program.
\hfill \cite{assertion_regehr}
\end{quotation}

%% \begin{quotation}
%% ASSERT(expr)

%% Asserts that an expression is true. The expression may or may not be evaluated.

%% If the expression is true, execution continues normally.
%% If the expression is false, what happens is undefined.
%% \hfill \cite{assertion_regehr}
%% \end{quotation}

\selectlanguage{french}

Cette définition présente les assertions comme un mécanisme de spécification
exécutable. Si une assertion n'est pas respectée, une erreur est présente dans
le programme. Les assertions sont présentes dans la plupart des langages de
programmation. Elles ont été naturellement reprises par les méthodes de
vérification formelle, dont le model-checking logiciel.

Les assertions permettent de vérifier des propriétés d'accessibilité. Ces
propriétés consistent à vérifier s’il est possible d'atteindre un état donné du
modèle. Cet état est appelé un état d'erreur. On peut aussi considérer
l'accessibilité d'un ensemble d'états d'erreurs.

Les propriétés d'accessibilité sont des propriétés relativement simples :
déterminer si un état est un état d'erreur ne dépend que de cet état. Un
programme contient donc une erreur dès qu'il est possible d'atteindre un état
d'erreur, indépendamment des états et des transitions empruntées au cours de
l'exécution. Une propriété d'accessibilité peut ainsi être déterminée par une
exploration (en largeur par exemple) du système.

Dans un programme, une assertion prend la forme d'une instruction telle que :\\
\texttt{assert(condition);}.
Si elle est exécutée et que la condition est évaluée à vrai, alors l'exécution
du programme continue normalement.
Sinon, le programme a atteint un état d'erreur. Son comportement est alors
indéterminé, bien que le comportement le plus fréquent soit de stopper son
exécution.

Plus formellement, l'assertion \texttt{assert(c)} désigne comme étant des
états d'erreur tous les états tels que :

\begin{itemize}
\item
  un des pointeurs d'instruction du programme pointe sur l'assertion
\item
  l'expression \texttt{c} s'évalue à faux
\end{itemize}

Dans le code suivant, une assertion permet de spécifier que la variable
\texttt{b} doit être non nulle. Dans le cas contraire, une erreur
arithmétique (division par zéro) pourrait avoir lieu à la ligne suivante.

\begin{lstlisting}[language=C, frame=single]
int int_div(int a, int b) {
    assert(b != 0);
    return a / b;
}
\end{lstlisting}

Les assertions sont très utilisées en raison de leur simplicité.
Elles peuvent de plus être placées en tant que spécification pour un outil de
vérification, ou permettre de signaler une erreur dans une version exécutable
du programme (il ne faut cependant pas les confondre avec un mécanisme de
gestion d'erreurs).

Cependant, le pouvoir d'expression des assertions est limité. Toutes les
propriétés d'accessibilité ne peuvent pas être exprimées par une assertion : il
n'est par exemple pas possible d'utiliser dans la condition d'une assertion des
variables hors du contexte courant. Il n'est pas non plus possible d'exprimer un
problème d'exclusion mutuelle : il faudrait pour cela référer à la position des
autres pointeurs d'exécution du code, ce qui n'est pas possible à l'aide d'une
assertion.

\subsection{Logique temporelle}

Pour exprimer des propriétés plus complexes et en particulier des propriétés qui
portent sur l'ensemble d'une trace d'exécution et l'ordre d'apparition de
certains évènements, il est plus approprié d'utiliser une logique temporelle.

Supposons que notre système soit le code d'un distributeur automatique.
On voudrait s'assurer que le distributeur ne délivre jamais le produit
avant que le client ait payé, ce qui va revenir dans le code à vérifier
que la fonction \texttt{livrer\_produit} (qui commanderait au système
de donner au client l'objet commandé) n'est jamais appelée avant la
fonction \texttt{accepter\_paiement} (qui validerait que le paiement a
été correctement effectué), au cours d'une transaction. Des assertions
permettent de déterminer si chacune de ces fonctions est atteinte, mais
sans introduire des variables auxiliaires, il n'est pas possible de
déterminer l'ordre des appels.

Les logiques temporelles permettent d'exprimer ce type de propriétés. On
peut ainsi spécifier, pour un programme, des propriétés sur la
succession des états d'une trace d'exécution.

Ci-dessous sont listés quelques-un des schémas de propriétés les plus
courantes. Davantage sont présenté par \citep{LTL_scheme}. Soit \(p\) une
propriété portant sur les états du système.

\begin{itemize}
\item
  propriété de sûreté : tous les états atteints pendant l'exécution
  vérifient la propriété \(p\).
\item
  propriété d'accessibilité : en un temps fini, un état vérifiant la
  propriété \(p\) est atteint.
\item
  propriété d'équité : on atteindra infiniment souvent un état vérifiant la
  propriété \(p\).
\end{itemize}

Les logiques temporelles les plus utilisées sont : \acl{LTL} (\ac{LTL})
et \ac{CTL}.

\ac{LTL} et \ac{CTL} différent principalement par leur vision de l'ensemble des
traces d'exécution. \ac{LTL} considère chaque trace indépendamment. Un
système est valide par rapport à une propriété \ac{LTL} si toutes ses traces
d'exécution respectent la propriété. \ac{CTL} considère l'ensemble des traces
comme un arbre et permet de quantifier universellement
ou existentiellement sur les successeurs de chaque nœud.

Les pouvoirs d'expressivités de \ac{LTL} et \ac{CTL} ne sont ni équivalents, ni même
comparables. En effet, \ac{LTL} ne permet pas de quantifier existentiellement :
une formule \ac{CTL} utilisant une quantification existentielle n'a donc pas
toujours d'équivalent en \ac{LTL}. La réciproque peut être présentée à l'aide de
la propriété \ac{LTL} \(F (p \land X p)\) (pour toutes les traces, on atteint un
état vérifiant \(p\) et dont le successeur vérifie \(p\)) n'a pas d'équivalent
dans \ac{CTL}. La formule \ac{CTL} \(AF (p \land AX p)\) (quelque soit le chemin
emprunté, on atteint un état vérifiant \(p\) et dont tous les successeurs
vérifient \(p\)) pourrait sembler un bon candidat, mais la Figure
\ref{LTL_vs_CTL} présente un modèle vérifiant \(F (p \land X p)\) mais ne
vérifiant pas \(AF (p \land AX p)\).

\begin{figure}
\begin{center}
\includegraphics[width=0.4\textwidth]{LTL_CTL_non_equivalent.png}
\caption{Modèle vérifiant \(F (p \land X p)\) mais pas \(AF (p \land AX p)\).}
\label{LTL_vs_CTL}
\end{center}
\end{figure}

\ac{CTL} est considérée comme plus difficile à comprendre par les ingénieurs,
plus habitués à penser à une unique exécution linéaire plutôt qu'à un arbre
d'exécution\cite{RCTL_formulas}. \ac{CTL} est par conséquent moins utilisée que
\ac{LTL} dans le cadre du model-checking logiciel. Par la suite, nous allons
donc nous concentrer sur \ac{LTL} uniquement.

\subsection{LTL : Logique Temporelle Linéaire}

La définition de \ac{LTL}\cite{pnueli_LTL} ajoute deux opérateurs temporels à la
logique classique, \(next\) (\(X\)) et \(until\) (\(U\)). La syntaxe d'une
formule \ac{LTL} est définie de la manière suivante, pour \(\phi\) et \(\psi\)
deux formules \ac{LTL} :

\[
\phi, \psi := \text{true }| \text{ false } | \text{ p } |
             \phi \land \psi | \lnot \phi | X \phi | \psi U \phi
\]

\(p\) est une proposition sur l'état du système. On nommera par la suite
ces propriétés des \emph{propositions atomiques}. On identifiera aussi
une proposition atomique avec sa fonction d'évaluation, c'est-à-dire la
fonction qui indique si un état du système vérifie la propriété ou non.

Étant donnée une trace d'exécution infinie \(s = (s_0, s_1, ...)\), \ac{LTL} a
la sémantique suivante :

\[
\begin{aligned}
s \models p & \equiv s_0 \models p \\
s \models X \phi & \equiv (s_1, s_2, \dots) \models \phi \\
s \models \phi U \psi & \equiv \exists k, (s_k, s_{k+1}, \dots) \models \psi
                        \land \forall i <= k, (s_i, s_{i+1}, \dots) \models \phi \\
\end{aligned}
\]

\(\lnot\), \(\land\), \(true\) et \(false\) s'interprètent de la manière
usuelle.

Une trace d'exécution est un modèle d'une proposition atomique si son premier
état est un modèle de la proposition atomique.
L'opérateur \(next\) signifie donc que la propriété \ac{LTL} passée en
paramètre doit être valide sur la trace privée de son premier état.
L'opérateur \(until\) signifie que la première propriété passée en
paramètre doit être vérifiée par tout les états de la trace jusqu'à ce qu'un
état vérifie la seconde formule.

À partir de ces opérateurs base, on définit les opérateurs \(\lor\),
\(\implies\), \dots de la manière classique. On définit aussi les
opérateurs temporels \(always\) (\(G\)), signifiant qu'une propriété est
vraie pour tous les états d'une trace et \(finally\) (\(F\)), signifiant
qu'un état vérifiant une propriété est atteint dans le futur.

\[
\begin{aligned}
F p & \equiv \text{true} U p \\
G p & \equiv \lnot F (\lnot p)\\
\end{aligned}
\]

Enfin, on définit un système comme étant valide par rapport à une
formule \ac{LTL} si toutes les exécutions de ce système sont des modèles de
la formule.

\subsection{Automates de Büchi}

Vérifier si un système respecte une proposition \ac{LTL} demande d'être capable de
manipuler efficacement ces dernières. On utilise pour cela les automates de
Büchi. Ils permettent de représenter une propriété \ac{LTL}. Sous cette forme, elles
sont plus simples à manipuler pour un model-checker. Toute formule \ac{LTL} peut être
représentée par un automate de Büchi. Il existe des algorithmes pour construire
efficacement et automatiquement cet automate\citep{ltl2ba}.

Afin de vérifier si un système respecte une propriété \ac{LTL}, une méthode classique
est de construire un automate de Büchi qui représente la négation de cette
formule \ac{LTL} et d'explorer la composition entre cet automate et le système de
transitions modélisant le système. Un chemin acceptant représente alors une
exécution du système qui viole la propriété.

\paragraph{Automates de Büchi}
Un automate de Büchi est un automate qui accepte un langage de mots
infinis. Formellement, un automate de Büchi est un quintuplé
\(B = (S, \Sigma, I, \delta, F)\), avec :

\begin{itemize}
\item
  \(S\) un ensemble d'états ;
\item
  \(\Sigma\) un alphabet ;
\item
  \(I \subset S\) est un ensemble d'états initiaux ;
\item
  \(\delta \subset (S, \Sigma, S)\) est la relation de transition ;
\item
  \(F \subset S\) est l'ensemble des états finaux.
\end{itemize}

Un calcul (les mots chemin ou trace sont aussi utilisés) dans \(B\) est une
suite infinie de transitions consécutives \(c \in \delta^\omega\), dont l'état
de départ est un état initial :

\[
c = (s_0, a_0, s_1)(s_1, a_1, s_2)\dots(s_n, a_n, s_{n+1})\dots
\]

L'étiquette de ce chemin est le mot infini \(a = (a_0, a_1, \dots, a_n) \in
\Sigma^\omega\). Le chemin \(c\) est réussi si et seulement s’il passe une
infinité de fois par un état final de \(B\).

L'automate \(B\) accepte un mot \(a\) si et seulement s’il existe un calcul réussi dans \(B\) ayant le mot \(a\) pour étiquette.

Pour représenter une formule \ac{LTL}, on prend \(\Sigma = 2^P\), avec \(P\)
l'ensemble des propositions atomiques de la formule. Une lettre de
l'alphabet représente ainsi une configuration des propositions atomiques
du système.

\paragraph{Exemples}

La figure \ref{fig:buchi_example} présente les automates de Büchi pour trois propriétés
\ac{LTL} simples : \(Gp\), \(F(p \lor q)\) et \(G(Fp)\). Seule la partie accessible des automates
est représentée, et les transitions ayant les mêmes sources et destinations sont fusionnées,
leurs étiquettes étant remplacées par une garde sous la forme d'une expression logique.

\begin{figure}
\begin{center}
\includegraphics[height=.17\textheight]{buchi_1.png}
\includegraphics[height=.3\textheight]{buchi_3.png}
\includegraphics[height=.3\textheight]{buchi_2.png}
\end{center}
\caption{Automates de Büchi pour les formules LTL $G p$, $F(p \lor q)$ et
$GF p$.}
\label{fig:buchi_example}
\end{figure}

\paragraph{Automate produit}

Soient un automate de Büchi \(B = (S_B, \Sigma, I_B, \delta, F_B)\) représentant
une propriété \ac{LTL} \(\phi\) et un système de transitions \(T = (S_T, \rightarrow,
I_B)\) modélisant un système.

L'automate produit de \(B\) et \(T\) est définit comme étant \(P = (S_P, \Sigma,
I_P, \delta_P, F_P)\), avec :

\begin{itemize}
\item
  \(S_P = S_B \times S_T\)
\item
  \(I_P = I_B \times I_T\)
\item
  \(F_P = F_B \times S_T\)
\item
  \(((p, q), s, (p', q')) \in \delta_P\) si et seulement si :

  \begin{itemize}
  \item
    \((p, p') \in \rightarrow\)
  \item
    \((q, s, q') \in \lambda\)
  \item
    \(p \models s\)
  \end{itemize}
\end{itemize}

Un mot est accepté par l'automate produit si et seulement s’il représente
une exécution valide dans le modèle du système et qu'il appartient au langage
de l'automate de Büchi.
L'automate produit \(P\) reconnaît donc exactement le langage des exécutions du
modèle \(T\) qui vérifient la propriété \ac{LTL} représentée par l'automate de
Büchi \(B\).

\paragraph{Vérifier une propriété LTL}

Vérifier si le modèle \(T\) respecte la propriété \(\phi\), revient à calculer
si le langage des exécutions valide de \(T\) est inclus dans le langage de
l'automate de Büchi \(B\). Cependant, vérifier une inclusion est une opération
complexe. Il est plus simple de calculer si langage est vide ou non. On va donc
reformuler le problème : on va chercher à déterminer si le langage \(L\) des
exécutions valides dans \(T\) et ne respectant \emph{pas} \(\phi\) est vide.
Si c'est le cas, le système est correct. Sinon, une erreur est présente et les
éléments de \(L\) représentent des contre-exemples.

Pour calculer le langage \(L\), on va construire le produit entre le système de
transitions du modèle et l'automate de Büchi représentant la \emph{négation} de
\(\phi\). Le langage de l'automate produit est alors constitué des exécutions valides du
modèle qui ne respectent pas \(\phi\), soit \(L\).

Il suffit alors d'explorer l'automate produit. S’il est possible, à partir
de l'état initial, d'atteindre un cycle contenant au moins un état final, alors,
\(L\) n'est pas vide : il contient au moins le mot composé par l'étiquette du
chemin allant jusqu'au cycle et d'une infinité de répétitions de l'étiquette du
cycle. Ce mot correspond à une exécution du système qui viole la propriété \(\phi\),
et pourra servir de contre-exemple à l'utilisateur.

L'encodage des propriétés \ac{LTL} à l'aide des automates de Büchi permet ainsi de
vérifier des propriétés complexes sur les modèles. Cependant, la taille d'un
automate de Büchi augmente exponentiellement avec la profondeur de la formule
qu'il représente. Dans le cas de formules de grande taille, il peut donc devenir
problématique de générer l'automate. Sa taille et son aspect non déterministe
viennent renforcer le problème d'explosion combinatoire déjà rencontré par les
techniques de model-checking.

\section{Techniques et outils de model-checking}
\label{sec:techniques-et-outils-de-model-checking}

Différents algorithmes sont utilisés par les outils de model-checking
afin d'explorer l'ensemble des états d'un modèle. Afin de lutter contre
l'explosion combinatoire, ils établissent des compromis au niveau de la
précision de la vérification et du type de propriétés qu'ils prennent en charge.
Leurs performances dépendent aussi fortement de la
structure du programme : les boucles, le non-déterminisme ou la gestion
de la concurrence sont plus ou moins bien supportés selon
les algorithmes.

À l'exception des techniques de séquentialisation, ces algorithmes ont été
initialement appliqués à des programmes séquentiels. Le support des programmes
concurrents est venu ensuite\footnote{Les algorithmes inspirés du model-checking
  de composants électroniques supportent des modèles concurrents depuis
  longtemps. Cependant, le support de la concurrence pour le model-checking
  logiciel a été ralenti par l'explosion combinatoire.}.
Ce dernier n'est pas forcément complexe, d'un point de vue théorique : un
programme multi thread peut s'exprimer comme un système de transitions
non déterministe. Les model-checkers sont capables de vérifier des modèles
non déterministes --- ils sont utilisés pour simuler les entrées possibles d'un
système par exemple --- les algorithmes de model-checking peuvent donc
théoriquement être étendus.
Le problème est bien plus complexe en pratique, en raison des contraintes de
temps et de mémoire que rencontrent les model-checkers. En raison de l'explosion
combinatoire provoquée par le nombre exponentiel d'entrelacements possibles
entre les threads, il est difficile de concevoir un algorithme performant.

Nous présentons dans cette partie les principaux algorithmes utilisés dans le
cadre du model-checking de logiciel multi thread, ainsi que certains des outils
qui les implémentent. Pour chaque algorithme, nous avons choisi les outils qui
nous ont parus les plus aboutis et ceux qui présentaient une approche originale.
Pour chaque algorithme et outil, nous tentons de mettre en valeur ses points
forts, ses faiblesses et ses spécificités.

\subsection{Model-checking explicite}

Le model-checking explicite consiste à énumérer individuellement les
états accessibles du modèle afin de les explorer. Le modèle prend la
forme d'un système de transitions et son exploration est réalisée à
l'aide d'algorithmes de graphes (exploration en largeur ou en
profondeur\dots).

Les algorithmes de model-checking explicite modélisent le programme par un
système de transitions. Ils construisent et explorent ce système de transitions
à la volée : partant de l'état de départ, les successeurs sont déterminés à
partir des instructions du programme. Ils sont alors visités à leur tour, leurs
successeurs sont calculés et ainsi de suite.
Pour ne pas explorer plusieurs fois un même état (l'exploration entrerait alors
dans un cycle infini), il est nécessaire de stocker les états du chemin
emprunté.

Le model-checking explicite est extrêmement vulnérable à l'explosion
combinatoire. Les états accessibles du système sont examinés individuellement.
S’ils sont en nombre infinis, une analyse peut ne pas terminer. Le stockage des
états explorés peut aussi devenir problématique. Plusieurs techniques existent
pour réduire l'impact de l'explosion combinatoire.

\paragraph{Stateless Model-checking}
Certains outils choisissent de ne pas maintenir la liste des états
visités afin de ne pas être limités par l'utilisation de la mémoire. On
parle alors de \emph{stateless model-checking}. Ces outils ne sont pas
capables de détecter un cycle dans le programme à analyser. Il est donc
nécessaire que toutes les exécutions du programme à analyser terminent.

\paragraph{State Hashing}
Le \emph{state hashing} consiste à ne conserver qu'une valeur de hachage des
états visités, et non pas l'état complet. L'utilisation de la mémoire est ainsi
réduite, et les performances peuvent être améliorées par l'utilisation de
structures de données capables de rechercher un élément de manière efficace
(ensemble ordonné, \dots). Cependant, il est possible (bien que très peu
probable) que deux états aient la même valeur de hachage. Ces états sont alors
considérés comme égaux, ce qui peut mener à un résultat faux de l'analyse.
Le state hashing rend donc les model-checkers incomplets.

\paragraph{Runtime model-checking}
Certains outils explorent l'espace d'états en se basant sur des
exécutions réelles du programme. Le modèle est alors le programme lui-même.
Son exécution est contrôlée par le model-checker. Ce dernier contrôle le
non-déterminisme de chaque exécution. Cette approche permet de
bénéficier des performances réelles du programme lors de l'exploration,
cependant le backtracking est plus compliqué : les outils de runtime
model-checking relancent en général une exécution du programme à tester
depuis l'état initial chaque fois qu'il est nécessaire de revenir à un
état précédent. Il est aussi complexe de mémoriser les états explorés,
cette technique est donc souvent combinée avec le \emph{stateless model
checking}.

\paragraph{Réduction par ordre partiel}
Les techniques de réduction par ordre partiel visent à réduire le nombre de
chemins à explorer dans un système concurrent. Elles permettent de supprimer les
chemins équivalents par rapport aux propriétés à vérifier, souvent en supprimant
des changements de contextes entre des instructions indépendantes. Un cas simple
de réduction consiste à regrouper en un bloc atomique une série d'instructions
manipulant uniquement des variables locales. L'entrelacement de ces instructions
avec d'autres threads n'a pas d'impact sur leur résultat, on peut donc conserver
uniquement l'ordonnancement consistant à les exécuter sans changement de
contexte.

La plupart des variantes ci-dessus sont aussi utilisées par les autres
algorithmes que nous présentons par la suite, en particulier les
techniques de réduction par ordre partiel. Cependant, le model-checking
explicite est particulièrement dépendant de ces techniques afin de
gagner en performances.

La principale force du model-checking explicite est sa précision. Explorer
l'ensemble des états de manière explicite permet de vérifier la plupart des
propriétés, spécifiées à l'aide d'assertions ou par une formule \ac{LTL}. Cependant,
il est extrêmement dépendant des techniques de réduction et a plus de
difficultés face à des programmes de grande taille.

\subsubsection{Outils}

SPIN\cite{SPIN} est l'un des premiers projets de model-checking
explicite. Il permet de vérifier des propriétés de la logique temporelle
\ac{LTL} sur des modèles exprimés en \bsc{PROMELA}. SPIN supporte nativement
les modèles concurrents et implémente de nombreuses méthodes de réduction
(ordre partiel, bit state hashing\dots). SPIN ne supporte pas le langage
C nativement, cependant des travaux ont été menés afin de traduire C
vers \bsc{Promela}\cite{jiang_C_to_Promela}, ce qui permet à SPIN de
vérifier un programme en C de manière indirecte. Cependant, cette traduction
ne supporte que partiellement les manipulations de la mémoire (pointeurs,
allocation dynamique\dots).

Pancam\cite{Pancam} se base sur SPIN pour vérifier du bytecode
LLVM\footnote{le bytecode LLVM peut être produit à partir d'un code C à
l'aide d'un compilateur tel que CLANG ou GCC}. Pancam exécute le
bytecode LLVM dans une machine virtuelle et utilise SPIN comme un
moniteur d'exécution afin de générer les différents entrelacements à
explorer.

\texttt{inspect}\cite{inspect} vérifie du code C et C++
multi thread en se basant sur un algorithme de runtime model-checking
avec une exploration \emph{stateless}. \texttt{inspect} instrumente
le programme à vérifier par des instructions lui permettant de
communiquer avec un ordonnanceur, selon une architecture client/serveur.
Le programme est ensuite exécuté pour chaque combinaison d'entrelacements,
afin d'explorer l'ensemble des traces possibles.

Divine\cite{Divine_3_0} est un model-checker capable de vérifier des
propriétés \ac{LTL} et des propriétés d'accessibilités. Il se base sur un
langage interne, DVE. Il est capable de traiter du bytecode LLVM en le
traduisant vers DVE, ce qui lui permet de supporter des langages comme C
et C++. La particularité de Divine est de mettre en place une analyse
concurrente et distribuée afin d'améliorer ses performances. Il utilise
aussi des méthodes de réduction de l'espace d'état (compression de
chemins, réduction par ordre partiel).

On retrouve à travers ces outils le besoin de performance des outils de
model-checking explicite. Divine est implémenté dans une architecture
concurrente pour améliorer ses performances, alors que \texttt{inspect} et
Pancam se basent sur du runtime model-checking. SPIN, \texttt{inspect}, Pancam
et Divine sont tous capables de vérifier des violations d'assertions et des
propriétés built-in, comme les deadlocks et les data-races, cependant, seul
Divine et SPIN sont capables de vérifier des propriétés \ac{LTL}.
\texttt{inspect} nécessite un système fermé, et n'est pas capable de gérer le
non-déterminisme. Il est donc dépendant d'une suite de tests pour fixer les
paramètres du programme à vérifier. Divine est arrivé sixième de la catégorie
portant sur les programmes concurrents lors de l'édition de 2016 de la
compétition de vérification logicielle SV-COMP\citep{svcomp_2016_result} (les
autres outils présentés ci-dessus n'ont pas participé à ces compétitions).

\subsection{Model-checking symbolique}

Les algorithmes de model-checking symbolique manipulent des ensembles d'états
du modèle représentés de manière abstraite, plutôt que d'énumérer chaque état
individuellement. Il est ainsi possible de manipuler des ensembles
d'états importants ou infinis de manière efficace. Contrairement au
model-checking explicite, le model-checking symbolique n'est donc pas limité à
un modèle fini.

L'abstraction utilisée pour représenter les ensembles d'états est un élément
clef du model-checking symbolique. Cette représentation doit permettre de
réaliser les opérations classiques sur les ensembles (union, intersection,
comparaison) de manière efficace. Les abstractions suivantes sont les plus
utilisées :

\begin{itemize}
\item
  les BDD (\emph{Binary Decision Diagrams}, Diagrammes de Décision Booléens) ;
\item
  les formules de la logique propositionnelle ;
\item
  automates finis.
\end{itemize}

Ces représentations permettent de définir des opérations efficaces sur les
ensembles. L'opération la plus complexe est généralement la comparaison de deux
ensembles, qui nécessite le choix d'une représentation canonique.

Cependant, la taille de la représentation d'un ensemble peut varier très
fortement selon le choix de la représentation canonique. Le défi du
model-checking symbolique est donc de construire et de maintenir efficacement
une représentation canonique de taille raisonnable.

\subsubsection{Représentation symbolique par des BDD}

Nous allons utiliser l'exemple des BDD pour illustrer les forces et les
défis rencontrés par les représentations symboliques.

Un BDD permet de représenter une fonction booléenne.
On va donc représenter un ensemble d'états par une fonction prenant
en paramètre un état (encodé par des variables booléennes) et indiquant s’il est dans l'ensemble représenté par la fonction ou non.

\paragraph{Encoder les états}
La première étape est d'encoder les états du système par des variables
booléennes. Une manière simple de le faire dans le cas d'un ensemble fini
d'états est de les numéroter, puis d'utiliser une variable booléenne pour chaque
chiffre de la représentation binaire (il faut donc \(log_2 n\) variables
booléennes pour \(n\) états).

Prenons l'exemple d'un système à huit états. En numérotant les états de 0 à 7 et
en prenant leur représentation binaire (sur trois bits), on obtient les
représentations suivantes : (\(000\), \(001\), \(010\), \dots, \(111\)).


Pour représenter l'ensemble composé des états \(110\) et \(111\), il suffit
de prendre sa fonction caractéristique, c'est-à-dire la fonction qui s'évalue à
vrai pour les éléments de l'ensemble et à faux sinon.

\begin{align}
  f \colon \{0, 1\}^3 & \to \{0, 1\} \\
  (x_1, x_2, x_3) & \mapsto
  \begin{cases*}
    1 & si \((x_1, x_2, x_3) \in \{(1,1,0), (1,1,1)\}\) \\
    0 & sinon
  \end{cases*}
\end{align}

Nous allons ensuite construire un BDD pour représenter cette fonction.

La construction d'un BDD repose sur la propriété suivante :

\paragraph{Décomposition des fonctions booléennes}
Soit \(k > 0\) et \(f: \mathbf{B}^k \to \mathbf{B}\) une
fonction booléenne. Alors il existe deux fonctions booléennes
\(g, h : \mathbf{B}^{k-1} \to \mathbf{B}\) telles que :

\[
f(x_1, ..., x_k) =
\begin{cases*}
  g(x_2, ..., x_k) \text{ si $x_0$ est vrai}\\
  h(x_2, ..., x_k) \text{ si $x_0$ est faux}
\end{cases*}
\]

Il est ainsi possible de définir une fonction booléenne par récursion
sur le nombre de ses variables. On va se servir de cette propriété pour
représenter la fonction booléenne par un arbre binaire de décision. On
fixe tout d'abord un ordre des variables (dans notre exemple, on prend
le bit de poids le plus fort en premier). On construit alors l'arbre
par induction :

\begin{itemize}
\item
  si la fonction est constante, on la représente par une feuille
  étiquetée par \(vrai\) ou \(faux\).
\item
  sinon, on décompose \(f\) entre \(g\) et \(h\) selon la propriété. On
  construit alors l'arbre dont le fils gauche est la représentation de
  \(g\) et le fils droit celle de \(h\) (construites selon la même méthode).
  On étiquette l'arrête vers le fils gauche par \(0\) et celle vers le fils
  droit par \(1\).
\end{itemize}

Les étiquettes d'un chemin de la racine de l'arbre à une feuille représentent
alors une valuation des variables de la fonction. L'étiquette de la feuille
indique la valeur de la fonction pour cette valuation.

Pour notre exemple, nous obtenons l'arbre de la Figure \ref{fig:BDD_tree}.

\begin{figure}[ht!]
\begin{center}
  \includegraphics[height=0.3\textheight]{BDD_tree.png}
\end{center}
\caption{Arbre binaire de la fonction \((x,y,z) \mapsto x \land y\).}
\label{fig:BDD_tree}
\end{figure}

La représentation par un arbre de décision contient \(2^{n+1} - 1\) nœuds pour
une fonction booléenne à \(n\) variables, on ne peut donc pas encore parler de
représentation efficace. Cependant, cet arbre de décision contient aussi
beaucoup de sous-arbres redondants. On va donc réduire cet arbre en un BDD en
fusionnant tous les sous-arbres identiques et en supprimant les états n'ayant
qu'un unique successeur. Dans notre exemple, on obtient alors le BDD de la
Figure \ref{fig:BDD_graph}.

\begin{figure}[h]
\begin{center}
\includegraphics[height=0.25\textheight]{BDD_graph.png}
\end{center}
\caption{BDD de la fonction \((x,y,z) \mapsto x \land y\).}
\label{fig:BDD_graph}
\end{figure}

Cependant, la taille du BDD est très dépendante de l'ordre choisi pour
les variables\cite{OBDD}. Dans cet exemple, l'ordre n'a pas une grande importance
mais lorsque le nombre de variables et la complexité de la fonction
augmentent, la taille du BDD peut exploser\footnote{Pour s'en
  convaincre, on peut calculer les BDD de la fonction
  \(x, y, z, t \mapsto x.z + y.t\), avec les ordres \(x<z<y<t\), puis
  \(x<y<z<t\).}.

À partir de la représentation sous forme de BDD, on peut évaluer la
fonction \(f\) efficacement, mais aussi réaliser des opérations logiques
performantes, en construisant le BDD résultant à partir des
feuilles et en réduisant au fur et à mesure. \cite{OBDD} explique cet
algorithme plus en détail . Dans le cas (simple) de la
négation, il suffit par exemple d'échanger les deux feuilles du BDD.

\paragraph{Vérification à l'aide des BDD}

On peut définir l'ensemble des états vérifiant une propriété \(\Phi\)
comme le point fixe d'une fonction (dépendant de \(\Phi\)). On calcule
alors ce point fixe de manière itérative, en utilisant les opérations définies
sur les BDD. Si l'état initial du système fait partie des états vérifiant la
propriété, le système est valide.

Par exemple, dans le cas d'une assertion, l'ensemble des états valides est
l'ensemble des états à partir desquels il n'est pas possible de déclencher
l'assertion. On peut le définir comme étant le point fixe de la fonction \(X
\mapsto E \cup X \cup Pre(X)\), avec \(X\) un ensemble d'états, \(E\) l'ensemble
des états d'erreurs et \(Pre\) la fonction qui associe à un ensemble d'états ses
prédécesseurs dans le modèle de système.

Tout comme les BDD, les autres représentations symboliques sont fortement
dépendantes d'un ordre des variables afin de conserver une taille
raisonnable. Les formules de la logique propositionnelle sont très
utilisées actuellement en raison des progrès des solveurs SAT.

Les techniques de model-checking symbolique sont efficaces afin de
prouver la validité d'une propriété (ce qui est le point faible du
model-checking de manière générale). Elles supportent bien l'explosion
combinatoire mais sont très dépendantes des performances des
représentations symboliques utilisées.

\subsubsection{Outils}

CIVL\cite{CIVL} est un framework pour la vérification de programmes concurrents.
Il définit un langage intermédiaire, CIVL-C, proche du C11, vers lequel plusieurs
langages et types de concurrence peuvent être traduits (dont pThread, OpenMP et
CUDA). CIVL utilise un mélange de model-checking explicite et symbolique : les
différents entrelacements d'un programme sont explorés de manière explicite et
une analyse symbolique est utilisée pour analyser chacun de ces entrelacements.
Les états sont représentés à travers des formules SMT, manipulées à l'aide de la
bibliothèque SARL\cite{SARL}. CIVL a pour ambition de constituer un point de
jonction commun entre plusieurs frontends et backends, afin de découpler les
problèmes de modélisation d'un langage et les problèmes de vérification. CIVL
adopte une approche conservatrice : il se veut correct mais il n'est pas
complet, il peut donc signaler des faux positifs.

SymDivine\cite{SymDivine} permet la vérification de propriétés \ac{LTL} sur du
bytecode LLVM concurrent (et donc, par extension, pour des programmes en C).
Tout comme CIVL, SymDivine utilise une combinaison de model-checking explicite
et symbolique. Le non-déterminisme au niveau des données est géré de manière
symbolique tandis que celui au niveau du contrôle est traité explicitement : tous
les entrelacements sont explorés individuellement. SymDivine peut utiliser les
formules SMT ou des BDD pour représenter les ensembles abstraits, les premières
étant plus efficaces dans la plupart de leurs expérimentations.

SymDivine et CIVL utilisent des méthodes de vérification semblables, mais
diffèrent par leurs langages de modélisation. CIVL utilise CIVL-C, un langage
relativement haut niveau pour faciliter la conception de modèles et la prise en
charge de plusieurs langages et paradigmes de concurrence. SymDivine
privilégie le bytecode LLVM, plus proche du programme compilé final afin de
réduire la distance entre le modèle et le système. CIVL a terminé troisième
(respectivement quatrième) lors des éditions 2016 (respectivement 2017) de
SV-COMP\cite{svcomp_2016_result, svcomp_2017_result}, dans la catégorie portant
sur la concurrence.

\subsection{Predicate abstraction}

L'abstraction par prédicat (\emph{predicate abstraction}) est une technique
proche de l'interprétation abstraite, dont nous avons parlé dans l'introduction
de ce mémoire. Elle consiste à construire un modèle du programme dans un
domaine abstrait, déterminé en partitionnant l'ensemble des états du système
selon un ensemble de prédicats. Le domaine abstrait utilisé par
l'interprétation abstraite est en général fixé par l'outil et
indépendant du programme analysé. Les techniques d'abstraction par prédicat
diffèrent sur ce point : elles construisent un modèle sur mesure pour le système
considéré et la propriété à vérifier\cite{abstract_state_graph, dsilva_survey_2008}.

L'objectif des techniques d'abstraction par prédicat est de construire une
abstraction du programme exacte par rapport à la propriété à vérifier. Cette
abstraction est réalisée en choisissant un certain nombre de prédicats portant
sur les états du programme. Le choix des prédicats est une étape cruciale,
puisque la précision de l'abstraction en dépend. Si l'abstraction n'est pas
suffisamment précise, des faux positifs ou des faux négatifs peuvent avoir lieu.

Une fois les prédicats choisis, les états de l'abstraction sont définis comme
les classes d'équivalence obtenues en partitionnant l'ensemble des états du
système par l'ensemble des prédicats. Une transition existe entre un état
abstrait \(A\) et un état abstrait \(B\) si une transition existe dans le
système d'origine entre un des états de \(A\) et un des états de \(B\).

On peut représenter le système abstrait comme un programme booléen, c'est-à-dire
un programme possédant les structures de contrôle classique du C, mais
uniquement des variables booléennes. Les variables de ce programme représentent
chacune un prédicat, et leur valeur dans un état du programme représente si
le prédicat est vérifié dans cet état ou non. On peut alors vérifier le système
abstrait à l'aide d'un model-checker pour programmes booléens (ceux-ci utilisent
fréquemment des méthodes de model-checking symbolique).

\paragraph{Exemple}

Considérons l'exemple du listing \ref{lst:bool_prog_init}. L'erreur qui a lieu
lorsque \texttt{x = 0} pourrait ici correspondre à une division par
zéro. Considérons les prédicats \(p_1 = (y == 1)\) et \(p_2 = (x == 0)\).

On détermine quatre états abstraits pour chaque position du programme,
correspondant aux différentes valeurs de \((p_1, p_2)\). On peut ensuite
déduire le programme booléen associé en supprimant les variables du programme
initial et en rajoutant des instructions pour actualiser les prédicats. On
obtient le programme booléen présenté dans le listing \ref{lst:bool_prog_bool}.

\noindent\begin{minipage}{.45\textwidth}
\begin{lstlisting}[language=C, caption=Code initial, frame=single, numbers=left,
    label=lst:bool_prog_init]
x = 3;
y = 2;
if (y == 1)
   x = 0;
if (x==0)
   assert(false);
\end{lstlisting}
\end{minipage}\hfill
\begin{minipage}{.45\textwidth}
\begin{lstlisting}[language=C, caption=Programme booléen,frame=single, numbers=left,
    label=lst:bool_prog_bool]
p_2 = false;
p_1 = false;
if (p_1)
    p_2 = true;
if (p_2)
    assert(false);
\end{lstlisting}
\end{minipage}

Ce programme peut facilement être analysé par un model-checker pour programmes
booléens.

Il reste cependant à déterminer automatiquement les prédicats utilisés. La
méthode la plus connue pour construire choisir les prédicats et construire une
abstraction dans le cas du model-checking logiciel est la méthode
\ac{CEGAR}\cite{dsilva_survey_2008}.

\subsubsection{Counter-Example Guided Abstraction Refinement (CEGAR)}

La méthode \ac{CEGAR} permet de construire une abstraction en itérant quatre phases :
Abstraction, Vérification, Simulation et Raffinement. Une première abstraction
(peu précise) du programme est réalisée. Si la vérification fournit un
contre-exemple, celui-ci est simulé dans le système initial afin de vérifier
qu'il y existe (afin d'éviter un faux-positif). Si ce n'est pas le cas, on
utilise cette information pour
construire une abstraction plus précise qui permet de rejeter cette trace et on
itère le processus.

Reprenons le programme du listing \ref{lst:bool_prog_init} comme exemple.

\begin{enumerate}
\def\labelenumi{\arabic{enumi})}
\item
  \textbf{Abstraction :} Dans un premier temps, on considère une abstraction
  basée sur un ensemble de prédicats vide. Il n'y a donc qu'une seule classe
  d'équivalence, contenant tous les états du programme. On ne conserve que les
  structures de contrôle du programme. Une analyse statique du programme peut
  aussi être utilisée pour définir ce premier ensemble de prédicats.
  Toutes les variables sont rendues abstraites, et leurs valeurs dans
  les structures de contrôle sont non déterministes. On obtient alors
  l'abstraction présentée dans le listing \ref{lst:CEGAR_1}.
\item
  \textbf{Vérification :} Un model-checker pour programmes booléens permet de
  trouver la trace exécutant les lignes 1, 2, 3, 5, et 6 qui mène à une erreur
  dans l'abstraction.
\item
  \textbf{Simulation :} Une simulation symbolique permet de constater que la
  trace n'est pas réalisable. En effet, \texttt{x} est non-nul lors
  du test à la ligne 5 donc la transition de la ligne 5 à la ligne 6 n'est pas
  réalisable.
\item
  \textbf{Raffinement :} L'incohérence présente dans la trace est utilisée pour
  raffiner l'abstraction, de manière à éliminer les chemins invalides. Ici, un
  prédicat \texttt{{x = 0}} indiquant si \texttt{x} est nul est ajouté. On
  obtient alors l'abstraction plus précise du listing \ref{lst:CEGAR_2}.
\item
  \textbf{Vérification :} Le model-checker trouve cette fois l'exécution
  passant par les lignes 1, 2, 3, 4, 5 et 6.
\item
  \textbf{Simulation :} cette trace est encore une fois incorrecte, puisque
  \texttt{y} ne vaut pas \texttt{1} à la ligne 3.
\item
  \textbf{Raffinement :} un prédicat \texttt{{y = 1}} est ajouté. On obtient
  l'abstraction du listing \ref{lst:CEGAR_3}.
\item
  \textbf{Vérification :} Le model-checker ne trouve pas de trace menant à une
  erreur dans l'abstraction. L'absence d'erreur est donc prouvée pour le
  programme.
\end{enumerate}

\noindent\begin{minipage}{.45\textwidth}
  \begin{lstlisting}[language=C, label=lst:CEGAR_1, numbers=left, frame=single,
    caption=Abstraction initiale]
;
;
if (*)
    ;
if (*)
   assert(false);
\end{lstlisting}
\end{minipage}\hfill
\begin{minipage}{.45\textwidth}
  \begin{lstlisting}[language=C, label=lst:CEGAR_2, numbers=left, frame=single,
    caption=Premier raffinement]
{x = 0} = {false};
;
if (*)
    {x = 0} = true;
if ({x = 0})
   assert(false);
\end{lstlisting}
\end{minipage}

\begin{minipage}{.45\textwidth}
\begin{lstlisting}[language=C, label=lst:CEGAR_3, numbers=left, frame=single,
  caption=Second raffinement]
{x = 0} = false;
{y = 1} = false;
if ({y = 1})
    {x = 0} = true;
if ({x = 0})
   assert(false);
\end{lstlisting}
\end{minipage}

La phase de raffinement est critique pour les performances de la méthode
\ac{CEGAR}, puisque améliorer le choix des nouveaux prédicats peut éviter des
itérations supplémentaires de l'algorithme. Des techniques d'analyse statique
sont souvent utilisées pour trouver les prédicats les plus forts éliminant un
chemin impossible du modèle. On peut en particulier calculer les plus faibles
préconditions d'un chemin impossible --- c'est-à-dire la plus faible propriété
qui implique que ce chemin va être emprunté.

\subsubsection{Lazy Abstraction with Interpolant (IMPACT)}

Les méthodes d'abstraction par prédicat, dont l'algorithme \ac{CEGAR}, se basent sur
le calcul d'une ou plusieurs abstractions du modèle. Calculer cette abstraction,
et en particulier les successeurs de chaque état abstrait est une tâche
coûteuse.

L'algorithme IMPACT\cite{IMPACT} permet d'éviter le calcul des successeurs des
états abstraits. Au lieu de raffiner une abstraction de manière itérative,
IMPACT raffine son modèle au fur et à mesure de sa construction, seulement
lorsque nécessaire. Pour ce faire, il déroule la relation de transition d'un
programme. L'arbre ainsi obtenu est ensuite exploré. Lorsqu'un état d'erreur
est atteint, les états du chemin vers cette erreur sont annotés par des invariants
prouvant que le chemin est impossible dans le programme d'origine (si de tels
invariants n'existent pas, une erreur est reportée). Le cœur de l'algorithme
IMPACT consiste à utiliser ces invariants pour éviter d'explorer des branches
entières de l'arbre. Si deux nœuds \(A\) et \(B\) partagent la même location et
que l'invariant de \(B\) est plus fort que celui de \(A\) (l'invariant de \(B\)
implique l'invariant de \(A\)), alors il
est inutile d'explorer les fils de \(B\) : une erreur accessible à partir de
\(B\) est accessible à partir de \(A\) aussi.

\subsubsection{Gestion de la concurrence}

Les techniques d'abstraction par prédicat reposent sur le fait que le problème
d'accessibilité dans un programme booléen séquentiel est décidable. On peut donc
déterminer si une trace menant à une erreur existe dans le modèle abstrait.
Cependant, le problème d'accessibilité dans un programme booléen concurrent
n'est pas décidable. C'est une grande difficulté pour les méthodes
d'abstraction par prédicats, qui reposent sur la terminaison et la rapidité de
l'analyse du programme booléen. C'est plus vrai encore pour l'algorithme \ac{CEGAR},
qui effectue plusieurs vérifications successives de ce programme.

Deux alternatives existent pour intégrer la concurrence dans les techniques
d'abstraction par prédicat :

\begin{enumerate}
\def\labelenumi{\arabic{enumi})}
\item
  Considérer les entrelacements du programme lors de la vérification du
  programme booléen. Pour pallier le problème de terminaison, il est
  possible de construire une sur approximation du programme booléen.
  Cette approximation permet d'assurer la terminaison du programme
  booléen, mais elle peut générer des faux positifs. Il reste possible
  de détecter ces faux positifs dans la partie \emph{Simulation} de
  l'algorithme \ac{CEGAR}\cite{predicate_abstraction_over_approximation}.
  Une autre solution est de limiter la profondeur d'exploration et le
  nombre de changements de contexte. Cependant, cette approche rend
  nécessairement la vérification incomplète.
\item
  Utiliser un raisonnement de type
  \emph{rely-garantee}\cite{thread_modular_abstraction} : les
  threads sont examinés un par un, de manière indépendante, et une approximation
  de l'effet des autres threads est construite pour examiner les différentes
  traces possibles.
\end{enumerate}

\subsubsection{Outils}

Impara\cite{Impara} adapte l'algorithme \emph{IMPACT}\cite{IMPACT} et le combine
à des techniques de réduction d'ordre partiel afin de l'étendre à des programmes
multi thread. Il réduit ainsi l'explosion combinatoire présente dans l'algorithme
IMPACT : dérouler naïvement la relation de transition d'un modèle concurrent,
comme le fait une extension directe de l'algorithme IMPACT, revient à
énumérer un nombre exponentiel d'entrelacements.

La plupart des model-checkers utilisant la méthode \ac{CEGAR} basent leurs procédures
de décision sur des prouveurs limités à l'arithmétique linéaire sur des nombres
réels uniquement. Ils ne peuvent alors pas raisonner sur des tableaux de bits.
Satabs\cite{Satabs}\cite{clarkesatabs} implémente ses procédures de décision à
l'aide d'un solveur SAT. Il parvient ainsi à manipuler des tableaux de bits, et
à représenter des structures plus complexes du langage C. Il gagne en précision
sur l'arithmétique des pointeurs en particulier. Satabs est ainsi capable de
vérifier un certain nombre de propriétés built-in (validité des pointeurs,
erreurs arithmétiques, accès hors des bornes d'un tableau), en plus de
propriétés spécifiées par des assertions.

Threader\cite{Threader}\cite{Threader_theory} se base aussi sur la méthode
\ac{CEGAR}. Cependant, il utilise des techniques basées sur des raisonnements de type
\emph{Rely-guarantee} afin de rechercher une preuve modulaire du programme.
Threader gagne ainsi en performance lorsqu'une preuve modulaire existe, mais
il peut être plus lent qu'une méthode classique dans les autres cas.

SKINK\cite{SKINK} permet vérifier des propriétés exprimées par des assertions à
partir de la représentation intermédiaire de LLVM. SKINK se base sur une
approche de type CEGAR, mais il raffine itérativement une abstraction de
l'ensemble des traces d'exécution au lieu de travailler sur une abstraction des
états. Les interpolants renvoyés par les solveurs SAT sont utilisés pour
construire un automate qui vient restreindre le modèle. SKINK prends en charge
la concurrence en construisant le produit entre les automates représentant
chaque threads. Des méthodes d'ordre partiel sont utilisées pour limiter
l'explosion combinatoire.

CPAchecker\cite{CPAChecker} est un framework qui a pris la suite de l'outil
BLAST \cite{BLAST}. Il permet une analyse configurable travers des CPAs
(\emph{Configurable program analysis}). Un CPA définit un domaine abstrait
(BDD, analyse de valeurs, analyse d'intervalles\dots). CPAchecker est capable
de mener une analyse d'accessibilité sur une combinaison arbitraire de CPAs. La
concurrence est prise en charge à l'aide d'un CPA qui définit une exploration
explicite des entrelacements\cite{CPAChecker_multithread}. De manière similaire
à CIVL, CPAchecker vise à réduire les efforts de développement nécessaires pour
tester de nouvelles approches pour les méthodes basées sur l'abstraction par
prédicat. CPAchecker permet de construire une spécification sous la forme d'un
automate, à l'aide d'un formalisme similaire à celui utilisé par BLAST.

Les techniques d'abstraction par prédicat sont principalement tournées vers les
propriétés d'accessibilité spécifiées par des assertions. Parmi les outils
supportant le C multi thread, tous prennent en charge les assertions. Cependant, seul
Satabs est orienté vers la vérification de propriétés built-in et aucun ne
permet de vérifier des propriétés \ac{LTL}.

De manière générale, les techniques d'abstraction par prédicat sont très
efficaces dans le cas du model-checking de logiciels séquentiels, car elles
supportent bien l'explosion combinatoire liée aux données. Cependant, elles ont
plus de difficultés à traiter des structures de données complexes (tableaux et
allocations dans le tas). La vérification de programmes concurrents constitue
aussi une difficulté majeure.

\subsection{Model-checking borné}

Le model-checking borné, ou \ac{BMC}, est l'une des
techniques les plus utilisées dans l'industrie des semi-conducteurs. Elle a été
adaptée dans le cas du model-checking logiciel. Le model-checking borné se base
sur les avancées des solveurs SAT et SMT pour effectuer les tâches de
vérification.

Au lieu d'explorer entièrement un modèle, les algorithmes de model-checking
bornés explorent uniquement des préfixes finis des exécutions possibles du
modèle. Une profondeur maximale d'exploration \texttt{k} est fixée.
Les \texttt{k} premiers pas de chaque exécution du système sont alors
encodée dans une formule de la logique propositionnelle.
On construit ensuite la conjonction entre cette formule et la négation d'une
formule représentant la spécification du programme, et on transmet le résultat
à un solveur SAT ou SMT. Si la formule est satisfiable, il existe une
exécution valide ne respectant pas la spécification (puisqu’elle satisfait la
négation de la spécification). Une erreur est donc reportée, la valuation qui a
satisfait la formule représente le contre-exemple.

\paragraph{Forme SSA}
Afin d'encoder les exécutions d'un programme dans une
formule, on met tout d'abord ce programme dans la forme \ac{SSA}.
Elle consiste à transformer le programme de telle sorte que chaque
variable ne soit assignée qu'une seule fois. Pour ce faire, les variables sont
dupliquées, une nouvelle copie étant utilisée à chaque assignation. Les
structures de contrôle sont remplacées par des expressions conditionnelles une
fois que les différents branchements se rejoignent. Les fonctions sont inlinées
--- leur corps est inséré directement à l'emplacement de l'appel --- et les
boucles sont déroulées un nombre de fois limité par la profondeur d'exploration,
en une cascade de blocs \texttt{if}.

Le passage en forme \ac{SSA} est illustré par l'exemple présenté dans les
listings \ref{lst::ssa_init} et \ref{lst::ssa_end}.

\noindent\begin{minipage}{.45\textwidth}
\begin{lstlisting}[language=C, caption=Code initial, frame=tlrb, label=lst::ssa_init]
int a = 0, b = 1, c;
a = a + b;
if (a > 0)
    c = 2;
else
    c = 3;
\end{lstlisting}
\end{minipage}\hfill
\begin{minipage}{.45\textwidth}
\begin{lstlisting}[language=C, caption=Forme SSA,frame=tlrb, label=lst::ssa_end]
int a_0 = 0;
int b_0 = 1;
int a_1 = a_0 + b_0;
int c_0 = 2;
int c_1 = 3;
int c_2 = a_1 > 0 ? c_0 : c_1;
\end{lstlisting}
\end{minipage}

\paragraph{Dépliage de la relation de transition}
Dans le cadre du \ac{BMC}, le programme est vu comme une relation de transition
\(R\). Étant donnés deux états \(s\) et \(q\) du programme, \(R(s, q)\) s'évalue
à vrai si et seulement s’il existe une transition dans le programme permettant
de passer de l'état \(s\) à l'état \(q\). On peut alors établir la formule
représentant les \(k\) premiers pas d'une exécution par :

\[
F = I(s_0) \land \bigwedge_{i\in \{1..k\}} R(s_{i-1}, s_i)
\]

Le solveur va tenter de résoudre la formule en trouvant une suite d'états
\(s_i\) qui la satisfait. Cette suite représente alors une exécution possible
dans le programme : \(I(s_0)\) impose que l'exécution commence dans un état
initial, et \(\bigwedge_{i\in \{1..k\}} R(s_{i-1}, s_i)\) impose que les
transitions entre deux états successifs de la trace soient valides.

Partant de la forme \ac{SSA}, la relation de transition se déplie facilement
en construisant la conjonction des instructions. Les variables de la forme \ac{SSA}
peuvent directement être considérées comme des variables logiques. La principale
difficulté consiste à établir une théorie pour représenter les
instructions, en particulier les utilisations dynamiques de la mémoire.

Dans notre exemple, la relation de transition se déplierait en :

\[
  \underbrace{a_0 = 0 \land b_0 = 1}_\text{État initial} \land
  \underbrace{a_1 = a_0 + b_0}_\text{Instruction 1} \land
  \underbrace{c_0 = 2 \land c_1 = 3}_\text{Branches du tests} \land
  \underbrace{(a_1 > 0 \implies c_2 = c_0)
    \land (\lnot a_1 > 0 \implies c_2 = c1)
  }_\text{Fusion des branches}
\]


Afin de vérifier une propriété \(\Phi\) du programme, on construit la
conjonction entre la formule représentant les exécutions du système et la
négation de la propriété.

\[
F' = \lnot \Phi \land I(s_0) \land \bigwedge_{i\in \{1..k\}} R(s_{i-1}, s_i)
\]

Un modèle de cette formule représente donc les \(k\) premiers pas d'une
exécution valide, et qui ne respecte pas la propriété \(\Phi\). Il s'agit donc
d'une exécution contenant une erreur.

Le \ac{BMC} explore un système sur une profondeur bornée, il est donc nécessairement
incomplet. Il permet de trouver des erreurs, ou de prouver des propriétés
d'accessibilité, mais il ne peut prouver des propriétés de sûreté ou trouver des
erreurs à des propriétés de vivacité : une erreur ou l'évènement attendu
pourraient être présents à une profondeur plus importante que la profondeur
maximale d'exploration.

Lorsque le modèle contient un nombre d'états finis, il est possible de rendre le
\ac{BMC} complet : il suffit de fixer une profondeur maximale d'exploration
suffisamment grande pour que toutes les traces du programme n'affichent plus de
nouveaux comportements une fois cette profondeur atteinte. Cette profondeur est appelée le seuil de
complétion (\emph{completeness threshold}). Calculer la valeur exacte du seuil de
complétion est un problème difficile. On utilise donc des sur approximations.
Cependant, l'exploration d'un système jusqu'au seuil de complétion est souvent
trop coûteuse en temps et en mémoire pour être réalisable. Le \ac{BMC} est donc
souvent utilisé pour détecter des erreurs dans un système plutôt que pour
prouver leur absence, en effectuant plusieurs explorations avec une profondeur
maximale croissante.

\paragraph{Support du multi threading}

Plusieurs approches ont été développées pour prendre étendre le \ac{BMC} à des
logiciels multi thread\cite{ESBMC_multithread}.

Deux techniques sont principalement utilisées.
La première consiste à encoder tout les entrelacements possibles entre les threads
du programme dans la relation de transition, et donc dans la formule de la
logique propositionnelle fournie à un SAT-solveur ou un SMT-solveur. Cette
technique permet au solveur d'optimiser sa preuve en utilisant les similarités
entre différents entrelacements. Cependant, les formules générées peuvent être
extrêmement grandes et dépasser les capacités des solveurs.

La seconde approche se base sur l'exploration explicite des entrelacements du
programme : une formule est passée au solveur pour chaque entrelacement, le
procédé étant interrompu dès qu'une erreur est détectée. Les formules générées
sont ainsi plus petites et plus simples à résoudre pour les solveurs, cependant,
un nombre exponentiel d'appels au solveur peut être nécessaire, avec des
formules parfois très semblables.

Le nombre de changements de contexte autorisés dans une exécution est
généralement borné de manière indépendante par rapport à la profondeur
d'exploration. On évite ainsi d'explorer un nombre trop élevé d'entrelacements.
Cette seconde borne repose sur l'observation empirique : la plupart des erreurs
dues au parallélisme sont peu \emph{profondes}, elles peuvent être atteintes en
un nombre réduit de changements de contexte.

Le model-checking borné est la méthode la plus efficace pour trouver des
erreurs peu profondes. Elle supporte bien le multi threading. Cependant,
cette méthode est peu efficace pour prouver l'absence d'erreur et ne
peut généralement valider un programme que si sa profondeur est faible.
Elle est particulièrement sensible aux boucles dont le nombre
d'itérations n'est pas borné.

\subsubsection{Outils}

CBMC\cite{CBMC} est le premier model-checker logiciel à avoir utilisé le
model-checking borné. Il est orienté vers les applications embarquées, et permet
de vérifier un programme par rapport à différents modèles d'exécution. Il se
base sur des SAT-solveurs afin de vérifier les conditions de vérification qu'il
génère. CBMC remplace les opérateurs arithmétiques par la description de
circuits équivalents afin de construire une formule logique
(\emph{bit-flattening}). CBMC supporte les programmes multi thread en encodant
symboliquement les entrelacements dans la formule passée au model-checker. Comme
la majorité des model-checkers, CBMC supporte la consistance séquentielle (i.e.
un ordre total entre les instructions est considéré). Cependant, CBMC est aussi
capable de traiter des modèles mémoires faibles (i.e. un ordre partiel entre les
instructions est considéré). En effet, les multiprocesseurs modernes ne
respectent pas la consistance séquentielle : d'autres comportements peuvent
apparaître, liés à la propagation des valeurs à travers les niveaux de cache.

ESBMC\cite{ESBMC} a été conçu en se basant sur CBMC, dont il reprend le
frontend. ESBMC utilise un solveur SMT (SAT Modulo Theory) au lieu d'un solveur
SAT pour vérifier les formules générées et utilise des théories différentes de
celles utilisées par CBMC afin de créer ces formules. ESBMC supporte les
programmes multi thread en énumérant explicitement tous les entrelacements
possibles du programme, et en effectuant une exploration symbolique sur chacun
d'entre eux. ESBMC améliore le support des boucles par l'utilisation de preuves
par induction\cite{ESBMC_k_induction}. Les boucles sont déroulées sur un nombre
\(k\) d'itérations. Si cela ne suffit pas à trouver un contre-exemple, ESBMC va
tenter d'établir un invariant en montrant qu'il est maintenu tout au long de la
boucle par une preuve par induction.

SMACK\cite{SMACK} permet de traduire un programme en C vers le langage de
modélisation Boogie. Il utilise pour cela la représentation
intermédiaire LLVM-IR et implémente des simplifications du code source pour
réduire le modèle. Le modèle en Boogie est ensuite vérifié à l'aide d'un
model-checker utilisé en backend. Par défaut, Coral est utilisé. Il s'agit d'un
model-checker borné se basant sur des solveurs SMT pour les programmes en
Boogie, supportant la concurrence. Ceci permet à SMACK de tirer parti de
backends éprouvé et des performances des solveurs. SMACK reste cependant
modulaire, il peut ainsi utiliser d'autres algorithmes que du BMC pour le
backend. SMACK est capable de vérifier des spécifications sous forme d'assertions
uniquement.

\subsection{Séquentialisation}

Les techniques de model-checking logiciel pour des programmes séquentiels ont
largement progressé au niveau de leurs performances et de leur précision.
Cependant, les model-checkers restent bien plus limités dans le cas de
programmes concurrents. La séquentialisation réduit le problème de la
vérification d'un programme multi thread à celui de la vérification d'un
programme séquentiel pour tirer partit des performances des outils dans ce cas.

La séquentialisation consiste à traduire un programme concurrent en un programme
séquentiel non déterministe équivalent. Le non-déterminisme lié au contrôle (les
différents entrelacements) est remplacé par un non-déterminisme lié aux données,
ce qui permet de mieux le gérer. Les techniques de séquentialisation ne
s'appliquent donc qu'à un programme multi thread, il est nécessaire de les
combiner avec un outil de model-checking adapté au cas séquentiel pour réaliser
les tâches de vérification.

La séquentialisation permet d'attaquer deux problèmes simultanément : améliorer
la prise en charge de la concurrence en se ramenant au cas séquentiel, et
proposer une interface de gestion de la concurrence compatible avec plusieurs
model-checkers, sans que ceux-ci n'aient à modifier leurs algorithmes.

Nous allons énumérer les idées clefs des algorithmes de séquentialisation, tout
en les illustrant par l'exemple de la transformation réalisée par
Lazy-CSeq\cite{LazyCSeq}.

Partons d'un programme concurrent très simple (nous nous inspirons ici
de l'exemple présenté dans \cite{LazyCSeq}, auquel on peut se
référer pour plus de détails) :

\begin{lstlisting}[language=C, frame=single]
int y = 0;
void* T0(void *d) {
  int x;
  x = y;
  y = x + 1;
}
void* T1(void *d) {
  y = 3;
}
int main() {
  pthread_t t0, t1;
  pthread_create(t0, NULL, T0, NULL);
  pthread_create(t1, NULL, T1, NULL);
}
\end{lstlisting}

Les idées clefs de la séquentialisation sont les suivantes :

\begin{enumerate}
\def\labelenumi{\arabic{enumi})}
\item
  Remplacer les structures concurrentes par des structures séquentielles
  (un thread est remplacé par une fonction\dots)
\item
  Instrumenter le code afin de simuler les changements de contexte.
\item
  Ajouter un ordonnanceur non-déterministe. Il dirige l'exécution et
  permet de générer tous les entrelacements (généralement de type
  circulaire) du programme.
\item
  Recourir à un model-checker pour explorer le résultat.
\end{enumerate}

Dans notre exemple, le premier point consiste à changer la fonction
\texttt{main}, qui représente le thread principal, en une fonction
séparée \texttt{main\_thread}. Les autres threads sont déjà
représentés par une unique fonction, il n'est donc pas utile de les
modifier (si deux threads exécutaient la même fonction, il serait
nécessaire de la dupliquer). La seconde étape est réalisée en
introduisant un saut conditionnel avant chaque instruction du programme,
à travers la macro \texttt{J}. Cette macro permet d'interrompre et de
reprendre l'exécution d'une fonction à partir d'une instruction donnée,
en sautant les instructions qui précèdent ou suivent. La variable
\texttt{cs} contient à chaque instant le numéro du label
correspondant à la prochaine interruption, et le tableau \texttt{pc}
contient le numéro du label de la dernière instruction atteinte lors de
l'exécution précédente. Pour conserver les valeurs des variables locales
pendant les interruptions, ces dernières sont rendues statiques. Enfin,
la fonction \texttt{main} contient l'ordonnanceur. Il va simuler
\texttt{k} passes d'un ordonnancement de type round-robin, en
choisissant de manière non déterministe le nombre d'instructions à
exécuter dans chaque thread avant de changer de contexte.

On obtient alors le code séquentialisé du listing \ref{lst:code_seq}.
Des structures de contrôle supplémentaires permettent de représenter la
synchronisation des threads.

\begin{lstlisting}[language=C, label=lst:code_seq, frame=single,
  caption=Code séquentialisé, float=*]
int cs,ct,pc[T],size[T]={2,4,7};
#define J(A,B) if(pc[ct]>A||A>=cs) goto _##B;
int y=0;
void T0(void *arg) {
          static int x;
_0:J(0,1) x = y;
_1:J(1,2) y = x + 1;
_2: ;
}
void T1(void *arg) {
_3:J(3,4) y = 3;
_4: ;
}
int main_thread() {
          static pthread_t t0,t1;
_5:J(5,6) pthread_create(&t0,NULL,T0,0,1);
_6:J(6,7) pthread_create(&t1,NULL,T1,0,2);
_7: ;
}
int main() {
  int K = 10; // Check over 10 rounds
  int r;
  for(r=1; r<=K; r++) {
    cs=pc[0] + nondet_uint();
    assume(cs<=size[0]);
    main_thread();
    pc[0]=cs;
    cs=pc[1] + nondet_uint();
    assume(cs<=size[1]);
    T1();
    pc[1]=cs;
    cs=pc[2] + nondet_uint();
    assume(cs<=size[2]);
    T1();
    pc[2]=cs;
  }
}
\end{lstlisting}

La séquentialisation ne préserve pas nécessairement toutes les
propriétés du code, selon les transformations effectuées. Les propriétés
d'accessibilité sont généralement préservées, cependant la plupart des
approches déplient les boucles et bornent la profondeur d'exploration et
le nombre de changements de contexte possibles.

\subsubsection{Outils}

CSeq\cite{CSeq} suit le schéma de séquentialisation de Lal/Reps. Ce dernier
consiste à considérer \(K\) copies de la mémoire partagée, pour \(K\) phases
d'un ordonnancement \emph{round-robin}. La fonction représentant chaque thread
va mettre à jour la i\textsuperscript{ème} copie jusqu'au i\textsuperscript{ème}
changement de contexte, avant de la passer au thread suivant. Enfin, toutes les
exécutions incohérentes (lorsque l'état de la mémoire à la fin d'un cycle ne
correspond pas à l'état deviné au début du cycle suivant) sont supprimées. CSeq
utilise des model-checkers bornés pour mener la vérification. Il est compatible
avec CBMC, ESBMC et LLBMC.

Lazy-CSeq\cite{LazyCSeq} se base sur le framework mis en place par CSeq mais
prend une approche paresseuse (\emph{lazy}). Il ajoute des instructions de
contrôle aux fonctions représentant les threads, ce qui permet d'interrompre ou
de reprendre l'exécution d'une fonction chaque fois qu'un changement de contexte
peut avoir lieu. L'ordre d'exécution des instructions dans un entrelacement est
ainsi préservé, ce qui permet de réduire considérablement l'utilisation du
non-déterminisme. Lazy-CSeq est compatible avec CBMC, ESBMC et LLBMC.

Mu-CSeq\cite{MuCSeq} se base sur le concept de \emph{memory
unwinding} (déroulement de la mémoire, MU). Un MU consiste en une
séquence d'écriture dans la mémoire partagée du programme. Mu-CSeq
choisit un MU de manière non déterministe et le confronte à une
simulation du programme afin de vérifier si les écritures correspondent.
Si c'est le cas, une exécution valide a été détectée. Cette approche
permet d'éviter d'explorer des entrelacements équivalents, seul l'entrelacement
des écritures étant pris en compte. Mu-CSeq utilise CBMC comme backend.

Unbounded-Lazy-CSeq\cite{ULCSeq} se base sur l'algorithme utilisé
par Lazy-CSeq, mais permet de ne pas limiter le nombre de changements de
contexte. Il est aussi capable de traiter des programmes non bornés.
Pour cela, UL-CSeq conserve les boucles mais se base sur
CPAChecker\cite{CPAChecker} en backend et non sur des model-checkers bornés.

De manière générale, les outils de séquentialisation ont obtenu
d'excellentes performances ces dernières années. Lazy-CSeq a obtenu la
médaille d'or de la catégorie \emph{concurrence} lors des éditions 2014
et 2015 de la SVCOMP et une médaille d'argent pour les éditions 2016 et
2017, tandis que Mu-CSeq a obtenu la médaille d'or lors de l'édition
2016 et la médaille d'argent lors des éditions 2014 et 2015
\cite{svcomp_2016_result, svcomp_2015_result,
svcomp_2014_result, svcomp_2017_result}.

\subsection{Bilan}

Différents outils et techniques de model-checking ont été développés pour
tenter de résoudre le problème principal du model-checking : l'explosion
combinatoire. Ces techniques ont différents points forts et défauts, qui
dépendent des propriétés types à vérifier, de la structure du modèle (boucles,
non-déterminisme sur les données ou  le contrôle...) et du résultat attendu
(détection d'erreurs, preuve de correction...).

Afin de gagner en performances, les différents outils implémentent des
optimisations et des techniques de réduction qui provoquent en général une
certaine perte de précision. La plupart des outils restreignent ainsi le type de
propriétés et de spécification qu'ils supportent afin de gagner en performances.
La figure \ref{tab:prop_type_table}, en annexe, récapitule les types de
propriétés qu'il est possible de vérifier avec chaque outil (selon les articles
que nous avons passés en revue). On remarque en particulier que si la plupart
des outils supportent une spécification à l'aide d'assertions, le support pour
\ac{LTL} est bien plus réduit.

La gestion de la concurrence reste encore un problème ouvert.
Elle est actuellement prise en charge par différents mécanismes : les
différents entrelacements peuvent être intégrés directement dans le modèle, ils
peuvent aussi être énumérés explicitement. Une autre approche est de mener une
preuve modulaire, en examinant chaque thread individuellement et en simulant ses
interactions avec le reste du système. Enfin, les techniques de
séquentialisation permettent de ramener le non-déterminisme dû aux
entrelacements à du non-déterminisme dû aux données que les model-checkers
savent mieux traiter.

Par la suite, nous allons utiliser des model-checkers en tant que backend
afin de vérifier du code instrumenté. L'objectif sera alors de considérer le
model-checker comme une boîte noire, dont il est inutile de connaître le
fonctionnement interne. Connaître les techniques utilisées restera cependant
utile pour comprendre les résultats et les performances obtenus.  % Revue de littérature.
\chapter{LTL avec support des positions et des variables locales}\label{sec:Theme1}

\section{Limitations des méthodes de spécification actuelles}

À travers la revue de littérature, nous avons dégagé deux principales
limitations concernant les méthodes de spécification utilisées pour
model-checking de programmes concurrents.

La première limitation concerne les assertions. Les assertions sont le mécanisme
de spécification le plus utilisé dans le cadre du model-checking logiciel.
Cependant, les propriétés que l'on peut spécifier par des assertions sont
limitées à un sous-ensemble des propriétés de sûretés --- une assertion spécifie
qu'une certaine condition doit être vraie à une certaine position du code. Dans
le cas de programmes concurrents, ce type de propriétés est souvent insuffisant.

En effet, l'entrelacement entre les instructions d'un programme concurrent est
généralement complexe. Il est donc fréquent de vouloir vérifier des propriétés
portant sur l'ordre d'apparition de certains évènements au cours d'une exécution
du programme. Les assertions ne sont pas adaptées pour spécifier ce type de
propriétés.

\paragraph{Exemple}
L'absence de conditions de concurrence dans un programme est une propriété
généralement désirable. Elle fait partie de l'ensemble plus vaste des propriétés
d'exclusion mutuelle : il s'agit de vérifier que deux instructions accédant à
une même variable partagée ne peuvent être exécutées simultanément. Les listings
\ref{lst:data-race_t1} et \ref{lst:data-race_t2} présentent un cas très simple
de data-race.

\noindent\begin{minipage}{.45\textwidth}
  \begin{lstlisting}[language=C, frame=single, caption=Thread 1,
    label=lst:data-race_t1]
int p = 0;
void* thread1(void* d) {
  ...
  p += 1;
  ...
}
\end{lstlisting}
\end{minipage}\hfill
\begin{minipage}{.45\textwidth}
\begin{lstlisting}[language=C, frame=single, caption=Thread 2,
    label=lst:data-race_t2]

void* thread2(void* d) {
  ...
  p += 1;
  ...
}
\end{lstlisting}
\end{minipage}

Si les instructions \texttt{p += 1} des deux threads ont lieu simultanément, la
valeur finale de \texttt{p} est indéterminée.

Spécifier l'absence de data-races dans ce programme à l'aide d'assertions n'est
pas immédiat : les assertions ne permettent pas d'exprimer la notion de
simultanéité. Il est nécessaire d'introduire des variables supplémentaires, et
donc de modifier le système. La spécification la plus simple que nous avons pu
produire est présentée dans les listings \ref{lst:data-race_assert1} et
\ref{lst:data-race_assert2}\footnote{On remarque que cette spécification permet
  d'atteindre une race-condition sans violer les assertions. Ce n'est pas un
problème ici, car s’il est possible d'atteindre une race-condition, alors au
moins un entrelacement viole les assertions, un model-checker signalera donc
correctement l'erreur.}.

\noindent\begin{minipage}{.45\textwidth}
  \begin{lstlisting}[language=C, frame=single, caption=Thread 1,
    label=lst:data-race_assert1]
int p = 0;
int flag = 0;
void* thread1(void* d) {
  ...
  assert(!flag);
  flag = 1;
  p += 1;
  flag = 0;
  ...
}
\end{lstlisting}
\end{minipage}\hfill
\begin{minipage}{.45\textwidth}
\begin{lstlisting}[language=C, frame=single, caption=Thread 2,
    label=lst:data-race_assert2]


void* thread2(void* d) {
  ...
  assert(!flag);
  flag = 1;
  p += 1;
  flag = 0;
  ...
}
\end{lstlisting}
\end{minipage}

Cet exemple souligne le fait que, dans un programme concurrent, on
s'intéresse fréquemment au lien entre plusieurs \emph{assertions} plutôt qu'au
déclenchement d'une assertion simple. On aimerait pouvoir formuler des
propriétés telles que ``si deux assertions sont violées simultanément, une
erreur est présente'' ou ``si une assertion est violée, puis une seconde
assertion est violée, une erreur est présente''.

La seconde limitation concerne les spécifications utilisant \ac{LTL}.
Les model-checkers logiciels supportant \ac{LTL} restreignent généralement les
propositions atomiques à des fonctions booléennes portant sur la valeur des
variables globales du programme.
Si toutes les variables globale ont la même valeur dans deux états distincts du
programmes (qui diffèrent par leur position dans le code et leurs variables
locales), alors toutes les formules \ac{LTL} vont s'évaluer de la même manière
dans ces deux états : il n'est pas possible de les distinguer dans la
spécification, ce qui limite le type de propriétés qu'il est possible de spécifier.

En particulier, ces restrictions rendent impossible d'exprimer une assertion par
une formule \ac{LTL}. Alors que les logiques temporelles permettent d'exprimer
facilement des relations d'ordre entre différents évènements, \ac{LTL} n'est donc
pas une solution aux limitations rencontrées par les assertions.

\paragraph{Exemple}
Reprenons l'exemple précédent. La propriété d'exclusion mutuelle que l'on veut
spécifier peut s'écrire \(G \lnot (p_1 \land p_2)\) où \(p_1\) (respectivement
\(p_2\)) est la proposition atomique indiquant si le thread 1 (respectivement le
thread 2) est dans sa zone critique. Mais on ne peut pas exprimer \(p_1\) et \(p_2\)
directement, sans mentionner des positions du programme. Encore une fois, il est
nécessaire d'introduire des variables supplémentaires et de modifier le système.
Les listings \ref{lst:data-race_ltl1} et \ref{lst:data-race_ltl2} présentent une
solution, associée avec la spécification \(G \lnot (flag1 == 1 \land flag_2 ==
1)\).

\noindent\begin{minipage}{.45\textwidth}
  \begin{lstlisting}[language=C, frame=single, caption=Thread 1,
    label=lst:data-race_ltl1]
int p = 0;
int flag1 = 0;
void* thread1(void* d) {
  ...
  assert(!flag);
  flag1 = 1;
  p += 1;
  flag1 = 0;
  ...
}
\end{lstlisting}
\end{minipage}\hfill
\begin{minipage}{.45\textwidth}
\begin{lstlisting}[language=C, frame=single, caption=Thread 2,
    label=lst:data-race_ltl2]

int flag2 = 0;
void* thread2(void* d) {
  ...
  assert(!flag);
  flag2 = 1;
  p += 1;
  flag2 = 0;
  ...
}
\end{lstlisting}
\end{minipage}

Les deux mécanismes de spécification les plus utilisés par le model-checking
logiciel sont donc limités, sans être complémentaires. Certaines propriétés sont
alors complexes à spécifier. En pratique, le cas des data-races présenté en
exemple est suffisamment pour être implémenté en tant que propriété built-in,
mais ce n'est pas le cas pour d'autres propriétés (d'exclusion mutuelle ou non).

Dans ce chapitre, nous allons présenter un nouveau formalisme de spécification
permettant de remédier à ce problème. Il se base sur une restriction de \ac{LTL}
plus souple que celle utilisée classiquement par les model-checkers logiciels.
Nous introduisons le concept de \emph{zones de validités} afin de permettre
l'utilisation de variables locales et de positions dans les propositions
atomiques. Nous produisons ainsi un formalisme de spécification qui englobe les
assertions et les propositions LTL (telles que classiquement utilisées par les
model-checkers logiciel), tout en permettant de surmonter les limitations
rencontrées par les utilisations classiques de LTL : l'utilisation des variables
locales et des positions du code source dans la spécification.

Nous allons tout d'abord présenter les différents obstacles à résoudre afin
d'atteindre cet objectif. Nous présenterons ensuite les solutions (parfois partielles)
que nous apportons à ces problèmes, ainsi que le formalisme de spécification que
nous avons établi. Enfin, nous justifierons nos choix et nous les comparerons
aux autres travaux connexes.

\section{Comment spécifier les propriétés d'un code ?}

Différents problèmes doivent être surmontés afin de permettre l'utilisation de
variables locales et de positions dans une spécification.
Limiter \ac{LTL} à des propositions atomiques portant sur les variables globales
a permis d'éviter ces problèmes, qui sont spécifiquement liés aux variables
locales et aux positions.

\subsection{Désignation des variables d'un programme}

Pour utiliser une variable dans la spécification d'un programme, il est
nécessaire de la désigner de manière unique.
Le language C n'autorise qu'une seule variable globale portant un nom
donné dans un fichier\footnote{Les mots clefs \texttt{extern} et
\texttt{static} permettent de déclarer plusieurs variables globales
portant le même nom dans un programme}. Une variable globale peut donc être
identifiée de manière unique par son nom dans la plupart des cas (en ajoutant le
nom du fichier où elle est déclarée si nécessaire). Ce n'est pas
aussi simple lorsqu'on considère des variables locales : deux variables
locales portant le même nom peuvent cohabiter sans faire référence au
même emplacement mémoire.

Deux variables locales distinctes (ne faisant pas référence à la même
adresse mémoire), mais ayant le même nom peuvent survenir dans diverses situations,
parmi lesquelles :

\begin{itemize}
\item
  une variable \texttt{foo} est définie dans plusieurs contextes
  distincts (des fonctions différentes, des blocs différents d'une même
  fonction\ldots{}). La connaissance du contexte de la définition permet
  de les distinguer.
\item
  une fonction \texttt{bar} contenant une variable \texttt{foo}
  est appelée plusieurs fois séquentiellement dans le programme. La
  variable \texttt{foo} dépend alors de l'appel à la fonction
  \texttt{bar}.
\item
  la variable \texttt{foo} est définie dans des threads distincts.
  Dans ce cas, les deux instances de la variable peuvent exister
  simultanément. L'instance dépend du thread dans lequel elle est
  définie.
\item
  la variable \texttt{foo} est définie dans une fonction récursive
  \texttt{bar}. Dans ce cas, une nouvelle instance de \texttt{foo}
  est définie à chaque appel récursif de \texttt{bar}.
\end{itemize}

On peut répartir ces cas en deux catégories :

\begin{itemize}
\item
  les deux variables sont lexicalement distinctes, elles sont issues de
  deux définitions distinctes dans le code source. Dans ce cas, on peut
  différencier les variables en ajoutant des informations
  supplémentaires à leurs noms, pour identifier les contextes de
  déclaration de manière unique (fichier, fonction, bloc\ldots{}).
\item
  les deux variables sont lexicalement identiques, mais elles sont
  redéfinies. Ce cas est plus complexe : les appels de fonctions et les
  créations de threads peuvent être dynamiques et dépendre de l'exécution.
  Ils sont donc difficiles à identifier de manière statique, et il en va
  de même pour les variables locales associées.
\end{itemize}

\subsection{Portée des propositions atomiques}

Dans une formule LTL, une proposition atomique est intrinsèquement globale :
pour tout état \(s\) du programme, il peut être nécessaire de déterminer si
\(s\) est un modèle de la proposition atomique. Cela est généralement déterminé
selon la valeur prise par une expression booléenne dans l'état \(s\).
L'expression est évaluée en utilisant la valeur des variables dans l'état \(s\).

Cependant, si cette expression dépend de variables locales, elle ne peut être
évaluée que si toutes ces variables locales existent dans l'état \(s\). Une
proposition atomique n'est donc correctement définie que dans le domaine où sont
définies les variables locales dont elle dépend.

Pour obtenir une définition correcte d'une proposition atomique dépendant de
variables locales, il est donc nécessaire de prolonger sa définition à
l'ensemble des états.

\subsection{Désignation des positions d'un programme}

Tout comme pour les variables, il est nécessaire de désigner de manière unique
une position du code pour l'utiliser dans une spécification.

Il faut donc concevoir un moyen de désigner une position dans le code qui soit
à la fois pratique pour l'utilisateur et robuste par rapport aux modifications du
code.
Il est aussi nécessaire de tenir compte de la précision recherchée. Sur ce
point, nous nous limiterons à désigner une instruction. En effet, si l'on veut
désigner une expression au sein d'une instruction, il est toujours possible de
la transformer en une instruction indépendante.

Plusieurs options peuvent être envisagées pour résoudre ce problème :

\begin{itemize}
\item
  indiquer la position à l'aide du numéro de la ligne et du nom du fichier ;
\item
  placer des marqueurs dans le code ;
\end{itemize}

La première méthode souffre d'un inconvénient majeur : la spécification doit
être corrigée dès qu'une ligne est ajoutée ou enlevée dans le code. Nous avons
donc retenu la seconde option. Elle impose cependant de rajouter une
instrumentation dans le code, ce que nous désirons minimiser.

Une fois la manière de désigner une position établie, il faut déterminer comment
l'utiliser dans la spécification. Nous dirons par la suite qu'un programme a
atteint une position dans le code lorsqu’un des pointeurs d'instruction du
programme pointe sur l'instruction désignée par la position.

Il est cependant nécessaire de définir comment et dans quelle mesure une
position peut intervenir dans une proposition atomique. Nous avons
considéré les options suivantes :

\begin{enumerate}
\def\labelenumi{\arabic{enumi})}
\item
  une proposition atomique désigne une position ou une condition sur les
  variables du programme, de manière exclusive. On fait le lien entre les
  positions et les valeurs des variables du programme à l'aide des opérateurs de
  la logique des prédicats. Si \texttt{pos1} et \texttt{pos2} désignent
  deux positions du programme, alors on spécifie que la variable \texttt{x}
  est non nulle entre \texttt{pos1} et \texttt{pos2} par \(G
  (\{\text{pos1}\}\implies \{x \neq 0\} U \{\text{pos1}\})\).
\item
  une proposition atomique est constituée d'une condition sur les variables du
  programme et d'une condition sur les positions. En reprenant l'exemple
  précédent, on écrirait alors \(G \lnot p\), avec \(p\) la proposition atomique
  qui est vérifiée par tous les états ayant une position entre \texttt{pos1}
  et \texttt{pos2} et où \texttt{x} est nulle.
\item
  les positions viennent contraindre les opérateurs temporels. On pourrait
  envisager une approche semblable à la logique MTL\cite{mtl_definition}, où les
  opérateurs temporels sont restreints à des intervalles de temps, mais en
  appliquant la restriction à des positions du programme. On écrirait alors
  \(G_{[pos1, pos2]} \lnot \{ x \neq 0 \}\).
\end{enumerate}

Nous avons décidé de ne pas explorer l'option 3. En effet, elle demande de
modifier le comportement standard des opérateurs LTL et nous avons préféré
limiter nos modifications aux propositions atomiques seulement.

\section{Syntaxe et sémantique de la spécification proposée}

\subsection{Syntaxe}

Nous allons tout d'abord présenter la grammaire de la spécification que
nous avons finalement retenue, avant de justifier nos choix.

Notre spécification est composée de deux parties. D'une part, une formule LTL
classique. D'autre part, la définition des propositions atomiques utilisées
par la formule LTL.

La définition d'une proposition atomique est composée des éléments suivants :

\begin{itemize}
\item
  \textbf{un nom} : il permet de référer facilement à la proposition atomique.
\item
  \textbf{une zone de validité} : elle délimite un bloc d'instructions dans
  le code (au sens du langage C). Une zone de validité est délimitée par deux
  labels (placés dans le code par l'ingénieur réalisant la spécification).
  L'entrée et la sortie d'une zone de validité doivent être dans le même
  contexte et tout pointeur d'exécution atteignant le label de début doit
  atteindre le label de fin avant de sortir du contexte (i.e. un branchement ne
  doit pas permettre de sortir d'une zone de validité en évitant l'instruction
  portant le label de fin).
\item
  \textbf{une fonction d'évaluation} : il s'agit d'une fonction booléenne pure,
  écrite en C. Cette fonction est utilisée pour déterminer si un état
  dans la zone de validité vérifie la proposition atomique ou non.
\item
\textbf{une liste de paramètres} : ils désignent les variables qui seront
  passées en argument de la fonction d'évaluation lorsque celle-ci est évaluée.
  Une variable locale est préfixée par le nom de la fonction dans laquelle elle
  est définie. Tous les paramètres doivent être définis dans la zone de validité
  de la proposition atomique.
\item
  \textbf{une valeur par défaut} : lorsque la proposition atomique n'est pas dans
  sa zone de validité, elle s'évalue à sa valeur par défaut. Cette valeur par
  défaut permet de prolonger la définition de la proposition atomique en dehors
  de sa zone de validité.
\end{itemize}

Le tableau \ref{tab:spe_gram} résume la grammaire de notre spécification.

\begin{table}[h]
\centering
\caption{Grammaire des propositions atomiques}
\label{tab:spe_gram}
\begin{tabular}[]{@{}rcl@{}}
\hline
<atomic-proposition>  & ::= & <proposition-id> <evaluation-function> <parameters>\\
                      &     &  <default> <validity-area>\\
<proposition-id>      & ::= & \emph{name of the proposition}\\
<evaluation-function> & ::= & \emph{C pure boolean function}\\
<parameters>          & ::= & <parameter> <parameters> \textbar{} \emph{nil}\\
<parameter>           & ::= & <global-parameter> \textbar{} <local-parameter>\\
<global-parameter>    & ::= & \emph{variable name}\\
<local-parameter>     & ::= & \emph{function name} :: \emph{variable name}\\
<default>             & ::= & \emph{boolean}\\
<validity-area>       & ::= & <label> <label>\\
<label>               & ::= & \emph{name of a C label}\\
\hline
\end{tabular}
\end{table}

La spécification est alors constituée d'une formule LTL portant sur des
propositions atomiques définies selon cette grammaire.

\subsection{Sémantique}

Les opérateurs de la logique classique et de la logique temporelle
suivent la sémantique usuelle de la logique propositionnelle et de LTL.

Les propositions atomiques s'évaluent selon la valeur des pointeurs
d'instruction du programme. On considère qu'un état \(p\) est dans la zone de
validité d'une proposition atomique lorsque, dans l'état \(p\), un pointeur
d'instruction du programme pointe sur une instruction comprise entre le label de
début de la zone de validité de la proposition et le label de fin. Cela signifie
que, quelque soit l'exécution menant à l'état \(p\), le pointeur d'exécution a
déjà atteint l'instruction désignée par le label de début, mais pas encore celle
désignée par le label de fin. Pour que la spécification soit valide, il est
nécessaire que, quelque soit l'exécution, si un pointeur d'instruction désigne
l'instruction marquée par un label de début, alors il atteigne celle marquée par
le label de fin avant de sortir du contexte courant.

On évalue alors la proposition atomique en un état \(p\) par :

\begin{itemize}
\item
  si l'état \(p\) n'est pas dans la zone de validité de la proposition
  atomique, alors la proposition atomique s'évalue à sa valeur par
  défaut.
\item
  si l'état \(p\) est dans la zone de validité de la proposition
  atomique, alors elle s'évalue à la valeur renvoyée par l'évaluation de
  sa fonction d'évaluation, avec comme paramètres la liste de variables
  de la proposition atomique, dans l'ordre.
\end{itemize}

\subsection{Exemple}

Nous avons choisi par la suite de représenter cette spécification en
JSON (JavaScript Object Notation)\cite{json}. Ce format est
relativement verbeux, mais présente l'avantage d'être facile à parser
tout en étant lisible pour l'être humain. Établir une syntaxe plus
concise pourrait représenter un développement futur.

Étant donné le code suivant, constitué de deux threads s'exécutant
simultanément (pour des raisons de concision, la fonction
\texttt{main} a été omise), nous allons spécifier le fait que la
variable \texttt{a} du premier thread n'est jamais nulle en même temps que
la variable \texttt{a} du second thread.

\noindent\begin{minipage}{.45\textwidth}
\begin{lstlisting}[language=C, frame=single, caption=Thread 1]
void* thread1(void* d) {
  int a
b_p1:
  a = 0;
  a = 1;
e_p1:
  pthread_exit(NULL);
}
\end{lstlisting}
\end{minipage}\hfill
\begin{minipage}{.45\textwidth}
\begin{lstlisting}[language=C, frame=single, caption=Thread 2]
void* thread2(void* d) {
  int a;
b_p2:
  a = 1;
  a = a - 1;
e_p2:
  pthread_exit(NULL);
}
\end{lstlisting}
\end{minipage}

Il est nécessaire de rajouter des fonctions d'évaluation pour les
propositions atomiques :

\noindent\begin{minipage}{.45\textwidth}
\begin{lstlisting}[language=C, frame=single, caption=Proposition atomique p1]
int f_ev_1(int x) {
    return x == 0;
}
\end{lstlisting}
\end{minipage}\hfill
\begin{minipage}{.45\textwidth}
\begin{lstlisting}[language=C, frame=single, caption=Proposition atomique p2]
int f_ev_2(int x) {
    return x == 0;
}
\end{lstlisting}
\end{minipage}

On obtient alors la spécification présentée dans le listing
\ref{lst:specifiation}.

\begin{figure}[h]
\begin{lstlisting}[frame=single, caption=Spécification, label=lst:specifiation]
{ "ltl": "G(! (p1 && p2))",
  "pa": [
    {"name": "p1",
      "default": false,
      "expr": "f_ev_p1",
      "span": ["b_p1", "e_p1"],
      "params": ["thread1::a"]
    },
    { "name": "p2",
      "default": false,
      "expr": "f_ev_p2",
      "span": ["b_p2", "e_p2"],
      "params": ["thread2::a"]
    }
  ]
}
\end{lstlisting}
\end{figure}

Dans cet exemple, le code ne respecte pas la spécification. L'exécution
\(t1: a = 0 \rightarrow t2: a = 1 \rightarrow t2: a = a - 1\) permet
d'obtenir un état qui ne satisfait pas la propriété.

\section{Motivation de nos choix}

Nous allons maintenant expliquer comment et dans quelle mesure ce formalisme
de spécification permet de surmonter les obstacles détaillés précédemment.

\subsection{Identification des variables}

Nous avons choisi d'identifier une variable globale par son nom et une
variable locale par le nom de la fonction dans laquelle elle est définie
et le nom de la variable. Cela n'apporte qu'une solution partielle au problème
de l'identification des variables.

\paragraph{Cas de deux variables lexicalement distinctes}
Notre solution permet de distinguer des variables lexicalement distinctes tant
qu'elles ne sont pas définies dans une même fonction. Nous avons jugé que ce cas
est suffisamment peu fréquent pour ne pas le traiter. Nous laissons donc le soin
à l'utilisateur de renommer les variables concernées.

\paragraph{Cas de plusieurs instances d'une variable lexicalement identique}
Nous ne proposons pas de solution dans ce cas de manière générale. Cependant,
il existe des solutions pour quelques cas particuliers.

L'un des cas où identifier une variable dépend d'une notion dynamique a lieu
lorsqu’une même fonction est utilisée pour créer plusieurs threads. En effet,
les threads peuvent être créés de manière dynamique, leur nombre et leur ordre
de création peuvent donc dépendre de l'exécution. Cependant, il est fréquent de
créer les threads d'un programme de manière statique, en créant toujours le même
nombre de threads dans le même ordre (dans la fonction \texttt{main} par
exemple). Dans ce cas, il est possible d'identifier un thread par un indice
correspondant à l'ordre de création des threads. On pourra alors identifier de
manière unique une variable dans un thread uniquement en rajoutant un préfixe à
son nom, qui prendra alors la forme
\texttt{{thread\_id}::{fonction}::{variable}}\footnote{\label{fn:notimplemented}
Cette fonctionnalité n'est pas implémentée dans l'outil présenté dans le
Chapitre~\ref{sec:Theme2}.}.

Un autre cas est lié à l'identification d'une variable locale à une fonction
appelée plusieurs fois dans un programme. Une instance de la variable est créée
à chaque appel. Si les appels à la fonction sont séquentiels (il
existe donc plusieurs instances de la variable dans le programme, mais une seule
existe à chaque instant), considérer que ces différentes instances correspondent
à la même variable dans la spécification est généralement le comportement
attendu.

Cependant, lors d'appels récursifs ou concurrents, plusieurs instances de la
variable existent simultanément : pour évaluer une proposition atomique, il
faut alors décider quelle instance doit être utilisée. Dans ce cas, nous ne
proposons pas de solution. Il est cependant possible de reprendre l'approche de
Divine\cite{Divine_LTL}, en l'adaptant à notre système de valeurs par
défaut\textsuperscript{\ref{fn:notimplemented}}. Nous présentons cette approche
plus en détail dans la section~\ref{sec:related_work}.

\subsection{Identification des positions dans le code}

Identifier une position dans le code sans y placer un marqueur n'est pas une
solution viable : elle est trop sensible aux modifications du code, d'autant
plus que lors de la vérification d'un programme, on s'attend à plusieurs
itérations avec des modifications mineures du code, afin de corriger les erreurs
détectées.

Nous avons donc utilisé des marqueurs. Nous avons choisi des labels, puisqu'ils
permettent de désigner une instruction du code. Des alternatives auraient été
d'utiliser des commentaires ou des macros afin de marquer les instructions. Nous
avons préféré les labels pour des raisons d'implémentation, les commentaires
n'étant pas conservés par de nombreuses bibliothèques de transformation de code,
mais aussi en raison de leur signification : un label a pour rôle de désigner
une instruction.

Cependant, la précision d'une position à l'aide d'un marqueur reste de, au
mieux une instruction. On ne peut donc pas désigner une
position plus précisément, au niveau d'une expression par exemple. Nous avons
considéré que ce cas est suffisamment peu fréquent pour qu'il soit possible de
l'ignorer. Il reste de plus possible de décomposer une instruction afin d'en
désigner ses parties.

Nous avons choisi de former des intervalles de positions, car ils représentent un
bon compromis entre une énumération explicite et des constructions plus
complexes (union ou intersections d'intervalles\ldots{}). Dans les cas où une
zone de validité plus complexe est nécessaire, il est possible de la simuler en
utilisant plusieurs propositions atomiques et les opérateurs de la logique
classique pour les combiner.

\subsection{Définition des propositions atomiques}

Afin de définir correctement les propositions atomiques dans tous les
états du programme, nous avons choisi d'utiliser des valeurs par défaut
combinées à des zones de validité.

L'intérêt des zones de validité est double. L'utilisateur peut spécifier
manuellement quand une proposition doit utiliser sa fonction d'évaluation. Cela
permet en particulier d'attendre que les variables soient correctement
initialisées. Les zones de validité permettent aussi d'utiliser des positions
du code dans la spécification, et de lier directement ces positions à une
proposition sur les valeurs des variables du programme. Il n'est donc pas
nécessaire d'utiliser deux propositions atomiques \texttt{b} et \texttt{e} pour
définir le début et la fin d'un intervalle de position, une troisième
proposition atomique \texttt{p} pour spécifier une propriété sur les valeurs des
variables et enfin, d'utiliser une formule telle que \(G (b \implies (p U e))\)
pour spécifier que \texttt{p} doit être valide entre \texttt{b} et \texttt{e}.
\(G p\) est suffisant, la zone de validité étant incluse dans la proposition
atomique \texttt{p}. On obtient ainsi une formule LTL plus simple.

Nous avons choisi d'utiliser une valeur par défaut que l'utilisateur peut fixer,
plutôt qu'imposer une valeur par défaut. Il serait aussi possible d'imposer la
valeur par défaut (à \emph{faux} par exemple), on pourrait alors
reproduire les propositions atomiques ayant \emph{vrai} comme valeur par
défaut en utilisant une double négation. Par exemple, pour une proposition
atomique \(p\) ayant \(vrai\) comme valeur par défaut et \(f\) comme fonction
d'évaluation, et pour la formule \(G p\), on pourrait utiliser de manière
équivalente la proposition atomique \(q\), avec \(faux\) comme valeur par défaut
et \(\lnot f\) comme fonction d'évaluation. La formule LTL serait alors \(G
\lnot q\).

Cependant, il nous a semblé plus pratique et confortable de permettre à
l'utilisateur de fixer cette valeur de manière cohérente avec
l'utilisation de la proposition atomique. Ainsi, dans le
cas d'une propriété de sûreté \(G p\), il est plus naturel de fixer la
valeur par défaut de \(p\) à \emph{vrai} plutôt que considérer une
double négation.

\section{Expressivité de la spécification proposée}

Notre spécification est une restriction de \ac{LTL} plus faible que celle utilisée
par la plupart des model-checkers logiciels. Exprimer une formule \ac{LTL} ne
portant que sur des variables globales est donc très simple. Il suffit de
désigner le programme complet comme zone de validité pour toutes les
propositions atomiques. Le reste de la spécification est inchangé.

On vérifie ainsi que notre formalisme de spécification englobe bien les formules
\ac{LTL} classiquement utilisées par les model-checkers.

Notre spécification permet aussi de spécifier l'équivalent d'une assertion. En
effet, une assertion signifie : ``si un pointeur d'instruction du
programme pointe sur l'assertion et que l'état ne vérifie pas la
condition exprimée, alors il y a une erreur''.

On peut donc représenter cela par une propriété de sûreté. On exprime
une assertion par la formule \(G p\), avec \(p\) une proposition
atomique telle que :

\begin{itemize}
\item
  sa zone de validité correspond à l'emplacement de l'assertion dans le
  programme ;
\item
  sa fonction d'évaluation est la condition utilisée dans l'assertion,
  elle s'évalue donc à la valeur de cette expression dans la zone de
  validité ;
\item
  sa valeur par défaut est \emph{vrai}e, elle s'évalue donc à \emph{vrai}
  hors de la zone de validité.
\end{itemize}

Notre formalisme de spécification englobe donc bien les spécifications à l'aide
d'assertions et les formules \ac{LTL} classiquement supportée, tout en tirant
parti des forces de ces deux formalismes.

\section{Travaux similaires}\label{sec:related_work}

Des travaux portant sur la prise en charge des variables locales ont été réalisés,
autour du model-checker Divine\cite{Divine_3_0}. Ils sont présentés dans
\cite{Divine_LTL}.

Divine met en place un ensemble de macros permettant de définir les propositions
atomiques utilisées dans une formule LTL. Deux macros, \texttt{ap(proposition)}
et \texttt{ap\_set(proposition, value)} permettent respectivement de modifier
directement la valeur d'une proposition atomique passée en argument. La macro \texttt{ap}
met la proposition atomique à \emph{vrai} uniquement pendant l'exécution de
la macro, tandis que \texttt{ap\_set} change la valeur d'une proposition
atomique de manière durable, jusqu'à une prochaine modification.

Divine permet aussi de lier une proposition à une fonction d'évaluation à l'aide
de deux autres macros, \texttt{ap\_global(proposition, fonction, N, arg1, \dots,
  argN)} et\\
\texttt{ap\_local(proposition, fonction, N, arg1, \dots, argN)}.

La macro \texttt{ap\_global} doit être appelée dans l'espace global. Elle permet
de lier une proposition atomique à une fonction booléenne dont tous les
paramètres sont des variables globales.

La macro \texttt{ap\_local} lie une proposition atomique à une fonction dont
certains des paramètres peuvent être locaux. Les paramètres sont identifiés
dans le contexte d'appel selon la norme du C, et peuvent ensuite être masqués.
Le lien entre la fonction et la proposition atomique dure jusqu'à la sortie du
contexte (où les paramètres locaux sont détruits). Ensuite, la proposition
atomique prend la valeur par défaut \emph{faux}.

\paragraph{Exemple}
Reprenons l'exemple d'une data-race (Listing \ref{lst:data-race_t1}). Les
listings~\ref{lst:data-race_divine1} et \ref{lst:data-race_divine2} présentent
la spécification de l'absence de data-race dans le formalisme de Divine. La
formule \ac{LTL} correspondante est \(G(\lnot (P1 \land P2)))\).

\noindent\begin{minipage}{.45\textwidth}
  \begin{lstlisting}[language=C, frame=single, caption=Thread 1,
    label=lst:data-race_divine1]
void* thread1(void* d) {
  ...
  ap_set(P1, 1);
  p += 1;
  ap_set(P1, 0);
  ...
}
\end{lstlisting}
\end{minipage}\hfill
\begin{minipage}{.45\textwidth}
\begin{lstlisting}[language=C, frame=single, caption=Thread 2,
    label=lst:data-race_divine2]
void* thread2(void* d) {
  ...
  ap_set(P2, 1);
  p += 1;
  ap_set(P2, 0);
  ...
}
\end{lstlisting}
\end{minipage}

\paragraph{Exemple}
Nous allons illustrer l'utilisation de \texttt{ap\_local} en spécifiant un
programme pour éviter une division par zéro. On considère pour cela la formule
\ac{LTL} \(G P\), avec \(P\) la proposition atomique indiquant que la variable
locale \(p\), dans le listing~\ref{lst:zdiv_divine} est non nulle. \(P\) est
définie à l'aide de la macro \texttt{ap\_local}.

\begin{figure}
\noindent\begin{minipage}[t]{.45\textwidth}
  \begin{lstlisting}[language=C, frame=single, caption=\texttt{ap\_local},
    label=lst:zdiv_divine]
int non_nul(int p) {
  return p != 0;
}
void* thread1(void* d) {
  int p;
  ap_local(P, non_nul, p);
  ...
  c = 1/p;
  ...
}
\end{lstlisting}
\end{minipage}\hfill
\begin{minipage}[t]{.45\textwidth}
\begin{lstlisting}[frame=single, language=C, caption=Fonction récursive,
  label=lst:divine-rec]
int p(int a) {
  return a == 8;
}
int count(int a) {
  ap_local(P, p, a);
  if (a < 10)
    return count(a + 1);
  return a;
}
int main() {
    count(6);
    return 0;
}
\end{lstlisting}
\end{minipage}
\end{figure}

L'usage de macro permet à Divine de mettre en place une spécification concise et
intuitive. La désignation des paramètres à passer aux fonctions d'évaluation, en
particulier, est à la fois moins verbeuse et plus précise --- elle permet de
designer une variable dont le nom est réutilisé dans la même fonction.

Divine est aussi confronté au problème posé par de multiples instances d'une
variable lexicalement identique. Divine choisi d'évaluer une proposition
atomique à vrai dès qu'il existe un contexte dans lequel elle s'évalue à vrai.
On considère donc la disjonction des valeurs prise par la proposition dans
l'ensemble des contextes actifs. Cette solution est justifiée par le fait qu'on
est généralement intéressé par l'instant où une proposition atomique ne prend
pas sa valeur par défaut (imposée à \emph{faux} par Divine).

\paragraph{Exemple}
Le code du listing~\ref{lst:divine-rec} présente une fonction récursive. Une
nouvelle instance de la variable \texttt{a} est donc déclarée à chaque appel
récursif, et les différentes instances existent simultanément.

Supposons que l'on veuille vérifier que la variable \texttt{a} ne vaut jamais 8
dans la fonction \texttt{count}. On cherche donc a vérifier formule \(G \lnot
P\). \(P\) est définie de sorte à être vrai si et seulement si \texttt{a} vaut
8. En déroulant les trois premiers appels, on obtient trois contextes, où
\texttt{a} vaut respectivement 6, 7 et 8. Pour \(a = 6\) et \(a = 7\), \(P\)
s'évalue à \(faux\). Pour \(a = 8\), \(P\) s'évalue à \(vrai\). On retient donc
cette valeur pour la proposition atomique. La formule est alors invalide, ce qui
permet de trouver une erreur.

Cette solution peut être adaptée dans le cas de notre formalisme de
spécification, en prenant la conjonction des valeurs si la valeur par défaut est
\emph{faux}, et la conjonction sinon.

Cependant, Divine ne propose pas l'équivalent d'un intervalle de validité.
Limiter la vérification d'une propriété à un intervalle de ligne dans le code
demande donc de définir des propositions atomiques supplémentaires, dont la
valeur est gérée par \texttt{ap} et \texttt{ap\_set}, et d'utiliser les
opérateurs \ac{LTL}. Cette approche produit des formules plus complexes, et donc
plus difficiles à vérifier.
             % Premier thème (Doctorat) ou "Détails de la Solution" (Maîtrise).
\chapter{Exprimer la spécification par des assertion à l'aide d'une
instrumentation}\label{sec:Theme2}

La majorité des outils de model-checking pour des programmes concurrents
en C ne permettent de vérifier que des programmes spécifié par des
assertions.

Afin de vérifier un programme spécifié à l'aide du formalisme présenté
dans la partie précédente, nous avons mis en place une instrumentation
du code du système. Cette dernière permet de construire un code spécifié
par des assertions dont la validité correspond au respect de la
spécification par le code d'origine.

Nous avons choisi de créer une transformation de sources à sources
plutôt que de travailler sur la représentation interne d'un
model-checker afin d'obtenir un outil compatible avec plusieurs backend,
ce qui permet de les comparer et de sélectionner le plus adapté pour un
problème spécifique.

Cela simplifie aussi les tâches d'implémentation, en nous évitant de
composer avec le code d'un système préexistant. Ce choix a aussi des
inconvénients. En particulier, une instrumentation représente une perte
de précision et de performance par rapport à un traitement direct : des
variables et des instructions sont ajoutées dans le système et vont être
traitée comme n'importe qu'elle instruction par les model-checker en
backend, alors qu'elles suivent une logique qui pourrait être utilisée
pour les traiter plus efficacement.

La transformation de source à source que nous proposons s'inspire
fortement de celle mise en place dans \cite{morse_ltl}. Elle se base
sur la composition d'un automate de Büchi représentant la négation de la
propriété LTL à prouver avec le programme à valider. Un model-checker
doit ensuite être utilisé pour explorer le produit système x automate.
La composition du système avec un automate de Büchi est une technique
classique pour le model-checking de propriétés
LTL\cite{25_years_of_model_checking}. La particularité de notre
approche est de construire ce produit au niveau du code source du
système, et non pas au niveau de la représentation interne d'un
model-checker (où le système et l'automate sont vue comme des système de
transitions).



\section{Construction de l'automate de Büchi}

Nous reprenons une technique classique de vérification des propriétés
LTL, utilisée en particulier par SPIN : construire un automate de Büchi
représentant la négation de la propriété LTL et le composer avec le
système. Une exécution acceptée par le système résultant est un
contre-exemple à la spécification.

Mais alors que de manière générale, le programme à vérifier est
représenté par un système de transitions avant d'être composé avec
l'automate de Büchi, nous allons ici représenter l'automate de Büchi en
C avant de le composer avec le programme.

Des algorithmes efficaces et éprouvés permettent de construire
l'automate de Büchi représentant une formule LTL\cite{ltl2ba}. Nous
allons donc nous concentrer sur l'implémentation d'un automate en C.

Vouloir vérifier une propriété LTL à l'aide d'assertion nous a aussi
imposé de travailler sur des traces finies. Nous allons voir dans un
premier temps comment interpréter une formule LTL dans le cas d'une
trace finie, avant d'expliquer notre construction en C de l'automate de
Büchi associé.

\subsection{Traces finies et logique à 4 valeurs de vérité}

Nous voulons instrumenter un système de manière à produire une
spécification à l'aide d'assertion qui corresponde à une propriété LTL.
Ce sont dont des assertions qui vont nous permettre de communiquer une
erreur au model-checker, en mettant fin à l'exploration lorsqu'elles
sont violées. Cependant, puisqu'une assertion met fin à l'exploration,
le fait qu'elle soit violée ou non ne dépend que du préfixe de la trace
qui a été exploré avant l'assertion. Il n'est donc pas possibles de
traiter des traces infinies : pour différencier une trace valide d'une
trace invalide, il nous faut lever une assertion, ce qui doit être fait
en temps fini.

Nous allons donc nous concentrer uniquement sur des traces finies.
Cependant, les propriétés LTL ne sont pas définies sur les traces
finies. Plusieurs variantes de LTL existent pour répondre à ce problème
(LTL3, FLTL, RV-LTL, infinite extension\ldots{}), aucune ne faisant
consensus.

Nous avons choisi l'approche du \emph{stuttering} : l'état final de la
trace finie est répété à l'infini afin d'étendre la trace de manière
infinie. Cette approche correspond bien avec un programme qui, lorsqu'il
se termine, n'a pas de raison de voir son état évoluer. Elle a de plus
l'avantage de ne pas modifier les opérateurs LTL, ce qui nous permet
d'utiliser les techniques de vérification pour LTL sans modifications.
Une trace finie est un modèle d'une propriété LTL dans la sémantique
avec \emph{stuttering} si et seulement si son extension infinie est un
modèle.

\textbf{TODO} Exemple

Il faut cependant noter que les différentes variantes de LTL pour des
traces finies diffèrent principalement par la prise en charge de
l'opérateur \emph{next}. Dans le cadre du software model-checking, le
sens donné à celui-ci est très variable et il est généralement
déconseillé de l'utiliser.

Il est possible de raffiner d'avantage les résultats en faisant la
distinction entre deux types de traces finies :

\begin{itemize}
\item
  le préfixe fini est suffisant pour déterminer si la trace est un
  modèle de la formule ou non, quelque soit l'extension
  (\emph{stuttering} ou autre)
\item
  le préfixe fini n'est pas suffisant, le résultat dépend de l'extension
\end{itemize}

Nous avons de plus utilisé une logique à 4 valeurs de vérités, comme
décrite dans \cite{morse_ltl}.

Les valeurs sont :

\begin{itemize}
\item
  True : utilisée lorsque, étant donnée une trace finie \texttt{t},
  toutes les traces ayant \texttt{t} pour préfixe sont acceptées par
  l'automate.
\item
  Maybe true : utilisée lorsque, étant donnée une trace finie
  \texttt{t}, l'extension infinie de \texttt{t} en répétant son
  dernier état est acceptée par l'automate.
\item
  Maybe false : utilisée lorsque étant donnée une trace finie
  \texttt{t}, l'extension infinie de \texttt{t} en répétant son
  dernier état est rejeté par l'automate mais que au moins une trace
  ayant \texttt{t} pour préfixe est acceptée.
\item
  False : utilisée lorsque, étant donnée une trace finie \texttt{t},
  toutes les traces ayant \texttt{t} pour préfixe sont rejetées par
  l'automate.
\end{itemize}

Dans les cas d'une conclusion \emph{Maybe true} ou \emph{Maybe false},
la connaissance supplémentaire du fait que le programme ait terminé ou
non son exécution (i.e on a vérifié l'exécution complète ou la
profondeur d'exploration a été bornée) peut permettre de préciser le
résultat.

\subsection{Implémentation de l'automate}

Nous allons construire une représentation en C de l'automate de Büchi
correspondant à la négation de la formule LTL de la spécification. Nous
avons suivis pour cela l'approche décrite dans \cite{morse_ltl}.

L'automate est implémenté par une fonction C, représentant sa fonction
de transition. Elle permet de déterminer l'évolution de son état selon
l'état courant et les valeurs des propositions atomiques.

Nous utilisons les variables et fonctions suivantes pour construire
l'automate :

\begin{itemize}
\item
  une variable globale, \texttt{ltl2ba\_state\_var}, contient l'état
  courant de l'automate sous la forme d'un entier (les états sont
  numérotés arbitrairement)
\item
  des variables globales, \texttt{ltl2ba\_atomic\_{name}} (où name est
  l'identifiant d'une proposition atomique), contiennent la valeur
  courante de chaque proposition atomique. Ces valeurs sont maintenues à
  jour pendant le déroulement du programme. Elles sont mises à jour par
  des instructions ajoutées dans le code du programme que nous
  détaillerons dans la partie sur l'instrumentation.
\item
  une fonction \texttt{\_ltl2ba\_transition} représente la fonction de
  transition de l'automate et fait évoluer l'état stocké dans
  \texttt{ltl2ba\_state\_var}. Étant donné l'état courant de
  l'automate, elle choisit de manière non-déterministe une transition
  valide partant de cet état et met à jour l'état courant. Si aucune
  transition n'est valable, le mot est rejeté (et l'exploration courante
  est stoppée). Si l'état courant est un état puits acceptant, le mot
  est accepté et une erreur est immédiatement remontée (on rappelle que
  l'automate représente la négation de la propriété à vérifier, une
  trace acceptée par l'automate est donc un contre-exemple à la
  propriété).

  Cette fonction permet de d'explorer l'automate de manière
  non-déterministe. Une valeur non-déterministe est produite l'aide de
  la fonction intrinsèque au model-checker, \texttt{nondet\_int()}. La
  transition à emprunter est désignée à l'aide de cette valeur. Le
  model-checker va alors explorer l'ensemble des évolutions possibles de
  l'automate. Les évolutions invalides sont immédiatement interrompues à
  l'aide d'assomptions.
\item
  une fonction \texttt{\_ltl2ba\_result} permet de déterminer le
  résultat d'une exploration. Elle est appelée après la dernière
  instruction du \texttt{main}. Elle se base sur l'état courant de
  l'automate et la valeur des propositions atomiques pour déterminer la
  valeur de vérité associée à cette exécution. Si une erreur est
  possible, elle est remontée au model-checker à l'aide d'une assertion.
  Les valeurs de vérités sont déterminées en utilisant des données
  pré-calculées par une analyse d'accessibilité dans l'automate de Büchi
  \cite{morse_ltl}.
\end{itemize}

Un model-checker n'a pas la possibilité de trancher entre les différents
résultats possibles de l'analyse à partir de cet automate. Nous
signalons donc toute les erreurs (\emph{Maybe true}, \emph{Maybe false},
\emph{Surely false}) par une assertion dont le message indique le type
d'erreur détecté. Il est donc possible qu'une erreur en maque une autre
: le model-checker arrête son exploration lors de la première violation
d'assertion qu'il détecte. Il peut alors s'agir d'un résultat moins fort
que celui qu'il aurait trouvé dans un entrelacement futur. Il peut donc
être nécessaire de relancer plusieurs fois l'outil en désactivant les
assertions donnant les résultats les plus faibles pour obtenir le
résultat final.

Notre implémentation de l'automate de Büchi en C diffère de celle de
\cite{morse_ltl} par un point d'importance : dans
\cite{morse_ltl}, l'automate est implémenté dans un thread
suplémentaire, utilisé comme un thread observateur. La fonction de
transition de l'automate est placée dans une boucle, synchronisée avec
le reste du code à l'aide d'une commande intrinsèque du model-checker.
Cette commande à été implémentée pour l'occasion. Dans notre cas, la
fonction de transition est représentée par une fonction dans le code,
appelée quand nécessaire par l'instrumentation. Cela nous permet de ne
pas se baser sur une commande intrinsèque au model-checker pour
synchroniser l'automate au code et ainsi, de supporter plus facilement
différents model-checkers en back-end.

L'automate pour la propriété LTL \(G p \implies F q\) est implémenté par
le code suivant.

\begin{lstlisting}[language=C]
_Bool _ltl2ba_atomic_p = 0;
_Bool _ltl2ba_atomic_q = 0;

_ltl2ba_state _ltl2ba_state_var = 0;

void _ltl2ba_transition() {
    int choice = nondet_uint();
    switch (_ltl2ba_state_var) {
    case 0:
        if (choice == 0) {
            assume(!_ltl2ba_atomic_p);
            _ltl2ba_state_var = 0;
        } else if (choice == 1) {
            assume(1);
            _ltl2ba_state_var = 1;
        } else if (choice == 2) {
            assume(_ltl2ba_atomic_q);
            _ltl2ba_state_var = 0;
        } else {
            assume(0);
        }
        break;
    case 1:
        if (choice == 0) {
            assume(1);
            _ltl2ba_state_var = 1;
        } else if (choice == 1) {
            assume(_ltl2ba_atomic_q);
            _ltl2ba_state_var = 0;
        } else {
            assume(0);
        }
        break;
    }
}

_Bool _ltl2ba_surely_accept[2] = {0, 0};
_Bool _ltl2ba_surely_reject[2] = {0, 0};
_Bool _ltl2ba_stutter_accept[8] = {1,0, 0,0, 1,1, 1,1,};

unsigned int _ltl2ba_sym_to_id() {
    unsigned int id = 0;

    id |= (_ltl2ba_atomic_p << 0);
    id |= (_ltl2ba_atomic_q << 1);
    return id;
};

void _ltl2ba_result() {
    _Bool reject_sure = _ltl2ba_surely_reject[_ltl2ba_state_var];
    assume(!reject_sure);

    _Bool accept_sure = _ltl2ba_surely_accept[_ltl2ba_state_var];
    assert(!accept_sure, "ERROR SURE");

    unsigned int id = _ltl2ba_sym_to_id();
    _Bool accept_stutter = _ltl2ba_stutter_accept[id * 2 + _ltl2ba_state_var];
    assert(!accept_stutter, "ERROR MAYBE");
    assert(accept_stutter, "VALID MAYBE");
}
\end{lstlisting}

\section{Instrumentation du code}

Le code est instrumenté afin de construire le produit entre le programme
et l'automate. L'instrumentation va maintenir à jour la valeur des
propositions atomiques tout au long de l'exécution du programme et
appeler la fonction de transition afin de faire évoluer l'automate.

Afin de minimiser l'instrumentation, la valeur d'une propositions
atomique est mise à jour uniquement lorsque sa valeur est susceptible de
changer. La fonction de transition de l'automate est appelée uniquement
à ces occasions. L'opérateur LTL \emph{Next} a alors pour signification
``lors du prochain appel à la fonction de transition'', c'est à dire la
prochaine fois qu'une proposition atomique est susceptible d'être
modifiée.

L'instrumentation est réalisée à l'aide de CIL (C Intermediate Langage).
CIL permet de simplifier le code en, entre autre, décomposant les
instructions complexes en instructions simples et sans effets de bord.
Nous supposons dans notre instrumentation que les instructions produites
par CIL sont atomiques.

\subsection{Frontière des zones de validité}

Les premiers points nécessitant une instrumentation sont les frontières
des zones de validité des propositions atomiques. Lorsqu'un pointeur
d'exécution atteint l'instruction portant le label d'entrée
(respectivement de sortie) d'une zone de validité, la proposition
atomique doit prendre la valeur calculée par sa fonction d'évaluation
(respectivement sa valeur par défaut).

On va donc réaliser l'instrumentation suivante pour un label d'entrée.
Ici, la proposition atomique \texttt{p} débute au label
\texttt{lbegin} et se termine au label \texttt{lend}.

\begin{lstlisting}[language=C]
__atomic_begin();
ltl2ba_atomic_p = fp(..);
_ltl2ba_transition();
__atomic_end();
lbegin: ....;
\end{lstlisting}

Et pour la sortie de la zone :

\begin{lstlisting}[language=C]
__atomic_begin();
ltl2ba_atomic_p = {default_val};
_ltl2ba_transition();
__atomic_end();
lend: ....;
\end{lstlisting}

Toutes les instructions sont regroupées dans un bloc atomique (ouvert
par l'instruction \texttt{\_\_atomic\_begin()} et fermé par
l'instruction \texttt{atomic\_end()}) afin de ne pas générer
d'entrelacements supplémentaires dans le code et créer des chemins
d'exécutions pouvant mener à une conclusion erronée : toutes les
instructions de l'instrumentation doivent être réalisées de manière
atomiques (sans changement de contexte possible) pour garantir un état
cohérent de l'automate. Si plusieurs entrées et sorties de zone ont
lieux au même label, les instructions d'instrumentations sont toutes
réunies dans le même bloc atomique. La fonction de transition n'est
appelée qu'une seule fois, après l'actualisation de toutes les
propositions atomiques.

Cependant, cette instrumentation des frontières des zones de validité
provoque tout de même une perte de précision. L'entrée dans une zone de
validité correspond à l'instant où le pointeur d'exécution pointe sur
l'instruction portant le label d'entrée, i.e le moment ou cette
instruction est susceptible d'être exécutée. Le pointeur d'exécution est
mis à jour immédiatement après l'exécution de l'instruction précédent
l'entrée dans la zone, avant tout changement de contexte. Par
conséquent, l'instrumentation devrait avoir lieu immédiatement après
cette instruction, sans permettre un changement de contexte, mais ce
n'est pas le cas dans notre instrumentation.

Illustrons cela sur un exemple.

\begin{lstlisting}[language=C]
instr1;
begin: instr2;
end: instr3;
\end{lstlisting}

Dans ce fragment de code, dès que l'instruction \texttt{instr1} est
exécutée, le pointeur d'exécution pointe sur l'instruction
\texttt{instr2}. On est donc immédiatement dans la zone de validité
délimitée par les labels \texttt{begin} et \texttt{end}, et il
n'est pas possible d'effectuer un changement de contexte après
\texttt{instr1} mais avant d'entrer dans la zone de validité.

Cependant, après instrumentation, on obtient :

\begin{lstlisting}[language=C]
instr1;
{intrumentation;}
begin: instr2;
{instrumentation;}
end: instr3;
\end{lstlisting}

L'entrée dans la zone de validité n'est réalisée qu'une fois que
l'instrumentation a été exécutée et que la valeur des propositions
atomiques a donc été recalculée. Il est donc maintenant possible de
réaliser un changement de contexte après \texttt{instr1}, mais avant
d'entrer dans la zone de validité : un nouveau chemin d'exécution a été
créé.

Pour éviter cette perte de précision, il serait nécessaire de rendre
atomique l'instruction \texttt{instr1} et le code d'instrumentation.
Ceci est extrêmement compliqué dans le cadre d'une instrumentation du
code. En effet, l'instruction qui précède un label n'est pas facilement
identifiable et peut varie selon l'exécution et selon les branchements
du programme. De plus, sans dépliage du code, une même instruction peut
précéder l'entrée ou la sortie d'une zone certaines fois mais pas
systématiquement (instruction en fin de boucle ou de fonction). Dans le
cadre d'une instrumentation du code, nous n'avons aucune solution face à
cette perte de précision. Une implémentation intégrée à un model-checker
serait nécessaire.

On peut cependant noter que les assertions utilisées de manière
classique permettent elles aussi ce changement de contexte immédiatement
avant elles, ce qui n'a généralement pas de conséquences.

\subsection{Variables locales}

Passons à la seconde raison pouvant amener à modifier la valeur d'une
proposition atomique. Il s'agit de la modification d'une variable dont
dépend la proposition atomique, dans sa zone de validité (en dehors de
la zone de validité, la proposition atomique s'évalue toujours à sa
valeur par défaut, la modification d'une variable n'a donc pas
d'impact). Nous allons tout d'abord nous concentrer uniquement sur le
cas des variables locales à une fonction.

Si une variable locale est modifiée dans la zone de validité d'une
proposition atomique et que la valeur de la proposition atomique dépend
de la variable, il est toujours nécessaire d'instrumenter l'affectation.
Instrumenter ces appels est aussi suffisant (car nous ne considérons pas
les accès indirects aux variables). La variable étant locale, elle ne
peut s'échapper hors de son contexte : elle ne peut donc pas être
modifiée par une instruction hors de la zone de validité alors qu'un
pointeur d'instruction est dans la zone de validité.

On peut alors ajouter une instrumentation autour de l'affectation. Après
la modification de la variable, les propositions atomiques qui en
dépendent sont évaluées à nouveau. On tente ensuite d'effectuer une
transition dans l'automate de Büchi.

\begin{lstlisting}[language=C]
__atomic_begin();
v = ....;
ltl2ba_atomic_p = fp(v, ...);
_ltl2ba_transition();
__atomic_end();
\end{lstlisting}

L'affectation de la variable à sa nouvelle valeur et l'instrumentation
sont placés dans un même bloc atomique, afin de la modification de
l'état soit répercutée sans changement de contexte possible.

Contrairement à l'instrumentation à l'entrée et la sortie des zones de
validité, ici, aucun nouveau chemin d'exécution n'est créé : une
affectation est un statement, que nous considérons comme atomique. Il
est donc possible de l'inclure dans un bloc atomique pour le
model-checker.

\subsection{Variables globales}

Passons maintenant au cas où la proposition atomique ne dépend que de
variables globales.

Tout comme une variable locale, si une variable globale est modifiée
dans la zone de validité d'une proposition atomique et que cette
proposition atomique dépend de la variable globale, il est nécessaire
d'actualiser la valeur de la proposition atomique et d'effectuer une
transition dans l'automate. Mais une variable globale peut aussi
affecter la valeur d'une proposition atomique lors d'une affectation
hors de la zone de validité de cette proposition : si un pointeur
d'exécution du programme est dans la zone de validité et qu'une variable
globale dont dépend la proposition est modifiée dans un autre thread,
cette modification a un impact sur la valeur de la proposition atomique.

L'exemple suivant illustre ce problème :

\begin{lstlisting}
int a = 1;

int pred(int a) {
    return a != 0;
}

void* thread1(void* d) {
    int b = *(int *)d;
    int r;
    begin: ;
    a = b;
    r = 10 / a;
    end: ;

}

void* thread2(void* d) {
    a = 0;
}
\end{lstlisting}

On désire s'assurer que la variable \texttt{a} n'est pas nulle dans
la zone de validité allant de \texttt{begin} à \texttt{end} dans
le première thread. L'affectation \texttt{a = b;} est susceptible de
faire évoluer cette propriété, il est donc nécessaire de l'instrumenter,
tout comme dans le cas des variables locales.

L'affectation \texttt{a = 0} dans le second thread peut aussi avoir
un impact sur la propriété, mais cela dépend de l'ordre d'exécution des
instructions. Par exemple :

\begin{itemize}
\item
  si l'ordre d'exécution est \texttt{a = b; a = 0; r = 10 / a}, alors
  l'affectation \texttt{a = 0;} du second thread doit impacter la
  valeur de la proposition atomique : \texttt{a} est maintenant nulle
  dans la zone de validité.
\item
  si le second thread s'exécute entièrement avant le premier, quand
  \texttt{a = 0;} est exécuté, la proposition atomique n'est pas dans
  sa zone de validité (le pointeur d'exécution du premier thread n'a pas
  encore atteint le label \texttt{begin}). Dans ce cas, la valeur de
  la proposition atomique ne doit pas être modifiée.
\end{itemize}

Il est donc nécessaire d'instrumenter aussi les affectation à des
variables globales hors de la zone de validité d'une proposition.
Cependant, une affectation en dehors d'une zone de validité ne doit pas
non plus systématiquement donner lieux à une actualisation des
proposition atomique : il ne faut actualiser la valeur de la proposition
atomique que lorsqu'un pointeur d'exécution est dans sa zone de
validité. C'est donc une notion dynamique, qui dépend de l'exécution.

Pour n'actualiser la valeur que lorsque c'est nécessaire,
l'instrumentation prendra la forme d'une condition. Nous ajoutons une
variable globale pour chaque proposition atomique, qui indique si le
programme est dans sa zone de validité. On maintient la valeur de cette
variable à l'entrée et à la sortie de chaque zone de validité. On peut
alors tester la valeur de cette variable pour définir si il faut
actualiser ou non la proposition atomique.

La valeur de cette nouvelle variable globale doit être actualisée
l'entrée et à la sortie des zones de validités. L'instrumentation de ces
zones devient donc :

\begin{lstlisting}[language=C]
__atomic_begin();
_ltl2ba_active_p = 1;
ltl2ba_atomic_p = fp(..);
_ltl2ba_transition();
__atomic_end();
lbegin: ....;
\end{lstlisting}

Et pour la sortie d'une zone :

\begin{lstlisting}[language=C]
__atomic_begin();
_ltl2ba_active_p = false;
ltl2ba_atomic_p = {default_val};
_ltl2ba_transition();
__atomic_end();
lend: ....;
\end{lstlisting}

L'instrumentation lors de la modification d'une variable globale prend
alors la forme suivante :

\begin{enumerate}
\def\labelenumi{\arabic{enumi})}
\item
  Pour des instructions situées dans la zone de validité de la
  proposition :
\end{enumerate}

\begin{lstlisting}[language=C]
__atomic_begin();
g = ....;
ltl2ba_atomic_p = fp(g, ...);
_ltl2ba_transition();
__atomic_end();
\end{lstlisting}

\begin{enumerate}
\def\labelenumi{\arabic{enumi})}
\setcounter{enumi}{1}
\item
  Pour des instructions situées hors de la zone de validité de la
  proposition :
\end{enumerate}

\begin{lstlisting}[language=C]
__atomic_begin();
g = ....;
if (_ltl2ba_active_p) {
    ltl2ba_atomic_p = fp(g, ...);
    _ltl2ba_transition();
}
__atomic_end();
\end{lstlisting}

\subsection{Mélange de variables globales et locales}

Enfin, plaçons nous dans le cas où une proposition atomique dépend à la
fois de variables globales et de variables locales.

Un nouveau problème apparaît : lorsqu'une variable globale est modifiée
hors de la zone de validité, il peut être nécessaire de mettre à jour la
proposition atomique à l'aide de sa fonction d'évaluation, comme nous
l'avons vu précédemment. Cette fonction prend en paramètre les variables
locales dont dépend la proposition atomique. Mais ces variables locales
ne sont pas dans le contexte de l'appel ! Celui-ci est réalisé depuis un
autre thread, depuis lequel ces variables locales ne sont pas
accessibles.

L'exemple suivant illustre ce problème :

\begin{lstlisting}[language=C]
int a = 1;

int pred(int a, int b) {
    return a == b;
}

void* thread1(void* d) {
    int b = *(int *)d;
    int r;
    begin: ;
    a = b;
    r = 10 / a;
    end: ;

}

void* thread2(void* d) {
    a = 0;
}
\end{lstlisting}

Dans ce code, on veut vérifier que les variables \texttt{a} et
\texttt{b} sont bien égale durant tout l'intervalle de validité de la
proposition. Le second thread serait instrumenté de la manière suivante
:

\begin{lstlisting}[language=C]
void* thread2(void* d) {
    __atomic_begin();
    a = 0;
    if (_ltl2ba_active_p) {
        ltl2ba_atomic_p = pred(a, b); // Here, b is not in the context
        _ltl2ba_transition();
    }
    __atomic_end();
}
\end{lstlisting}

On remarque que la variable \texttt{b} n'est pas accessible lorsqu'il
est nécessaire d'appeler la fonction d'évaluation.

Cependant, on sait que lorsque cet appel a lieu, un pointeur d'exécution
du programme est dans la zone de validité de la proposition : toutes les
variables locales dont dépend la proposition sont donc dans la pile. On
peut alors y accéder de manière indirecte, en utilisant des pointeurs
globaux contenant les adresses de ces variables. Nous allons donc
compléter l'instrumentation pour maintenir des adresses de toutes les
variables utilisées comme paramètre d'une proposition atomique.

Pour chaque variable utilisée comme paramètre d'une proposition
atomique, une variable globale supplémentaire est créée. A l'entrée dans
la zone de validité d'une proposition, cette variable globale est
assignée avec l'adresse du paramètre auquel elle correspond. Il est
alors possible d'accéder à la valeur de la variable en déréférençant ce
pointeur global.

Notre exemple devient alors, instrumenté en tenant compte de ces
modifications :

\begin{lstlisting}[language=C]
void *thread1(void *d) {

int a = 1;
int _ltl2ba_active_p = 0;
int *_ltl2ba_ptr_b;
int _ltl2ba_atomic_p = {default-val};

int pred(int a, int b) {
    return a == b;
}

void* thread1(void* d) {
    int b = *(int *)d;
    int r;

    begin: ;
    __atomic_begin();
    _ltl2ba_active_p = 1;
    _ltl2ba_ptr_b = &b;
    ltl2ba_atomic_p = pred(a, *_ltl2ba_prt_b);
    _ltl2ba_transition();
    __atomic_end();

    a = b;
    r = 10 / a;

    __atomic_begin();
    _ltl2ba_active_p = 0;
    ltl2ba_atomic_p = {default_val};
    _ltl2ba_transition();
    __atomic_end();
    end: ;

}

void* thread2(void* d) {
    __atomic_begin();
    a = 0;
    if (_ltl2ba_active_p) {
        ltl2ba_atomic_p = pred(a, *_ltl2ba_ptr_b);
        _ltl2ba_transition();
    }
    __atomic_end();
}
\end{lstlisting}

\section{Implémentation}

Nous avons implémenté cette transformation dans un outil, BaProduct.

Nous générons les automates de Büchi correspondant aux formules LTL en
utilisant le logiciel \emph{LTL2BA}\cite{ltl2ba}. LTL2BA implémente
une méthode rapide afin de construire l'automate de Büchi associé à une
formule LTL, en utilisant un co-automate de Büchi alternant très faible
(\emph{very weak alternating co-Büchi automaton}) et un automate de
Büchi généralisé (*generalised Büchi automaton) comme intermédiaires.

L'instrumentation est réalisée en utilisant la bibliothèque
CIL\cite{cil}, en OCaml\cite{ocamlrefman}. CIL permet de parser
du code C et de construire un arbre de syntaxe abstraite plus concis que
celui du C mais d'un niveau plus haut que celui des langages
intermédiaires utilisés dans les procédés de compilation. La
bibliothèque ocamlgraph\cite{ocamlgraph} est utilisée pour réaliser
l'analyse d'accessibilité dans les automates.

Notre implémentation comprend :

\begin{itemize}
\item
  la version modifiée de LTL2BA. Son fonctionnement est identique à
  celui de la version d'origine, seule une option \texttt{-t} a été
  ajoutée afin de choisir le type de la sortie. Cette option peut
  prendre les valeurs \texttt{spin}, \texttt{c} ou
  \texttt{json} pour obtenir un automate en Promela, C et JSON
  respectivement.
\item
  L'utilitaire \texttt{baProduct}. Il s'agit du programme réalisant
  l'instrumentation. Il prends en entré un fichier C préprocessé et un
  fichier de spécification, et il produit un fichier instrumenté. Cet
  utilitaire a besoin de la version modifiée de LTL2BA pour fonctionner.
\item
  Un script de lancement, \texttt{baProduct.py}. Il préprocesse les
  fichiers en entrée et appelle \texttt{baProduct} avec un ensemble
  d'options classiques.
\end{itemize}

Une utilisation classique de notre implémentation ressemble donc à
l'appel suivant :

\begin{lstlisting}
~ ./baProduct.py -i test.c -s test.spec
~ esbmc --depth 100 test_instr.c
\end{lstlisting}

Le code de ces outils et une série de tests sont disponible à {[}TODO:
mettre sur github, ajouter adresse{]}
             % Second thème (Doctorat) ou "Résultats théoriques et expérimentaux" (Maîtrise).
\Chapter{CONCLUSION}\label{sec:Conclusion}

%%
%%  SYNTHESE DES TRAVAUX
%%
\section{Synthèse des travaux}

Dans ce mémoire, nous avons examiné l'état de l'art des techniques de
model-checking logiciel, en nous concentrant sur la vérification de programmes
concurrents en C, suivant la norme POSIX (pThread). Nous décrivons les
principales approches actuellement développées ainsi que les principaux projets
actifs implémentant ces approches. Nous identifions deux principales
limitations. L'explosion combinatoire reste, malgré les optimisations et les
techniques de réduction, le principal obstacle à surmonter afin de permettre au
model-checking logiciel de passer à l'échelle. De plus, les mécanismes de
spécification dans le cas de programmes concurrents sont peu développés : les
assertions sont peu adaptées à la concurrence, tandis que peu d'outils sont
capables de vérifier des propriétés \ac{LTL}, et toujours dans une forme
limitée.

Nous nous sommes alors intéressés aux limitations des formalismes de
spécifications. Nous avons mis en place un nouveau formalisme, basé sur
\ac{LTL}, et utilisant le concept de \emph{zones de validité} afin de permettre
de manipuler des positions du programme et des variables locales dans les
propriétés \ac{LTL}. Nous vérifions que ce formalisme permet de représenter à la
fois la restriction de \ac{LTL} classiquement utilisée ainsi que des assertions,
tout en permettant de dépasser leurs principales limitations.

Enfin, nous présentons une transformation de source à source d'un programme en C
permettant de vérifier un code spécifié par notre formalisme. Nous contournons
le manque d'outils supportant le model-checking de propriétés LTL en reprenant
l'approche présentée par \cite{morse_ltl} et nous l'adaptons à notre cas. Nous
construisons ainsi un code instrumenté représentant le produit entre le code
d'origine et un automate de Büchi, spécifié par des assertions. Ce code
instrumenté peut ensuite être vérifié par un model-checker, sans nécessiter un
support de LTL. Cette solution reste cependant limitée à l'exploration de traces
finies. Nous avons testé cette instrumentation sur un jeu de tests, à l'aide des
model-checkers CBMC et ESBMC, ce qui nous a permis de valider notre approche. Il
est donc possible de vérifier des programmes spécifiés dans notre formalisme à
l'aide de n'importe quel model-checker borné, d'autres techniques de
model-checking étant probablement utilisables aussi au prix de modification de
l'instrumentation ou d'une intégration de cette dernière directement dans le
model-checker. Cependant, notre approche n'est pas encore capable de passer à
l'échelle, car elle n'intègre pas de technique de réduction permettant de
limiter l'explosion combinatoire due à la concurrence.

%%
%%  LIMITATIONS
%%
% \section{Limitations de la solution proposée}\label{sec:Limitations}
% Déjà fait avant

%%
%%  AMELIORATIONS FUTURES
%%
\section{Améliorations futures}

Le formalisme de spécification que nous proposons permet d'exprimer un ensemble
de propriétés significativement plus large que celui exprimable par des
assertions. Afin de mesurer l'impact de ce formalisme, des tests plus conséquents
sur du code issus de systèmes réels seraient nécessaires. Une telle étude
permettrait aussi de déterminer si des extensions plus larges sont encore
nécessaires pour exprimer des propriétés fréquemment utilisées.

Une autre piste de travail intéressante serait d'intégrer la procédure de
vérification de notre instrumentation directement dans un outil de
model-checking. On perdrait alors les avantages d'une instrumentation, mais il
serait possible de gagner en performances et de contourner un certain nombre des
limitations actuelles de baProduct. Améliorer les performances pourrait aussi
passer par l'intégration de techniques de réduction.

Enfin, il serait possible d'étudier comment tirer parti de l'instrumentation dans
d'autres contextes de vérification. En particulier, l'instrumentation donne
accès à l'automate de Büchi. En le modifiant, il est possible de restreindre les
exécutions explorées par un model-checker et se concentrer sur des scénarios
critiques. L'instrumentation pourrait aussi être modifiée afin d'être rendue
exécutable, quitte à remplacer les choix non déterministes par des choix
aléatoires. Cela ouvre la porte à des approches de vérification de notre
spécification en utilisant des outils de test (stress testing ou fuzzing par
exemple).

\section{Remarques conclusives}

À travers ce mémoire, nous avons contribué à étendre légèrement les possibilités
de spécification pour le model-checking de programmes concurrents. Cependant,
ces progrès viennent avec un coût de vérification plus important. Ces coûts de
vérification, liés en grande partie à l'explosion combinatoire, constituent la
principale limite du model-checking actuellement. Ce problème ne semble pas
proche d'être résolu, à moins d'une percée spectaculaire dans ce domaine : on
assiste plutôt à un ensemble d'optimisations qui permettent de vérifier des
programmes de taille toujours plus grande. De manière concomitante, de nouveaux
langages de programmation émergent, axés sur la sécurité et permettant de
prouver certaines propriétés d'un programme à sa compilation. Un exemple est
Rust\cite{Rust}, qui permet d'assurer l'absence de data-race et une certaine
forme de consistance entre les threads. Le model-checking pourrait profiter des
propriétés assurées par ces langages afin de réduire davantage la taille des
modèles.         % Conclusion.
%\backmatter
\renewcommand\bibname{RÉFÉRENCES}
\bibliography{Document}
%\bibliographystyle{polymtl}  % Format de la bibliographie.
\bibliographystyle{IEEEtranSN-francais}  % Format de la bibliographie.
%
\ifthenelse{\equal{\AnnexesPresentes}{O}}{
\appendix%
\newcommand{\Annexe}[1]{\annexe{#1}\setcounter{figure}{0}\setcounter{table}{0}\setcounter{footnote}{0}}%
%%
%%  Annexes.
%%
%%  Note: Ne pas modifier la ligne ci-dessous.
\addcontentsline{toc}{compteur}{ANNEXES}
%%
%%
%%  Toutes les annexes doivent être inclues dans ce document
%%  les unes à la suite des autres.

\begin{landscape}
\Annexe{Types de propriétés supportés par différents model-checkers}
\begin{table}[ht]
\centering
\caption{Types de propriétés prise en charge par les outils présentés.}
\begin{tabular}{|l|c|c|c|c|c|c|c|c|c|}
\hline
Outils     & \multicolumn{9}{c|}{Types de propriétés vérifiées}                                                                                     \\
           & Assertions   &      LTL     & Arithmétique &     Tableaux & Pointeurs    & Data races   & Deadlock     & Cast         & Init           \\ \hline
SPIN       &              & $\checkmark$ &              &              &              &              &              &              &                \\
Pancam     & $\checkmark$ &              &              &              &              &              &              &              &                \\
Inspect    & $\checkmark$ &              &              &              &              &              &              &              &                \\
Divine     & $\checkmark$ & $\checkmark$ &              &              &              &              &              &              &                \\
CIVL       & $\checkmark$ &              & $\checkmark$ & $\checkmark$ & $\checkmark$ & $\checkmark$ & $\checkmark$ & $\checkmark$ & $\checkmark$   \\
Impara     & $\checkmark$ &              &              &              &              &              &              &              &                \\
Satabs     & $\checkmark$ &              & $\checkmark$ & $\checkmark$ & $\checkmark$ &              &              &              &                \\
Threader   & $\checkmark$ &              &              &              &              &              &              &              &                \\
CPAChecker & $\checkmark$ &              &              &              &              &              &              &              &                \\
CBMC       & $\checkmark$ &              & $\checkmark$ & $\checkmark$ & $\checkmark$ &              &              &              &                \\
ESBMC      & $\checkmark$ &              & $\checkmark$ & $\checkmark$ & $\checkmark$ & $\checkmark$ & $\checkmark$ &              &                \\
CSeq       & $\checkmark$ &              &              &              &              &              &              &              &                \\
Lazy-CSeq  & $\checkmark$ &              &              &              &              &              & $\checkmark$ &              &                \\
Mu-CSeq    & $\checkmark$ &              &              &              &              &              &              &              &                \\
UL-CSeq    & $\checkmark$ &              &              &              &              &              &              &              &                \\ \hline
\end{tabular}
\label{tab:prop_type_table}
\end{table}

\end{landscape}



\Annexe{Exemple de spécification}

Considérons le problème suivant : un système contient deux batteries. Les
batteries se vident et chacune est changée lorsqu'elle est épuisée. Tant qu'une
des batteries est présente, le système fonctionne. Cependant, si les deux
batteries sont vides en même temps, le système n'a plus d'énergie et cesse de
fonctionner.

Un tel système peut être modélisé par le code ci-dessous. Deux threads,
\texttt{battery1} et \texttt{battery2} contrôlent le rythme auquel les
batteries se vident. Le pourcentage d'énergie dans la batterie est représenté
par la variable \texttt{energy}. Celle-ci est décrémentée dans une
boucle, ce qui représente la consommation de l'énergie, avant d'être remise à
une valeur de 100, ce qui simule le remplacement. Dans ce système simpliste,
aucune synchronisation n'existe entre les threads. L'état d'erreur peut donc
évidement avoir lieu. Nous allons tout de même essayer de détecter cette erreur
à l'aide d'un model-checker logiciel.

La propriété que nous souhaitons vérifier est que, quelque soit
l'exécution, les deux batteries ne seront jamais simultanément vide : il
s'agit donc d'un problème d'exclusion mutuelle. Afin d'exécuter la
vérification, il va tout d'abord être nécessaire de spécifier cette
propriété.

\noindent\begin{minipage}{.45\textwidth}
\begin{lstlisting}[language=C, frame=single, caption=Thread 1]
void* battery1(void* d) {
  int energy = 100;
  while(1) {
    // Empty the battery
    for(; energy>0; energy--);
    // Replace the battery
    energy = 100;
  }
  pthread_exit(NULL);
}
\end{lstlisting}
\end{minipage}\hfill
\begin{minipage}{.45\textwidth}
\begin{lstlisting}[language=C, frame=single, caption=Thread 2]
void* battery2(void* d) {
  int energy = 100;
  while(1) {
    // Empty the battery
    for(; energy>0; energy--);
    // Replace the battery
    energy = 100;
  }
  pthread_exit(NULL);
}
\end{lstlisting}
\end{minipage}

Comme nous l'avons vu dans l'état de l'art, la principale limitation du
model-checking est actuellement l'explosion combinatoire. Les outils de
model-checking pour des logiciels concurrents restreignent le type de
propriétés qu'ils vérifient pour optimiser leurs performances. Ainsi, la
plupart de ces outils ne permettent de vérifier que des propriétés
exprimées à l'aide d'assertions.

Dans un premier temps, nous allons donc spécifier notre propriété par
des assertions.

Cependant, sans modification du code, des assertions ne permettent pas
de spécifier une exclusion mutuelle. Nous allons donc rajouter une
variable indicatrice, qui va signaler lorsque un pointeur d'exécution
est dans une des zones critiques (c'est à dire lorsque la batterie est
en remplacement).

On obtient alors le code suivant :

\noindent\begin{minipage}{.45\textwidth}
\begin{lstlisting}[language=C, frame=single, caption=Thread 1]
int critical = 0;
void* battery1(void* d) {
  int energy = 100;
  while(1) {
    // Empty the battery
    for(; energy>0; energy--);
    // Replace the battery
    assert(critical != 1);
    critical = 1;
    energy = 100;
    critical = 0;
  }
  pthread_exit(NULL);
}
\end{lstlisting}
\end{minipage}\hfill
\begin{minipage}{.45\textwidth}
\begin{lstlisting}[language=C, frame=single, caption=Thread 2]

void* battery2(void* d) {
  int energy = 100;
  while(1) {
    // Empty the battery
    for(; energy>0; energy--);
    // Replace the battery
    assert(critical != 1);
    critical = 1;
    energy = 100;
    critical = 0;
  }
  pthread_exit(NULL);
}
\end{lstlisting}
\end{minipage}

% Cette instrumentation souffre cependant d'un problème : les instructions
% supplémentaires créent de nouveaux chemins d'exécutions. Ces chemins
% n'existent pas dans le programme d'origine et pourraient permettre
% d'atteindre des états auparavant inaccessibles. Ce n'est heureusement
% pas le cas ici. Nous reviendrons plus tard {[}TODO: indiquer où{]} sur
% ce problème.

L'utilisation de ces variables indicatrices définit des états (dans et
hors de la zone critique pour chaque thread), qui peuvent rappeler la
structure d'un automate de Büchi associé à la propriété ``l'exclusion
mutuelle est toujours respectée''.

D'autres exemples --- tels que ``une \emph{assertion} doit être atteinte
avant une autre pour qu'une erreur soit présente'' --- nécessitent aussi
d'instrumenter le code à l'aide de variables indicatrices. De manière
similaire, ces variables définissent des états et une évolution entre
ces états qui rappelle la construction d'un automate de Büchi.

Quitte à construire cet automate de Büchi, pourquoi ne pas spécifier nos
propriétés en utilisant directement LTL ? Les propriétés LTL sont fréquemment
représentée par des automates de Büchi par les algorithmes de model-checking, on
pourrait ainsi déléguer la construction de l'automate au model-checker.
On réduit ainsi le risque d'introduire des erreurs dans la construction de cet
automate et son intégration dans le modèle, tout en profitant de la syntaxe plus
simple de LTL.

Cependant, les outils de model-checking logiciel supportant le C restreignent
généralement les propositions atomiques des formules LTL qu'ils vérifient à des
expressions pures (sans effet de bord) du C, ne faisant référence qu'à des
variables globales. On ne peut donc pas mentionner directement une variable
locale ou une position du programme dans ces spécifications.

Tentons donc de spécifier notre modèle à l'aide d'une formule LTL. Deux
options sont possibles. La première consiste à vérifier que les deux
batteries ne sont pas vide en même temps, c'est à dire que les variables
\texttt{energy} des threads \texttt{battery1} et
\texttt{battery2} ne valent pas \texttt{0} simultanément. Mais les
variables \texttt{energy} sont des variables locales : la
plupart des outils ne permettent pas de les mentionner dans une formule
LTL. On peut aussi considérer que les opérations de changement de
batterie ne doivent pas avoir lieu simultanément, en définissant deux
zones mutuellement exclusives. Mais les outils ne permettent
généralement pas de mentionner une position dans le code.

Spécifier le problème en utilisant LTL ne permet donc pas de se libérer
d'une instrumentation manuelle. Il est nécessaire d'introduire des
variables supplémentaires pour rendre accessibles les variables locales
ou pour délimiter les zones critiques.

Dans notre exemple, on pourrait alors rajouter deux variables booléennes
\texttt{c1} et \texttt{c2} pour représenter les propositions atomiques
``être dans une zone critique''. La spécification serait alors
la formule \(G \lnot (c1 \land c2)\), pour le code instrumenté suivant.

\noindent\begin{minipage}{.45\textwidth}
\begin{lstlisting}[language=C, frame=single, caption=Thread 1]
int c1;
int c2;
void* battery1(void* d) {
  int energy = 100;
  while(1) {
    // Empty the battery
    for(; energy>0; energy--);
    // Replace the battery
    c1 = 1;
    energy = 100;
    c1 = 0;
  }
  pthread_exit(NULL);
}
\end{lstlisting}
\end{minipage}\hfill
\begin{minipage}{.45\textwidth}
\begin{lstlisting}[language=C, frame=single, caption=Thread 1]


void* battery2(void* d) {
  int energy = 100;
  while(1) {
    // Empty the battery
    for(; energy>0; energy--);
    // Replace the battery
    c2 = 1;
    energy = 100;
    c2 = 0;
  }
  pthread_exit(NULL);
}
\end{lstlisting}
\end{minipage}

Cependant, utiliser des variables indicatrices revient à introduire une
instrumentation manuelle dans le code. Cela peut introduire des
différences entre le modèle et le système initial, et donc provoquer des
faux positifs ou des faux négatifs lors de la vérification.

Cet exemple nous a permis de mettre en évidence certaines limites des mécanismes
de spécification actuellement utilisé par les outils de model-checking logiciel
actuels. Les assertions ne sont pas adaptées pour vérifier des propriétés
faisant intervenir plusieurs threads ou des notions temporelles. Peut d'outils
supportent les logiques temporelles, et lorsque c'est le cas, des modifications
des modèle sont nécessaire en raison des restrictions imposées sur la
définitions des propositions atomiques.

Dans cette partie, nous présentons un nouveau formalisme de spécification, basé
sur LTL. Notre objectif est de permettre de spécifier d'avantage de propriétés
sans qu'il soit nécessaire d'instrumenter manuellement le programme à vérifier.
Pour ce faire, nous nous concentrons sur le support des variables locales et la
possibilité d'utiliser des positions du code dans la spécification.
}{}
\end{document}
