\Chapter{LTL avec support des positions et des variables locales}\label{sec:Theme1}

Considérons le problème suivant : un système contient deux batteries. Les
batteries se vident et chacune est changée lorsqu'elle est épuisée. Tant qu'une
des batteries est présente, le système fonctionne. Cependant, si les deux
batteries sont vides en même temps, le système n'a plus d'énergie et cesse de
fonctionner.

Un tel système peut être modélisé par le code ci-dessous. Deux threads,
\lstinline!battery1! et \lstinline!battery2! contrôlent le rythme auquel les
batteries se vident. Le pourcentage d'énergie dans la batterie est représenté
par la variable \lstinline!energy_stored!. Celle-ci est décrémentée dans une
boucle, ce qui représente la consommation de l'énergie, avant d'être remise à
une valeur de 100, ce qui simule le remplacement. Dans ce système simpliste,
aucune synchronisation n'existe entre les threads. L'état d'erreur peut donc
évidement avoir lieu. Nous allons tout de même essayer de détecter cette erreur
à l'aide d'un model-checker logiciel.

La propriété que nous souhaitons vérifier est que, quelque soit
l'exécution, les deux batteries ne seront jamais simultanément vide : il
s'agit donc d'un problème d'exclusion mutuelle. Afin d'exécuter la
vérification, il va tout d'abord être nécessaire de spécifier cette
propriété.

\begin{lstlisting}[language=C]
void* battery1(void* d) {
    int energy_stored = 100;
    while(1) {
      // Empty the battery
      for(; energy_stored > 0; energy_stored--);
      // Replace the battery
      energy_stored = 100;
    }
    pthread_exit(NULL);
}

void* battery2(void* d) {
    int energy_stored = 100;
    while(1) {
      // Empty the battery
      for(; energy_stored > 0; energy_stored--);
      // Replace the battery
      energy_stored = 100;
    }
    pthread_exit(NULL);
}
\end{lstlisting}

Comme nous l'avons vu dans l'état de l'art, la principale limitation du
model-checking est actuellement l'explosion combinatoire. Les outils de
model-checking pour des logiciels concurrents restreignent le type de
propriétés qu'ils vérifient pour optimiser leurs performances. Ainsi, la
plupart de ces outils ne permettent de vérifier que des propriétés
exprimées à l'aide d'assertions.

Dans un premier temps, nous allons donc spécifier notre propriété par
des assertions.

Cependant, sans modification du code, des assertions ne permettent pas
de spécifier une exclusion mutuelle. Nous allons donc rajouter une
variable indicatrice, qui va signaler lorsque un pointeur d'exécution
est dans une des zones critiques (c'est à dire lorsque la batterie est
en remplacement).

On obtient alors le code suivant :

\begin{lstlisting}[language=C]
int critical = 0;

void* battery1(void* d) {
    int energy_stored = 100;
    while(1) {
      // Empty the battery
      for(; energy_stored > 0; energy_stored--);
      // Replace the battery
      assert(critical != 1);
      critical = 1;
      energy_stored = 100;
      critical = 0;
    }
    pthread_exit(NULL);
}

void* battery2(void* d) {
    int energy_stored = 100;
    while(1) {
      // Empty the battery
      for(; energy_stored > 0; energy_stored--);
      // Replace the battery
      assert(critical != 1);
      critical = 1;
      energy_stored = 100;
      critical = 0;
    }
    pthread_exit(NULL);
}
\end{lstlisting}

% Cette instrumentation souffre cependant d'un problème : les instructions
% supplémentaires créent de nouveaux chemins d'exécutions. Ces chemins
% n'existent pas dans le programme d'origine et pourraient permettre
% d'atteindre des états auparavant inaccessibles. Ce n'est heureusement
% pas le cas ici. Nous reviendrons plus tard {[}TODO: indiquer où{]} sur
% ce problème.

L'utilisation de ces variables indicatrices définit des états (dans et
hors de la zone critique pour chaque thread), qui peuvent rappeler la
structure d'un automate de Büchi associé à la propriété ``l'exclusion
mutuelle est toujours respectée''.

D'autres exemples --- tels que ``une \emph{assertion} doit être atteinte
avant une autre pour qu'une erreur soit présente'' --- nécessitent aussi
d'instrumenter le code à l'aide de variables indicatrices. De manière
similaire, ces variables définissent des états et une évolution entre
ces états qui rappelle la construction d'un automate de Büchi.

Quitte à construire cet automate de Büchi, pourquoi ne pas spécifier nos
propriétés en utilisant directement LTL ? Les propriétés LTL sont fréquemment
représentée par des automates de Büchi par les algorithmes de model-checking, on
pourrait ainsi déléguer la construction de l'automate au model-checker.
On réduit ainsi le risque d'introduire des erreurs dans la construction de cet
automate et son intégration dans le modèle, tout en profitant de la syntaxe plus
simple de LTL.

Cependant, les outils de model-checking logiciel supportant le C restreignent
généralement les propositions atomiques des formules LTL qu'ils vérifient à des
expressions pures (sans effet de bord) du C, ne faisant référence qu'à des
variables globales. On ne peut donc pas mentionner directement une variable
locale ou une position du programme dans ces spécifications.

Tentons donc de spécifier notre modèle à l'aide d'une formule LTL. Deux
options sont possibles. La première consiste à vérifier que les deux
batteries ne sont pas vide en même temps, c'est à dire que les variables
\lstinline!energy_stored! des threads \lstinline!battery1! et
\lstinline!battery2! ne valent pas \lstinline!0! simultanément. Mais les
variables \lstinline!energy_stored! sont des variables locales : la
plupart des outils ne permettent pas de les mentionner dans une formule
LTL. On peut aussi considérer que les opérations de changement de
batterie ne doivent pas avoir lieu simultanément, en définissant deux
zones mutuellement exclusives. Mais les outils ne permettent
généralement pas de mentionner une position dans le code.

Spécifier le problème en utilisant LTL ne permet donc pas de se libérer
d'une instrumentation manuelle. Il est nécessaire d'introduire des
variables supplémentaires pour rendre accessibles les variables locales
ou pour délimiter les zones critiques.

Dans notre exemple, on pourrait alors rajouter deux variables booléennes
\lstinline!c1! et \lstinline!c2! pour représenter les propositions atomiques
``être dans une zone critique''. La spécification serait alors
la formule \(G \lnot (c1 \land c2)\), pour le code instrumenté suivant.

\begin{lstlisting}[language=C]
int c1;
int c2;

void* battery1(void* d) {
    int energy_stored = 100;
    while(1) {
        // Empty the battery
        for(; energy_stored > 0; energy_stored--);

        // Replace the battery
        c1 = 1;
        energy_stored = 100;
        c1 = 0;
    }
    pthread_exit(NULL);
}

void* battery2(void* d) {
    int energy_stored = 100;
    while(1) {
        // Empty the battery
        for(; energy_stored > 0; energy_stored--);
        // Replace the battery
        c2 = 1;
        energy_stored = 100;
        c2 = 0;
    }
    pthread_exit(NULL);
}
\end{lstlisting}

Cependant, utiliser des variables indicatrices revient à introduire une
instrumentation manuelle dans le code. Cela peut introduire des
différences entre le modèle et le système initial, et donc provoquer des
faux positifs ou des faux négatifs lors de la vérification.

Dans cette partie, nous présentons un nouveau formalisme de spécification, basé
sur LTL. Notre objectif est de permettre de spécifier d'avantage de propriétés
sans qu'il soit nécessaire d'instrumenter manuellement le programme à vérifier.
Pour ce faire, nous nous concentrons sur le support des variables locales et la
possibilité d'utiliser des positions du code dans la spécification.