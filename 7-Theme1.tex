\Chapter{LTL avec support des positions et des variables locales}\label{sec:Theme1}

\section{Présentation du problème}

Considérons le problème suivant : un système contient deux batteries. Les
batteries se vident et chacune est changée lorsqu'elle est épuisée. Tant qu'une
des batteries est présente, le système fonctionne. Cependant, si les deux
batteries sont vides en même temps, le système n'a plus d'énergie et cesse de
fonctionner.

Un tel système peut être modélisé par le code ci-dessous. Deux threads,
\lstinline!battery1! et \lstinline!battery2! contrôlent le rythme auquel les
batteries se vident. Le pourcentage d'énergie dans la batterie est représenté
par la variable \lstinline!energy_stored!. Celle-ci est décrémentée dans une
boucle, ce qui représente la consommation de l'énergie, avant d'être remise à
une valeur de 100, ce qui simule le remplacement. Dans ce système simpliste,
aucune synchronisation n'existe entre les threads. L'état d'erreur peut donc
évidement avoir lieu. Nous allons tout de même essayer de détecter cette erreur
à l'aide d'un model-checker logiciel.

La propriété que nous souhaitons vérifier est que, quelque soit
l'exécution, les deux batteries ne seront jamais simultanément vide : il
s'agit donc d'un problème d'exclusion mutuelle. Afin d'exécuter la
vérification, il va tout d'abord être nécessaire de spécifier cette
propriété.

\begin{lstlisting}[language=C]
void* battery1(void* d) {
    int energy_stored = 100;
    while(1) {
      // Empty the battery
      for(; energy_stored > 0; energy_stored--);
      // Replace the battery
      energy_stored = 100;
    }
    pthread_exit(NULL);
}

void* battery2(void* d) {
    int energy_stored = 100;
    while(1) {
      // Empty the battery
      for(; energy_stored > 0; energy_stored--);
      // Replace the battery
      energy_stored = 100;
    }
    pthread_exit(NULL);
}
\end{lstlisting}

Comme nous l'avons vu dans l'état de l'art, la principale limitation du
model-checking est actuellement l'explosion combinatoire. Les outils de
model-checking pour des logiciels concurrents restreignent le type de
propriétés qu'ils vérifient pour optimiser leurs performances. Ainsi, la
plupart de ces outils ne permettent de vérifier que des propriétés
exprimées à l'aide d'assertions.

Dans un premier temps, nous allons donc spécifier notre propriété par
des assertions.

Cependant, sans modification du code, des assertions ne permettent pas
de spécifier une exclusion mutuelle. Nous allons donc rajouter une
variable indicatrice, qui va signaler lorsque un pointeur d'exécution
est dans une des zones critiques (c'est à dire lorsque la batterie est
en remplacement).

On obtient alors le code suivant :

\begin{lstlisting}[language=C]
int critical = 0;

void* battery1(void* d) {
    int energy_stored = 100;
    while(1) {
      // Empty the battery
      for(; energy_stored > 0; energy_stored--);
      // Replace the battery
      assert(critical != 1);
      critical = 1;
      energy_stored = 100;
      critical = 0;
    }
    pthread_exit(NULL);
}

void* battery2(void* d) {
    int energy_stored = 100;
    while(1) {
      // Empty the battery
      for(; energy_stored > 0; energy_stored--);
      // Replace the battery
      assert(critical != 1);
      critical = 1;
      energy_stored = 100;
      critical = 0;
    }
    pthread_exit(NULL);
}
\end{lstlisting}

% Cette instrumentation souffre cependant d'un problème : les instructions
% supplémentaires créent de nouveaux chemins d'exécutions. Ces chemins
% n'existent pas dans le programme d'origine et pourraient permettre
% d'atteindre des états auparavant inaccessibles. Ce n'est heureusement
% pas le cas ici. Nous reviendrons plus tard {[}TODO: indiquer où{]} sur
% ce problème.

L'utilisation de ces variables indicatrices définit des états (dans et
hors de la zone critique pour chaque thread), qui peuvent rappeler la
structure d'un automate de Büchi associé à la propriété ``l'exclusion
mutuelle est toujours respectée''.

D'autres exemples --- tels que ``une \emph{assertion} doit être atteinte
avant une autre pour qu'une erreur soit présente'' --- nécessitent aussi
d'instrumenter le code à l'aide de variables indicatrices. De manière
similaire, ces variables définissent des états et une évolution entre
ces états qui rappelle la construction d'un automate de Büchi.

Quitte à construire cet automate de Büchi, pourquoi ne pas spécifier nos
propriétés en utilisant directement LTL ? Les propriétés LTL sont fréquemment
représentée par des automates de Büchi par les algorithmes de model-checking, on
pourrait ainsi déléguer la construction de l'automate au model-checker.
On réduit ainsi le risque d'introduire des erreurs dans la construction de cet
automate et son intégration dans le modèle, tout en profitant de la syntaxe plus
simple de LTL.

Cependant, les outils de model-checking logiciel supportant le C restreignent
généralement les propositions atomiques des formules LTL qu'ils vérifient à des
expressions pures (sans effet de bord) du C, ne faisant référence qu'à des
variables globales. On ne peut donc pas mentionner directement une variable
locale ou une position du programme dans ces spécifications.

Tentons donc de spécifier notre modèle à l'aide d'une formule LTL. Deux
options sont possibles. La première consiste à vérifier que les deux
batteries ne sont pas vide en même temps, c'est à dire que les variables
\lstinline!energy_stored! des threads \lstinline!battery1! et
\lstinline!battery2! ne valent pas \lstinline!0! simultanément. Mais les
variables \lstinline!energy_stored! sont des variables locales : la
plupart des outils ne permettent pas de les mentionner dans une formule
LTL. On peut aussi considérer que les opérations de changement de
batterie ne doivent pas avoir lieu simultanément, en définissant deux
zones mutuellement exclusives. Mais les outils ne permettent
généralement pas de mentionner une position dans le code.

Spécifier le problème en utilisant LTL ne permet donc pas de se libérer
d'une instrumentation manuelle. Il est nécessaire d'introduire des
variables supplémentaires pour rendre accessibles les variables locales
ou pour délimiter les zones critiques.

Dans notre exemple, on pourrait alors rajouter deux variables booléennes
\lstinline!c1! et \lstinline!c2! pour représenter les propositions atomiques
``être dans une zone critique''. La spécification serait alors
la formule \(G \lnot (c1 \land c2)\), pour le code instrumenté suivant.

\begin{lstlisting}[language=C]
int c1;
int c2;

void* battery1(void* d) {
    int energy_stored = 100;
    while(1) {
        // Empty the battery
        for(; energy_stored > 0; energy_stored--);

        // Replace the battery
        c1 = 1;
        energy_stored = 100;
        c1 = 0;
    }
    pthread_exit(NULL);
}

void* battery2(void* d) {
    int energy_stored = 100;
    while(1) {
        // Empty the battery
        for(; energy_stored > 0; energy_stored--);
        // Replace the battery
        c2 = 1;
        energy_stored = 100;
        c2 = 0;
    }
    pthread_exit(NULL);
}
\end{lstlisting}

Cependant, utiliser des variables indicatrices revient à introduire une
instrumentation manuelle dans le code. Cela peut introduire des
différences entre le modèle et le système initial, et donc provoquer des
faux positifs ou des faux négatifs lors de la vérification.

Cet exemple nous a permis de mettre en évidence certaines limites des mécanismes
de spécification actuellement utilisé par les outils de model-checking logiciel
actuels. Les assertions ne sont pas adaptées pour vérifier des propriétés
faisant intervenir plusieurs threads ou des notions temporelles. Peut d'outils
supportent les logiques temporelles, et lorsque c'est le cas, des modifications
des modèle sont nécessaire en raison des restrictions imposées sur la
définitions des propositions atomiques.

Dans cette partie, nous présentons un nouveau formalisme de spécification, basé
sur LTL. Notre objectif est de permettre de spécifier d'avantage de propriétés
sans qu'il soit nécessaire d'instrumenter manuellement le programme à vérifier.
Pour ce faire, nous nous concentrons sur le support des variables locales et la
possibilité d'utiliser des positions du code dans la spécification.

\section{Établir une spécification qui étend LTL}

Les outils de model-checking pour des programmes C multi-thread
supportant LTL que nous avons rencontrés restreignent généralement les
propositions atomiques à une expression pure du C n'utilisant que des
variables globales. À notre connaissance, seul
Divine\cite{Divine_LTL} propose un support plus étendu de LTL,
auquel nous allons comparer ce travail.

Notre objectif est de proposer une spécification basée sur LTL
permettant de surmonter les restrictions classiques faites sur les
propositions atomiques. Nous désirons pouvoir faire référence à des
emplacements du programme afin de pouvoir exprimer des formules
reproduisant le comportement d'une assertion --- et combiner ces
assertions par des opérateurs de la logique temporelle. Toujours dans
l'optique de spécifier un programme sans le modifier, nous désirons
aussi pouvoir faire référence directement à des variables locales de ce
programme.

Le formalisme que nous proposons est une restriction de LTL plus souple
que les restrictions utilisées dans les outils étudiés. Elle se base sur
une définition plus large des propositions atomiques, en ajoutant à ces
dernières une \emph{zone de validité}.

\subsection{Difficultés du problème}

Nous allons tout d'abord présenter les différents problèmes à résoudre
dans le but d'étendre la spécification.

\subsubsection{Désigner les variables}

Afin d'utiliser une variable dans une spécification, il est nécessaire
de la désigner de manière unique.

Le language C n'autorise qu'une seule variable globale portant un nom
donné dans un fichier\footnote{Les mots clefs \lstinline!extern! et
\lstinline!static! permettent de déclarer plusieurs variables globales
portant le même nom dans un programme}. Ce nom est donc généralement
suffisant pour identifier uniquement une variable globale. Ce n'est pas
aussi simple lorsqu'on considère des variables locales : deux variables
locales portant le même nom peuvent cohabiter sans faire référence au
même emplacement mémoire.

Deux variables locales distinctes (ne faisant pas référence à la même
adresse mémoire) mais ayant le même nom peuvent survenir dans les cas
suivants :

\begin{itemize}
\item
  une variable \lstinline!foo! est définie dans plusieurs contextes
  distincts (des fonctions différentes, des blocs différents d'une même
  fonction\ldots{}). La connaissance du contexte de la définition permet
  de les distinguer.
\item
  une fonction \lstinline!bar! contenant une variable \lstinline!foo!
  est appelée plusieurs fois séquentiellement dans le programme. La
  variable \lstinline!foo! dépend alors de l'appel à la fonction
  \lstinline!bar!.
\item
  la variable \lstinline!foo! est définie dans des threads distincts.
  Dans ce cas, les deux instances de la variable peuvent exister
  simultanément. L'instance dépend du thread dans lequel elle est
  définie.
\item
  la variable \lstinline!foo! est définie dans une fonction récursive
  \lstinline!bar!. Dans ce cas, une nouvelle instance de \lstinline!foo!
  est définie à chaque appel récursif de \lstinline!bar!.
\end{itemize}

On peut répartir ces cas en deux catégories :

\begin{itemize}
\item
  les deux variables sont lexicalement distinctes, elles sont issues de
  deux définitions distinctes dans le code source. Dans ce cas, on peut
  différencier les variables en ajoutant des informations
  supplémentaires à son nom, pour identifier le contexte de la
  déclaration de manière unique (fichier, fonction, bloc\ldots{}).
\item
  les deux variables sont lexicalement identiques mais elles sont
  redéfinies. Ce cas est plus complexe : les appels de fonctions et les
  création de thread peuvent être dynamiques et dépendre de l'exécution.
  Ils sont donc difficile à identifier de manière statique, et il en va
  de même pour les variables locales associées.
\end{itemize}

\subsubsection{Définir globalement une proposition atomique dépendant
d'une variable locale}

Dans une formule LTL, une proposition atomique est intrinsèquement
globale : pour tout état \(p\) du programme, il peut être nécessaire de
déterminer si \(p\) est un modèle de la proposition atomique. Cela est
généralement déterminé par la valeur prise par une expression booléenne
dans l'état \(p\). L'expression est évaluée en utilisant la valeur des
variables dans l'état \(p\). Cependant, si cette expression dépend de
variables locales, elle ne peut être évaluée que si toutes ces variables
locales existent dans l'état \(p\). Une proposition atomique n'est
correctement définie que dans le domaine où sont définies les variables
locales dont elle dépend.

Il est donc nécessaire de prolonger la définition des propositions
atomiques à l'ensemble des états.

\subsubsection{Exprimer une position dans le programme}

Un des objectifs de notre spécification est de permettre de manipuler
des positions dans le programme. Il est donc nécessaire de proposer un
système permettant de les désigner. La difficulté est de proposer un
moyen robuste aux modifications du code et suffisamment précis.

TODO : Parler du fait que on s'arrête à la précision du statement. En
parler aussi pour les entrelacement

Plusieurs options viennent rapidement à l'esprit :

\begin{itemize}
\item
  indiquer la position à l'aide de la ligne et du fichier ;
\item
  placer des marqueurs dans le code ;
\end{itemize}

La première méthode souffre d'un inconvénient majeur : la spécification
n'est plus valide dès qu'une ligne est ajoutée ou enlevée dans le code.
Nous avons donc retenu la seconde option. Elle impose cependant de
rajouter une instrumentation dans le code, ce que nous désirions
minimiser.

\subsubsection{Utilisation des positions dans la formule}

La manière dont les positions du code peuvent être utilisées dans la
formule et dans les propositions atomiques en particulier doit aussi
être définie. Le programme a atteint une position dans le code lorsque
un des pointeurs d'exécution pointe sur la position désignée.

Il est cependant nécessaire de définir comment et dans quelle mesure une
position peut intervenir dans une proposition atomique. Nous avons
considéré les options suivantes :

\begin{enumerate}
\def\labelenumi{\arabic{enumi})}
\item
  une proposition atomique désigne une location ou une condition sur les
  variables du programme, de manière exclusive. On fait le lien entre les
  positions et les valeurs des variables du programme à l'aide des opérateurs de
  la logique des prédicats. Si \lstinline!pos1! et \lstinline!pos2! désignent
  deux positions du programme, alors on spécifie que la variable \lstinline!x!
  est non-nulle entre \lstinline!pos1! et \lstinline!pos2! par \(G
  (\{\text{pos1}\}\implies \{x \neq 0\} U \{\text{pos1}\})\).
\item
  une proposition atomique est constituée d'une condition sur les variables du
  programme et d'une condition sur les positions. En reprenant l'exemple
  précédent, on écrirait alors \(G \lnot p\), avec \(p\) la proposition atomique
  qui est vérifiée par tous les états ayant une position entre \lstinline!pos1!
  et \lstinline!pos2! et où \lstinline!x! est nulle.
\item
  les positions viennent contraindre les opérateurs temporels. On pourrait
  envisager une approche semblable à la logique MTL\cite{mtl_definition}, où les
  opérateurs temporels sont restreints à des intervalles de temps, mais en
  appliquant la restriction à des positions du programme. On écrirait alors,
  \(G_{[pos1, pos2]} \lnot \{ x \neq 0 \}\).
\end{enumerate}

Nous avons décidé de ne pas explorer l'option 3. En effet, elle demande
de modifier le comportement standard des opérateurs LTL et nous avons
préféré limiter nos modifications aux propositions atomiques seulement.
