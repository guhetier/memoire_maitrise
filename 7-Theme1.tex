\Chapter{LTL avec support des positions et des variables locales}\label{sec:Theme1}

\section{Présentation du problème}

À travers la revue de littérature, nous avons dégagés deux principales
limitations concernant les méthodes de spécification utilisées pour
model-checking de programmes concurrents.

La première limitation concerne les assertions. Les assertions sont le mécanisme
de spécification le plus utilisé dans le cadre du model-checking logiciel.
Cependant, les propriétés que l'on peut spécifier par des assertions sont
limitées à un sous-ensemble des propriétés de sûretés --- une assertion spécifie
qu'une certaine condition doit être vrai à une certaine position du code. Dans
le cas de programmes concurrents, ce type de propriétés est souvent insuffisant.

En effet, l'entrelacement entre les instruction d'un programme concurrent est
généralement complexe. Il est donc fréquent de vouloir vérifier des propriétés
portant sur l'ordre d'apparition de certains évènements au cours d'une exécution
du programme. Les assertions ne sont pas adaptées pour spécifier ce type de
propriétés.

\paragraph{Exemple}
L'absence de data-races dans un programme est une propriété
généralement désirable. Elle fait partie de l'ensemble plus vaste des propriétés
d'exclusion mutuelle : il s'agit de vérifier que deux instructions accédant à
une même variable partagée ne peuvent être exécutée simultanément. Les listings
\ref{lst:data-race_t1} et \ref{lst:data-race_t2} présentent un cas très simple
de data-race.

\noindent\begin{minipage}{.45\textwidth}
  \begin{lstlisting}[language=C, frame=single, caption=Thread 1,
    label=lst:data-race_t1]
int p = 0;
void* thread1(void* d) {
  ...
  p += 1;
  ...
}
\end{lstlisting}
\end{minipage}\hfill
\begin{minipage}{.45\textwidth}
\begin{lstlisting}[language=C, frame=single, caption=Thread 2,
    label=lst:data-race_t2]

void* thread2(void* d) {
  ...
  p += 1;
  ...
}
\end{lstlisting}
\end{minipage}

Si les instructions \texttt{p += 1} des deux threads ont lieux simultanément, la
valeur finale de \texttt{p} est indéterminée.

Spécifier l'absence de data-race dans ce programme à l'aide d'assertion n'est
pas immédiat : les assertions ne permettent pas d'exprimer la notion de
simultanéité. Il est nécessaire d'introduire des variables supplémentaires, et
donc de modifier le système. La spécification la plus simple que nous avons pu
produire est présentée dans les listings \ref{lst:data-race_assert1} et
\ref{lst:data-race_assert2}\footnote{On remarque que cette spécification permet
  d'atteindre une race condition sans violer les assertions. Ce n'est pas un
problème ici car si il est possible d'atteindre une race-condition, alors au
moins un entrelacement viole les assertions, un model-checker signalera donc
correctement l'erreur.}.

\noindent\begin{minipage}{.45\textwidth}
  \begin{lstlisting}[language=C, frame=single, caption=Thread 1,
    label=lst:data-race_assert1]
int p = 0;
int flag = 0;
void* thread1(void* d) {
  ...
  assert(!flag);
  flag = 1;
  p += 1;
  flag = 0;
  ...
}
\end{lstlisting}
\end{minipage}\hfill
\begin{minipage}{.45\textwidth}
\begin{lstlisting}[language=C, frame=single, caption=Thread 2,
    label=lst:data-race_assert2]


void* thread2(void* d) {
  ...
  assert(!flag);
  flag = 1;
  p += 1;
  flag = 0;
  ...
}
\end{lstlisting}
\end{minipage}

Cet exemple souligne le fait que, dans un programme concurrent, on
s'intéresse fréquemment au lien entre plusieurs \emph{assertions} plutôt qu'au
déclenchement d'une assertion simple. On aimerait pouvoir formuler des
propriétés telles que ``si deux assertions sont violée simultanément, une
erreur est présente'' ou ``si une assertion est violée, puis une seconde
assertion est violée, une erreur est présente''.

La seconde limitation concerne les spécification utilisant \ac{LTL}.
Les model-checker logiciels supportant LTL restreignent en généralement les
propositions atomiques à des fonctions booléenne portant sur la valeur des
variables globales du programme. Il n'est donc pas possible de différencier des
états selon leur position dans le code ou la valeur de variables locales, ce qui
limite le type de propriétés qu'il est possible de spécifier.

En particulier, ces restrictions rendent impossible d'exprimer une assertion par
une formule \ac{LTL}. Alors que les logiques temporelles permettent d'exprimer
facilement des relation d'ordre entre différents évènements, \ac{LTL} n'est donc
pas une solution aux limitations rencontrés par les assertions.

\paragraph{Exemple}
Reprenons l'exemple précédent. La propriété d'exclusion mutuelle que l'on veut
spécifier peut s'écrire \(G \lnot (p_1 \land p_2)\) où \(p_1\) (respectivement
\(p_2\)) est la proposition atomique indiquant si le thread 1 (respectivement le
thread 2) est dans sa zone critique. Mais on ne exprimer \(p_1\) et \(p_2\)
directement, sans mentionner des positions du programme. Encore une fois, il est
nécessaire d'introduire des variables supplémentaires et de modifier le système.
Les listings \ref{lst:data-race_ltl1} et \ref{lst:data-race_ltl2} présentent une
solution, associée avec la spécification \(G \lnot (flag1 == 1 \land flag_2 ==
1)\).

\noindent\begin{minipage}{.45\textwidth}
  \begin{lstlisting}[language=C, frame=single, caption=Thread 1,
    label=lst:data-race_ltl1]
int p = 0;
int flag1 = 0;
void* thread1(void* d) {
  ...
  assert(!flag);
  flag1 = 1;
  p += 1;
  flag1 = 0;
  ...
}
\end{lstlisting}
\end{minipage}\hfill
\begin{minipage}{.45\textwidth}
\begin{lstlisting}[language=C, frame=single, caption=Thread 2,
    label=lst:data-race_ltl2]

int flag2 = 0;
void* thread2(void* d) {
  ...
  assert(!flag);
  flag2 = 1;
  p += 1;
  flag2 = 0;
  ...
}
\end{lstlisting}
\end{minipage}

Les deux mécanismes de spécification les plus utilisés par le model-checking
logiciel sont donc limités, sans être complémentaires. Certaines propriétés sont
alors complexes à spécifier. En pratique, le cas des data-races présenté en
exemple est suffisamment pour être implémenté en tant que propriété built-in,
mais ce n'est pas le cas pour d'autres propriétés (d'exclusion mutuelles ou non).

Dans ce chapitre, nous allons présenter un nouveau mécanisme de spécification
permettant de remédier à ce problème. Il se base sur une restriction de \ac{LTL}
plus souple que celle utilisée classiquement par les model-checker logiciel.
Nous introduisons le concept de \emph{zones de validités} afin de permettre
l'utilisation de variables locales et de positions dans les propositions
atomiques. Nous produisons ainsi un formalisme de spécification qui englobe les
assertions et les proposition LTL (telles que classiquement utilisées par les
model-checkers logiciel), tout en permettant de surmonter les limitations que
nous avons présenté.

Nous allons tout d'abord présenter les différents obstacles a résoudre afin
d'étendre \ac{LTL}. Nous présenterons ensuite les solutions (parfois partielles)
que nous apportons à ces problèmes, ainsi que le formalisme de spécification que
nous avons établi. Enfin, nous justifierons nos choix et nous les comparerons
aux autres travaux sur le sujet.

\section{Difficultés du problème}

Nous allons tout d'abord présenter les différents problèmes à résoudre
dans le but d'étendre la spécification.

\subsection{Désigner les variables}

Afin d'utiliser une variable dans une spécification, il est nécessaire
de la désigner de manière unique.

Le language C n'autorise qu'une seule variable globale portant un nom
donné dans un fichier\footnote{Les mots clefs \texttt{extern} et
\texttt{static} permettent de déclarer plusieurs variables globales
portant le même nom dans un programme}. Ce nom est donc généralement
suffisant pour identifier uniquement une variable globale. Ce n'est pas
aussi simple lorsqu'on considère des variables locales : deux variables
locales portant le même nom peuvent cohabiter sans faire référence au
même emplacement mémoire.

Deux variables locales distinctes (ne faisant pas référence à la même
adresse mémoire) mais ayant le même nom peuvent survenir dans les cas
suivants :

\begin{itemize}
\item
  une variable \texttt{foo} est définie dans plusieurs contextes
  distincts (des fonctions différentes, des blocs différents d'une même
  fonction\ldots{}). La connaissance du contexte de la définition permet
  de les distinguer.
\item
  une fonction \texttt{bar} contenant une variable \texttt{foo}
  est appelée plusieurs fois séquentiellement dans le programme. La
  variable \texttt{foo} dépend alors de l'appel à la fonction
  \texttt{bar}.
\item
  la variable \texttt{foo} est définie dans des threads distincts.
  Dans ce cas, les deux instances de la variable peuvent exister
  simultanément. L'instance dépend du thread dans lequel elle est
  définie.
\item
  la variable \texttt{foo} est définie dans une fonction récursive
  \texttt{bar}. Dans ce cas, une nouvelle instance de \texttt{foo}
  est définie à chaque appel récursif de \texttt{bar}.
\end{itemize}

On peut répartir ces cas en deux catégories :

\begin{itemize}
\item
  les deux variables sont lexicalement distinctes, elles sont issues de
  deux définitions distinctes dans le code source. Dans ce cas, on peut
  différencier les variables en ajoutant des informations
  supplémentaires à son nom, pour identifier le contexte de la
  déclaration de manière unique (fichier, fonction, bloc\ldots{}).
\item
  les deux variables sont lexicalement identiques mais elles sont
  redéfinies. Ce cas est plus complexe : les appels de fonctions et les
  création de thread peuvent être dynamiques et dépendre de l'exécution.
  Ils sont donc difficile à identifier de manière statique, et il en va
  de même pour les variables locales associées.
\end{itemize}

\subsection{Définir globalement une proposition atomique dépendant
d'une variable locale}

Dans une formule LTL, une proposition atomique est intrinsèquement
globale : pour tout état \(p\) du programme, il peut être nécessaire de
déterminer si \(p\) est un modèle de la proposition atomique. Cela est
généralement déterminé par la valeur prise par une expression booléenne
dans l'état \(p\). L'expression est évaluée en utilisant la valeur des
variables dans l'état \(p\). Cependant, si cette expression dépend de
variables locales, elle ne peut être évaluée que si toutes ces variables
locales existent dans l'état \(p\). Une proposition atomique n'est
correctement définie que dans le domaine où sont définies les variables
locales dont elle dépend.

Il est donc nécessaire de prolonger la définition des propositions
atomiques à l'ensemble des états.

\subsection{Exprimer une position dans le programme}

Un des objectifs de notre spécification est de permettre de manipuler
des positions dans le programme. Il est donc nécessaire de proposer un
système permettant de les désigner. La difficulté est de proposer un
moyen robuste aux modifications du code et suffisamment précis.

TODO : Parler du fait que on s'arrête à la précision du statement. En
parler aussi pour les entrelacement

Plusieurs options viennent rapidement à l'esprit :

\begin{itemize}
\item
  indiquer la position à l'aide de la ligne et du fichier ;
\item
  placer des marqueurs dans le code ;
\end{itemize}

La première méthode souffre d'un inconvénient majeur : la spécification
n'est plus valide dès qu'une ligne est ajoutée ou enlevée dans le code.
Nous avons donc retenu la seconde option. Elle impose cependant de
rajouter une instrumentation dans le code, ce que nous désirions
minimiser.

\subsection{Utilisation des positions dans la formule}

La manière dont les positions du code peuvent être utilisées dans la
formule et dans les propositions atomiques en particulier doit aussi
être définie. Le programme a atteint une position dans le code lorsque
un des pointeurs d'exécution pointe sur la position désignée.

Il est cependant nécessaire de définir comment et dans quelle mesure une
position peut intervenir dans une proposition atomique. Nous avons
considéré les options suivantes :

\begin{enumerate}
\def\labelenumi{\arabic{enumi})}
\item
  une proposition atomique désigne une location ou une condition sur les
  variables du programme, de manière exclusive. On fait le lien entre les
  positions et les valeurs des variables du programme à l'aide des opérateurs de
  la logique des prédicats. Si \texttt{pos1} et \texttt{pos2} désignent
  deux positions du programme, alors on spécifie que la variable \texttt{x}
  est non-nulle entre \texttt{pos1} et \texttt{pos2} par \(G
  (\{\text{pos1}\}\implies \{x \neq 0\} U \{\text{pos1}\})\).
\item
  une proposition atomique est constituée d'une condition sur les variables du
  programme et d'une condition sur les positions. En reprenant l'exemple
  précédent, on écrirait alors \(G \lnot p\), avec \(p\) la proposition atomique
  qui est vérifiée par tous les états ayant une position entre \texttt{pos1}
  et \texttt{pos2} et où \texttt{x} est nulle.
\item
  les positions viennent contraindre les opérateurs temporels. On pourrait
  envisager une approche semblable à la logique MTL\cite{mtl_definition}, où les
  opérateurs temporels sont restreints à des intervalles de temps, mais en
  appliquant la restriction à des positions du programme. On écrirait alors,
  \(G_{[pos1, pos2]} \lnot \{ x \neq 0 \}\).
\end{enumerate}

Nous avons décidé de ne pas explorer l'option 3. En effet, elle demande
de modifier le comportement standard des opérateurs LTL et nous avons
préféré limiter nos modifications aux propositions atomiques seulement.

\section{Syntaxe et sémantique de la spécification}

\subsection{Syntaxe}

Nous allons tout d'abord présenter la grammaire de la spécification que
nous avons finalement retenue, avant de justifier nos choix.

Notre spécification est composés de partie. D'une part, une formule LTL
classique. D'autre part, la définition des propositions atomiques utilisées
par la formule LTL.

La définition d'une proposition atomique est composée des éléments suivants :

\begin{itemize}
\item
  \textbf{un nom} : il permet de référer facilement à la proposition atomique.
\item
  \textbf{une zone de validité} : elle délimite un bloc d'instructions dans
  le code (au sens du langage C). Une zone de validité est délimitée par deux
  labels (placés dans le code par l'ingénieur réalisant la spécification). La
  proposition atomique est dans sa zone de validité à partir de l'instant où un
  pointeur d'instruction du programme pointe sur l'instruction portant le label
  de début et jusqu'à ce qu'il pointe sur l'instruction portant le label de fin.
  L'entrée et la sortie d'une zone de validité doivent être dans le même
  contexte et tout pointeur d'exécution atteignant le label de début doit
  atteindre le label de fin avant de sortir du contexte (i.e un branchement ne
  doit pas permettre de sortir d'une zone de validité en évitant l'instruction
  portant le label de fin).
\item
  \textbf{une fonction d'évaluation} : il s'agit d'une fonction booléenne pure,
  écrite en C. Cette fonction est utilisée pour déterminer si un état
  dans la zone de validité vérifie la proposition atomique ou non.
\item
\textbf{une liste de paramètres} : ils désignent les variables qui seront
  passées en argument de la fonction d'évaluation lorsque celle-ci est évaluée.
  Une variable locale est préfixée par le nom de la fonction dans laquelle elle
  est définie. Tous les paramètres doivent être définis dans la zone de validité
  de la proposition atomique.
\item
  \textbf{une valeur par défaut} : lorsque la proposition atomique n'est pas dans
  sa zone de validité, elle s'évalue à sa valeur par défaut. Cette valeur par
  défaut permet de prolonger la définition de la proposition atomique en dehors
  de sa zone de validité.
\end{itemize}

Le tableau \ref{tab:spe_gram} résume la grammaire de notre spécification.

\begin{table}[h]
\centering
\caption{Grammaire des propositions atomiques}
\label{tab:spe_gram}
\begin{tabular}[]{@{}rcl@{}}
\hline
atomic-proposition  & := & proposition-id \(\times\) evaluation-function \(\times\) parameters\\
                    &    & \(\times\) default \(\times\) validity-area\\
proposition-id      & := & \emph{name of the proposition}\\
evaluation-function & := & \emph{C pure boolean function}\\
parameters          & := & parameter list\\
parameter           & := & global-parameter\\
                    & \textbar{} & local-parameter\\
global-parameter    & := & \emph{variable name}\\
local-parameter     & := & \emph{function name} \(\times\) \emph{variable name}\\
default             & := & \emph{boolean}\\
validity-area       & := & label \(\times\) label\\
label               & := & \emph{name of a C label}\\
\hline
\end{tabular}
\end{table}

La spécification est alors constituée d'une formule LTL portant sur des
propositions atomiques définies selon cette grammaire.

\subsection{Sémantique}

Les opérateurs de la logique classique et de la logique temporelle
suivent la sémantique usuelle de la logique propositionnelle et de LTL.

Les propositions atomiques s'évaluent selon la valeur des pointeurs
d'instruction du programme. On considère qu'un état \(p\) est dans la zone de
validité d'une proposition atomique lorsque, dans l'état \(p\), un pointeur
d'instruction du programme pointe sur une instruction comprise entre le label de
début de la zone de validité de la proposition et le label de fin. Cela signifie
que, quelque soit l'exécution menant à l'état \(p\), le pointeur d'exécution a
déjà atteint l'instruction désignée par le label de début, mais pas encore celle
désignée par le label de fin (pour que la spécification soit valide, il est
nécessaire que quelque soit l'exécution, si un pointeur d'instruction désigne
l'instruction marquée par un label de début, alors il atteigne celle marquée par
le label de fin avant de sortir du contexte courant).

On évalue alors la proposition atomique en un état \(p\) par :

\begin{itemize}
\item
  si l'état \(p\) n'est pas dans la zone de validité de la proposition
  atomique, alors la proposition atomique s'évalue à sa valeur par
  défaut.
\item
  si l'état \(p\) est dans la zone de validité de la proposition
  atomique, alors elle s'évalue à la valeur renvoyée par l'évaluation de
  sa fonction d'évaluation, avec comme paramètres la liste de variable
  de la proposition atomiques, dans l'ordre.
\end{itemize}

\subsection{Exemple}\label{exemple}

Nous avons choisi par la suite de représenter cette spécification en
JSON (JavaScript Object Notation)\cite{json}. Ce format est
relativement verbeux, mais présente l'avantage d'être facile à parser
tout en étant lisible pour l'être humain. Établir une syntaxe plus
concise pourrait représenter un développement futur.

Étant donné le code suivant, constitué de deux threads s'exécutant
simultanément (pour des raisons de concision, la fonction
\texttt{main} a été omise), nous allons spécifier le fait que la
variable \texttt{a} du premier thread n'est jamais nulle en même temps que
la variable \texttt{a} du second thread.

\noindent\begin{minipage}{.45\textwidth}
\begin{lstlisting}[language=C, frame=single, caption=Thread 1]
void* thread1(void* d) {
  int a
b_p1:
  a = 0;
  a = 1;
e_p1:
  pthread_exit(NULL);
}
\end{lstlisting}
\end{minipage}\hfill
\begin{minipage}{.45\textwidth}
\begin{lstlisting}[language=C, frame=single, caption=Thread 2]
void* thread2(void* d) {
  int a;
b_p2:
  a = 1;
  a = a - 1;
e_p2:
  pthread_exit(NULL);
}
\end{lstlisting}
\end{minipage}

Il est nécessaire de rajouter des fonctions d'évaluation pour les
propositions atomiques :

\begin{lstlisting}[language=C]
int f_ev_1(int x) {
    return x == 0;
}

int fp2(int x) {
    return x == 0;
}
\end{lstlisting}

Nous avons donc la spécification suivante :

\begin{lstlisting}
{
  "ltl": "G(! (p1 && p2))",
  "pa": [
    {
      "name": "p1",
      "default": false,
      "expr": "f_ev_p1",
      "span": ["b_p1", "e_p1"],
      "params": ["thread1::a"]
    },
    {
      "name": "p2",
      "default": false,
      "expr": "f_ev_p2",
      "span": ["b_p2", "e_p2"],
      "params": ["thread2::a"]
    }
  ]
}
\end{lstlisting}

Ici, le code ne respecte pas cette spécification. L'exécution
\(t1: a = 0 \rightarrow t2: a = 1 \rightarrow t2: a = a - 1\) permet
d'obtenir un état qui ne satisfait pas la propriété.

\subsection{Justification de la spécification}

Nous allons maintenant expliquer comment et dans quelle mesure ce formalisme
de spécification permet de surmonter les obstacles détaillées précédemment.

\paragraph{Identification des variables locales}

Nous avons choisi d'identifier une variable globale par son nom et une
variable locale par le nom de la fonction dans laquelle elle est définie
suivie de son nom. Nous n'apportons pas de solution pour le cas où deux
variables ont le même nom dans une fonction : il est laissé à
l'utilisateur le soin de ne pas définir une variable utilisée dans la
spécification portant le même nom qu'une autre variable de la fonction.
En pratique, il est peut fréquent de définir plusieurs variables ayant le
même nom dans une fonction, ce problème n'est donc pas une réelle
limitation.

Divine \cite{Divine_LTL} propose une approche différente pour
identifier les paramètres d'une formule LTL. La spécification de Divine
met en place des macros qui permettent de lier les variables à une
proposition atomique, directement dans le code source. Cela permet en
particulier d'identifier une variable locale même si d'autres variables
du même nom sont définie dans la même fonction.

Divine permet de lier une proposition atomique \(P\) à sa fonction
d'évaluation \texttt{p}, dépendant d'une variable locale
\texttt{y} de la manière suivante :

\begin{lstlisting}
void* thread1(void* d) {
    int y;
    ap_local(P, p, y);
    ....
}
\end{lstlisting}

L'instruction \texttt{ap\_local(P, p, y);} lie la proposition atomique
identifiée par \(P\) dans la formule LTL à la fonction \texttt{p}, qui prend
en paramètre la variable \texttt{p}. La norme du C permet d' identifier la
variable désignée par le nom \texttt{y} dans le contexte d'appel de la macro
\texttt{ap\_local}. Cependant, cette approche introduit d'avantage de
spécification dans le code source, ce que nous avons voulu éviter.

\emph{TODO}: pas encore implémenté.

Lorsque la même fonction est utilisée pour créer plusieurs threads, le
nom de la fonction ne suffit pas à identifier une variable de manière
unique : il est nécessaire d'indiquer le thread auquel on se réfère. La
création de thread est une notion dynamique. Il est donc difficile d'
identifier un thread de manière statique. Cependant, les threads sont
souvent utilisés de manière statique : ils sont tous créés dès le début
de l'exécution par une même fonction. Dans ce cas, il est possible
d'identifier un thread par un indice correspondant à l'ordre de création
des threads. On pourra alors identifier une variable dans un thread
uniquement en rajoutant un préfixe à son nom, qui pendra alors la forme
\texttt{{thread\_id}::{fonction}::{variable}}.

\emph{TODO}: pas encore implémenté.

Dans le cas où plusieurs instances d'une seule variable lexicale sont
crées dans le programme, nous n'apportons pas de solution dans le cas
général. Ce cas arrive lorsque une fonction est appelée plusieurs fois à
travers différents threads, de manière séquentielle ou lors d'appels
récursifs. Dans le cas d'appels séquentiels, on désire la plupart du
temps évaluer la proposition atomique lors de chaque appel, ce n'est
donc pas un problème majeur. Cependant, lors d'appels récursifs ou
concurrents, plusieurs instances de la variables existent simultanément
: il faut alors décider laquelle doit être utilisée pour évaluer la
proposition atomique.

Nous reprenons l'approche utilisée par Divine \cite{Divine_LTL} en
l'adaptant aux valeurs par défaut. Si plusieurs zones de validité pour
une même proposition atomique sont ouverte en même temps, alors on
évaluera la proposition atomique à sa valeur par défaut si et seulement
si elle s'évalue à sa valeur par défaut en utilisant les variables de
chaque zone de validité.

Par exemple, considérons la fonction suivante :

\begin{lstlisting}
int count(int a) {
   begin:
   if (a < 10)
       return count(a + 1);
   end:
   return a;
}

int main() {
    count(6);
    return 0;
}
\end{lstlisting}

Supposons que l'on veuille vérifier que entre \texttt{begin} et
\texttt{end}, on a toujours \lstinline"a != 8". On utilise donc la
formule \(G p\), avec \(p\) la proposition vraie par défaut et
s'évaluant à \(a != 8\) entre \texttt{begin} et \texttt{end}. En
déroulant les trois premiers appels, on obtient trois contextes, où
\texttt{a} vaut respectivement 6, 7 et 8. Pour \(a = 6\) et
\(a = 7\), \(p\) s'évalue à \(vrai\), qui est aussi la valeur par défaut
de \(p\). Pour \(a = 8\), \(p\) s'évalue à \(faux\). On retient donc
cette valeur pour \(p\), ce qui permet de trouver une erreur.

Cette approche fait remonter les cas où la proposition atomique ne prend
pas sa valeur par défaut, ce qui constitue généralement l'évènement
attendu, en particulier pour des propriétés de sûreté ou
d'accessibilité.

\paragraph{Identification des positions dans le code}

Identifier une position dans le code sans y placer un marqueur n'est pas
une solution viable : elle est trop sensible aux modifications du code.
Ces dernières sont de plus attendues lors de la vérification d'un
programme, afin de corriger une éventuelle erreur.

Nous avons donc utilisé des marqueurs. Nous avons choisis des labels,
puisqu'ils permettent de désigner une instruction du code. Une
alternative aurait été de marquer les instructions à l'aide de
commentaires dans le code. Nous avons préféré les labels pour des
raisons d'implémentation, les commentaires n'étant pas conservé par de
nombreuses bibliothèques de transformation de code.

Cependant, la précision d'une position à l'aide d'un marqueur reste de,
au mieux, une ligne, et donc d'une instruction. On ne peut donc pas
désigner une position plus précisément, au niveau d'une expression par
exemple. Nous avons considéré que ce besoin est rarement présent. Il
reste de plus possible de décomposer une instruction afin d'en désigner
ses sous-ensembles.

Nous avons choisi de former des intervalles de positions car ils
représentent un bon compromis entre une énumération explicite et des
constructions plus complexes (union ou intersections
d'intervalles\ldots{}). Dans les cas où une zone de validité plus
complexe est nécessaire, il reste toujours possible de la simuler en
utilisant plusieurs propositions atomiques et les opérateurs de la
logiques classiques.

\paragraph{Définition des propositions atomiques}

Afin de définir correctement les propositions atomiques dans tous les
états du programme, nous avons choisi d'utiliser des valeurs par défaut
combinées à des zones de validité. Hors de la zone de validité (un
intervalle de positions dans le code représenté comme expliqué
précédemment), la proposition atomique prend sa valeur par défaut,
sinon, sa fonction d'évaluation est utilisée.

L'intérêt des zones de validité est double : l'utilisateur peut
spécifier précisément quand une proposition peut utiliser la valeur des
variables locales, ce qui permet en particulier d'attendre qu'elles
soient correctement initialisées. Les zones de validités permettent
aussi d'utiliser des positions du code dans la spécification, et de lier
directement ces positions à une propositions sur les valeurs des
variables du programme. Il n'est donc pas nécessaire d'utiliser deux
propositions atomiques \texttt{b} et \texttt{e} pour définir le
début et la fin d'un intervalle de position, une troisième proposition
atomique \texttt{p} pour spécifier une propriété sur les valeurs des
variables et enfin, d'utiliser une formule telle que
\(G (b \implies (p U e))\) pour spécifier que \texttt{p} doit être
valide entre \texttt{b} et \texttt{e}. \(G p\) est suffisant, la
zone de validité étant incluse dans la proposition atomique
\texttt{p}. On obtient ainsi une formule LTL plus simple.

Nous avons choisi d'utiliser une valeur par défaut que l'utilisateur
peut fixer. Ce n'est pourtant strictement nécessaire : une valeur par
défaut pourrait être imposée (\emph{faux} par exemple), et les autres
cas peuvent être obtenus en utilisant des négations dans la formule
logique et la définition de la fonction d'évaluation. Par exemple, pour
une proposition atomique \(p\) s'évaluant à \(vrai\) par défaut et en
utilisant la fonction \(f\) sinon, et pour la formule \(G p\), on
pourrait utiliser de manière équivalente la proposition atomique \(q\),
avec comme valeur par défaut \(faux\) et comme fonction d'évaluation
\(\lnot f\). La formule LTL serait alors \(G \lnot q\). Divine
\cite{Divine_LTL} utilise ce principe.

Cependant, il nous a semblé plus pratique et confortable de permettre à
l'utilisateur de fixer cette valeur de manière cohérente avec
l'utilisation qui est faite de la proposition atomique. Ainsi, dans le
cas d'une propriété de sûreté \(G p\), il est plus naturel de fixer la
valeur par défaut de \(p\) à \emph{vrai} plutôt que considérer une
double négation.

\subsection{Exprimer des propositions atomiques globales sur des
variables globales}

Les propositions atomiques utilisées usuellement par les outils de model
checking logiciels (des expressions pures portant uniquement sur les
variables globales) peuvent s'exprimer à l'aide de notre spécification
en choisissant le programme entier comme zone de validité.

\subsection{Exprimer des assertions}

Notre spécification permet de spécifier l'équivalent d'une assertion. En
effet, une assertion signifie : ``si un pointeur d'instruction du
programme pointe sur l'assertion et que l'état ne vérifie pas la
condition exprimée, alors il y a une erreur''.

On peut donc représenter cela par une propriété de sûreté. On exprime
une assertion par la formule \(G p\), avec \(p\) une proposition
atomique telle que :

\begin{itemize}
\item
  sa zone de validité correspond à l'emplacement de l'assertion dans le
  programme
\item
  sa fonction d'évaluation est la condition utilisée dans l'assertion,
  elle s'évalue donc à la valeur de cette expression dans la zone de
  validité
\item
  sa valeur par défaut est \emph{vrai}, elle s'évalue donc à \emph{vrai}
  hors de la zone de validité
\end{itemize}

Notre spécification correspond donc bien à une extension recouvrant à la
fois les formules LTL classiquement utilisées et les assertions.